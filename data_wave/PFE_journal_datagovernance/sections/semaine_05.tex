\subsection{Semaine 5 : 17-21 Mars 2025 - Module ScanLogic et Orchestration Intelligente}

\subsubsection{Lundi - Conception architecture ScanLogic}

\textbf{Activités réalisées :}
\begin{itemize}
    \item Développement moteur de scan intelligent
    \item Poursuite du développement des fonctionnalités avancées
    \item Tests et validation des développements
    \item Documentation technique et utilisateur
    \item Collaboration avec l'équipe technique
\end{itemize}

\textbf{Apprentissages :}
Approfondissement des compétences techniques et métier dans le domaine de la gouvernance des données. Développement de l'expertise sur les outils et technologies utilisés.

\textbf{Compétences mobilisées :}
Développement logiciel, architecture système, gestion de projet, collaboration équipe.

\subsubsection{Mardi - Implémentation scan parallélisé}

\textbf{Activités réalisées :}
\begin{itemize}
    \item Optimisation performance scans massifs
    \item Poursuite du développement des fonctionnalités avancées
    \item Tests et validation des développements
    \item Documentation technique et utilisateur
    \item Collaboration avec l'équipe technique
\end{itemize}

\textbf{Apprentissages :}
Approfondissement des compétences techniques et métier dans le domaine de la gouvernance des données. Développement de l'expertise sur les outils et technologies utilisés.

\textbf{Compétences mobilisées :}
Développement logiciel, architecture système, gestion de projet, collaboration équipe.

\subsubsection{Mercredi - Réunion superviseur}

\textbf{Activités réalisées :}
\begin{itemize}
    \item Validation approche technique et planning
    \item Poursuite du développement des fonctionnalités avancées
    \item Tests et validation des développements
    \item Documentation technique et utilisateur
    \item Collaboration avec l'équipe technique
\end{itemize}

\textbf{Apprentissages :}
Approfondissement des compétences techniques et métier dans le domaine de la gouvernance des données. Développement de l'expertise sur les outils et technologies utilisés.

\textbf{Compétences mobilisées :}
Développement logiciel, architecture système, gestion de projet, collaboration équipe.

\subsubsection{Jeudi - Tests performance ScanLogic}

\textbf{Activités réalisées :}
\begin{itemize}
    \item Benchmarking et optimisation
    \item Poursuite du développement des fonctionnalités avancées
    \item Tests et validation des développements
    \item Documentation technique et utilisateur
    \item Collaboration avec l'équipe technique
\end{itemize}

\textbf{Apprentissages :}
Approfondissement des compétences techniques et métier dans le domaine de la gouvernance des données. Développement de l'expertise sur les outils et technologies utilisés.

\textbf{Compétences mobilisées :}
Développement logiciel, architecture système, gestion de projet, collaboration équipe.

\subsubsection{Vendredi - Intégration modules existants}

\textbf{Activités réalisées :}
\begin{itemize}
    \item Tests intégration complète
    \item Poursuite du développement des fonctionnalités avancées
    \item Tests et validation des développements
    \item Documentation technique et utilisateur
    \item Collaboration avec l'équipe technique
\end{itemize}

\textbf{Apprentissages :}
Approfondissement des compétences techniques et métier dans le domaine de la gouvernance des données. Développement de l'expertise sur les outils et technologies utilisés.

\textbf{Compétences mobilisées :}
Développement logiciel, architecture système, gestion de projet, collaboration équipe.

\textbf{Résumé de la semaine :}
Cette semaine 5 a permis de consolider les développements en cours et d'atteindre les objectifs fixés. Les réalisations techniques continuent de démontrer la valeur du projet pour NXCI.

\textbf{Objectifs pour la semaine suivante :}
\begin{itemize}
    \item Poursuivre le développement des fonctionnalités avancées
    \item Maintenir la qualité et les performances du système
    \item Préparer les livrables et démonstrations
\end{itemize}

