\subsection{Semaine 8 : 7-11 Avril 2025 - Module Compliance et Conformité Réglementaire}

\textbf{Note :} Semaine avec jour férié (9 avril 2025)

\subsubsection{Lundi 7 Avril 2025 - Architecture Compliance}

\textbf{Activités réalisées :}
\begin{itemize}
    \item Conception de l'architecture du module Compliance
    \item Analyse des exigences RGPD, SOX, HIPAA et autres réglementations
    \item Développement des modèles de données pour la conformité
    \item Création des services de base dans \texttt{compliance\_service.py}
    \item Définition des règles de conformité automatisées
\end{itemize}

\textbf{Apprentissages :}
La conformité réglementaire en gouvernance des données nécessite une approche multicouche : détection automatique, alertes, rapports et audit trail. J'ai approfondi ma compréhension des exigences RGPD et de leur implémentation technique.

\textbf{Compétences mobilisées :}
Conformité réglementaire, audit automatisé, modélisation de processus métier.

\subsubsection{Mardi 8 Avril 2025 - Détection Automatique Non-Conformité}

\textbf{Activités réalisées :}
\begin{itemize}
    \item Implémentation des algorithmes de détection de non-conformité
    \item Développement des règles RGPD automatisées
    \item Création du système d'alertes en temps réel
    \item Intégration avec le module Classifications pour détecter les données sensibles
    \item Tests sur des datasets de conformité
\end{itemize}

\textbf{Apprentissages :}
La détection automatique de non-conformité nécessite une combinaison de règles métier strictes et d'analyse contextuelle. Les faux positifs doivent être minimisés pour maintenir l'efficacité opérationnelle.

\textbf{Compétences mobilisées :}
Algorithmes de détection, règles métier complexes, systèmes d'alertes.

\textbf{Code développé :}
\begin{lstlisting}[language=Python, caption=Détection RGPD automatisée]
class ComplianceEngine:
    async def check_gdpr_compliance(self, data_asset: DataAsset):
        violations = []
        
        # Vérification présence données personnelles sans consentement
        if await self.has_personal_data(data_asset):
            consent = await self.check_consent_records(data_asset)
            if not consent:
                violations.append(GDPRViolation.NO_CONSENT)
        
        # Vérification durée de rétention
        retention_check = await self.check_retention_policy(data_asset)
        if retention_check.is_expired:
            violations.append(GDPRViolation.RETENTION_EXCEEDED)
        
        return ComplianceResult(violations=violations)
\end{lstlisting}

\subsubsection{Mercredi 9 Avril 2025 - JOUR FÉRIÉ}

\textbf{Jour de congé} - Pas d'activités professionnelles

\subsubsection{Jeudi 10 Avril 2025 - Rapports et Audit Trail}

\textbf{Activités réalisées :}
\begin{itemize}
    \item Développement du système de rapports de conformité
    \item Implémentation de l'audit trail complet
    \item Création des tableaux de bord compliance
    \item Intégration avec les systèmes de notification
    \item Tests des rapports automatisés
\end{itemize}

\textbf{Apprentissages :}
Les rapports de conformité doivent être à la fois détaillés pour les auditeurs et synthétiques pour les managers. J'ai appris l'importance de la traçabilité complète des actions sur les données sensibles.

\textbf{Compétences mobilisées :}
Reporting automatisé, audit trail, visualisation de données de conformité.

\textbf{Métriques obtenues :}
\begin{itemize}
    \item Temps de génération rapport RGPD : 2.3 secondes
    \item Couverture règles de conformité : 98.5\%
    \item Précision détection violations : 96.8\%
    \item Délai d'alerte temps réel : <500ms
\end{itemize}

\subsubsection{Vendredi 11 Avril 2025 - Intégration et Validation}

\textbf{Activités réalisées :}
\begin{itemize}
    \item Intégration complète du module Compliance avec l'écosystème
    \item Tests de charge sur les systèmes de conformité
    \item Validation avec des experts conformité de l'entreprise
    \item Documentation des processus de conformité
    \item Préparation de la démonstration client
\end{itemize}

\textbf{Apprentissages :}
La validation par des experts métier est cruciale pour s'assurer que les règles techniques correspondent aux exigences réglementaires réelles. Certains ajustements ont été nécessaires suite à leurs retours.

\textbf{Compétences mobilisées :}
Validation métier, tests de charge, documentation processus.

\textbf{Résumé de la semaine :}
Malgré le jour férié, cette huitième semaine marque l'achèvement d'un module critique : la conformité réglementaire. Le système développé offre une couverture complète des principales réglementations avec une automatisation poussée.

\textbf{Réalisations techniques majeures :}
\begin{itemize}
    \item Moteur de conformité multi-réglementation (RGPD, SOX, HIPAA)
    \item Détection automatique en temps réel des violations
    \item Système de rapports automatisés pour auditeurs
    \item Audit trail complet de toutes les opérations
\end{itemize}

\textbf{Objectifs pour la semaine suivante :}
\begin{itemize}
    \item Finaliser le module RBAC/Control System
    \item Implémenter la gestion granulaire des permissions
    \item Intégrer l'authentification multi-facteur
\end{itemize}