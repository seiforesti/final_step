\subsection{Semaine 1 : 17-21 Février 2025 - Découverte et Immersion}

\subsubsection{Lundi 17 Février 2025 - Premier Jour}

\textbf{Activités réalisées :}
\begin{itemize}
    \item Accueil officiel chez NXCI aux Berges du Lac 1, Tunis
    \item Présentation de l'équipe technique et du management
    \item Visite des locaux et découverte de l'environnement de travail
    \item Présentation générale du projet de gouvernance des données
    \item Configuration de l'environnement de développement (IDE, accès VPN, comptes)
\end{itemize}

\textbf{Apprentissages :}
J'ai découvert l'ampleur du projet et sa complexité technique. La problématique de Microsoft Purview concernant les limitations d'extraction de schémas est plus critique que prévu. L'entreprise NXCI possède une expertise solide dans les partenariats canado-tunisiens et une approche innovante des solutions data.

\textbf{Compétences mobilisées :}
Capacité d'adaptation, écoute active, compréhension des enjeux business et techniques.

\textbf{Défis rencontrés :}
Compréhension initiale de la complexité architecturale du système à développer. La documentation Microsoft Purview s'avère parfois incomplète pour certains cas d'usage avancés.

\subsubsection{Mardi 18 Février 2025 - Étude Documentaire}

\textbf{Activités réalisées :}
\begin{itemize}
    \item Étude approfondie de la documentation Microsoft Purview
    \item Analyse des APIs Microsoft Azure pour la gouvernance des données
    \item Découverte des fonctionnalités Databricks et de leur intégration possible
    \item Lecture de la documentation technique sur les Integration Runtime
    \item Première compréhension des limites actuelles de l'écosystème Microsoft
\end{itemize}

\textbf{Apprentissages :}
Microsoft Purview présente des limitations significatives :
\begin{itemize}
    \item Support incomplet pour certains types de bases de données (notamment les bases NoSQL complexes)
    \item Classificateurs prédéfinis insuffisants pour les données métier spécifiques
    \item Data lineage limité dans les architectures multi-cloud
    \item Catalogage automatique incomplet pour les sources de données personnalisées
\end{itemize}

\textbf{Compétences mobilisées :}
Analyse technique approfondie, capacité de synthèse, recherche documentaire avancée.

\textbf{Réflexions :}
La nécessité de développer une solution native devient évidente. Les limitations de Microsoft justifient pleinement notre approche innovante.

\subsubsection{Mercredi 19 Février 2025 - Analyse Technique}

\textbf{Activités réalisées :}
\begin{itemize}
    \item Analyse détaillée des APIs Microsoft Graph et Purview
    \item Tests pratiques des fonctionnalités d'extraction de métadonnées
    \item Identification des gaps fonctionnels spécifiques
    \item Documentation des cas d'usage non couverts par les outils Microsoft
    \item Première ébauche de l'architecture solution
\end{itemize}

\textbf{Apprentissages :}
J'ai appris à utiliser les outils de diagnostic Microsoft et à identifier précisément les limitations techniques. La compréhension des Integration Runtime et de leur fonctionnement m'a permis de mieux cerner les défis à relever.

\textbf{Compétences mobilisées :}
Diagnostic technique, analyse d'APIs, compréhension des architectures distribuées.

\textbf{Défis rencontrés :}
La complexité des APIs Microsoft nécessite une courbe d'apprentissage importante. Certaines fonctionnalités sont mal documentées ou présentent des comportements inattendus.

\subsubsection{Jeudi 20 Février 2025 - Réunion d'Équipe}

\textbf{Activités réalisées :}
\begin{itemize}
    \item Présentation de mes analyses à l'équipe technique
    \item Discussion sur les approches possibles pour surmonter les limitations Microsoft
    \item Brainstorming sur l'architecture du système à développer
    \item Définition des objectifs techniques à court et moyen terme
    \item Planification des phases de développement
\end{itemize}

\textbf{Apprentissages :}
La collaboration en équipe chez NXCI suit une méthodologie agile adaptée. J'ai appris l'importance de la communication technique claire et de la documentation des décisions architecturales.

\textbf{Compétences mobilisées :}
Présentation technique, travail collaboratif, argumentation des choix techniques.

\textbf{Réflexions :}
L'équipe a validé mon analyse et mes recommandations. Cette validation renforce ma confiance dans la compréhension du projet et des enjeux techniques.

\subsubsection{Vendredi 21 Février 2025 - Planification}

\textbf{Activités réalisées :}
\begin{itemize}
    \item Définition détaillée du plan de travail pour les 6 mois
    \item Répartition des responsabilités et des modules à développer
    \item Mise en place des outils de gestion de projet (Git, Jira, documentation)
    \item Configuration de l'environnement de développement complet
    \item Première réunion avec l'encadrant pour validation du plan
\end{itemize}

\textbf{Apprentissages :}
La gestion de projet dans le contexte d'un PFE nécessite une planification rigoureuse et des jalons clairs. J'ai appris l'importance de définir des livrables mesurables et des critères de réussite précis.

\textbf{Compétences mobilisées :}
Planification de projet, gestion du temps, définition d'objectifs SMART.

\textbf{Résumé de la semaine :}
Cette première semaine m'a permis de comprendre pleinement les enjeux du projet et les défis techniques à relever. L'immersion chez NXCI s'est parfaitement déroulée, et l'équipe a montré un excellent niveau d'expertise. Les limitations de Microsoft Purview sont maintenant clairement identifiées, ce qui valide l'approche innovante que nous allons développer.

\textbf{Objectifs pour la semaine suivante :}
\begin{itemize}
    \item Commencer le développement de l'architecture backend
    \item Mettre en place les premiers modules (DataSource et DataCatalog)
    \item Développer les premiers connecteurs de bases de données
\end{itemize}