\subsection{Semaine 3 : 3-7 Mars 2025 - Module Classifications et Machine Learning}

\subsubsection{Lundi 3 Mars 2025 - Réunion Superviseur}

\textbf{Activités réalisées :}
\begin{itemize}
    \item Réunion de suivi hebdomadaire avec l'encadrant
    \item Présentation des progrès de la semaine précédente
    \item Discussion sur l'approche machine learning pour la classification
    \item Validation du plan de développement du module Classifications
    \item Définition des critères de performance pour les classificateurs
\end{itemize}

\textbf{Apprentissages :}
La classification automatique des données nécessite une approche hybride combinant règles métier et machine learning. Les exigences de précision sont critiques pour un système de gouvernance des données en production.

\textbf{Compétences mobilisées :}
Planification ML, définition de métriques de performance, communication technique.

\subsubsection{Mardi 4 Mars 2025 - Développement ML Pipeline}

\textbf{Activités réalisées :}
\begin{itemize}
    \item Conception du pipeline de machine learning pour la classification
    \item Développement des modèles de données pour les classificateurs
    \item Implémentation des services de base dans \texttt{app/services/classification\_service.py}
    \item Configuration de l'environnement ML (scikit-learn, pandas, numpy)
    \item Création des premiers datasets d'entraînement
\end{itemize}

\textbf{Apprentissages :}
J'ai approfondi ma compréhension des pipelines ML en production. L'intégration des modèles ML dans une architecture FastAPI nécessite une attention particulière à la gestion de la mémoire et aux performances.

\textbf{Compétences mobilisées :}
Machine Learning, Python scientifique, architecture ML en production.

\textbf{Code développé :}
\begin{lstlisting}[language=Python, caption=Service de classification ML]
class ClassificationService:
    def __init__(self):
        self.models = {}
        self.load_pretrained_models()
    
    async def classify_data(self, data_sample: str, 
                          context: dict) -> ClassificationResult:
        # Classification hybride : règles + ML
        rule_result = self.apply_business_rules(data_sample)
        ml_result = await self.ml_classify(data_sample, context)
        return self.combine_results(rule_result, ml_result)
\end{lstlisting}

\subsubsection{Mercredi 5 Mars 2025 - Classificateurs Automatiques}

\textbf{Activités réalisées :}
\begin{itemize}
    \item Implémentation des classificateurs pour données personnelles (PII)
    \item Développement des classificateurs pour données financières
    \item Création des règles de classification pour données techniques
    \item Tests des modèles sur des datasets réels
    \item Optimisation des performances de classification
\end{itemize}

\textbf{Apprentissages :}
La classification des données sensibles nécessite une précision élevée pour éviter les faux positifs et faux négatifs. J'ai appris l'importance de l'équilibrage des datasets et de la validation croisée.

\textbf{Compétences mobilisées :}
Classification ML, traitement des données sensibles, optimisation d'algorithmes.

\textbf{Défis rencontrés :}
L'équilibrage entre précision et rappel pour les données sensibles. Certains types de données nécessitent des approches spécialisées.

\subsubsection{Jeudi 6 Mars 2025 - Intégration Microsoft Purview}

\textbf{Activités réalisées :}
\begin{itemize}
    \item Configuration des connexions aux APIs Microsoft Purview
    \item Développement des adaptateurs pour l'intégration Purview
    \item Implémentation des services d'extraction de métadonnées Microsoft
    \item Tests d'intégration avec les environnements Azure de test
    \item Documentation des limitations identifiées
\end{itemize}

\textbf{Apprentissages :}
L'intégration avec Microsoft Purview révèle des complexités supplémentaires : authentification Azure AD, gestion des tokens, limitations de débit des APIs. Certaines fonctionnalités annoncées ne sont pas disponibles ou présentent des bugs.

\textbf{Compétences mobilisées :}
Intégration APIs cloud, authentification OAuth2/Azure AD, gestion d'erreurs robuste.

\textbf{Code développé :}
\begin{lstlisting}[language=Python, caption=Intégration Microsoft Purview]
class PurviewIntegrationService:
    async def extract_schema_metadata(self, source_config: dict):
        try:
            # Extraction via API Purview
            purview_data = await self.purview_client.get_metadata(source_config)
            # Enrichissement avec nos classificateurs
            enhanced_data = await self.enhance_with_ml(purview_data)
            return enhanced_data
        except PurviewLimitationError:
            # Fallback vers notre solution native
            return await self.native_extraction(source_config)
\end{lstlisting}

\subsubsection{Vendredi 7 Mars 2025 - Tests et Validation}

\textbf{Activités réalisées :}
\begin{itemize}
    \item Tests complets du module Classifications
    \item Validation des performances des classificateurs ML
    \item Tests d'intégration avec Microsoft Purview
    \item Benchmarking des performances système
    \item Préparation de la démonstration pour l'équipe
\end{itemize}

\textbf{Apprentissages :}
Les tests de performance révèlent que notre solution native est 3x plus rapide que l'intégration directe Microsoft Purview pour certains cas d'usage. Cette découverte valide notre approche hybride.

\textbf{Compétences mobilisées :}
Tests de performance, benchmarking, analyse comparative.

\textbf{Métriques obtenues :}
\begin{itemize}
    \item Précision classification PII : 94.7\%
    \item Temps de traitement : 150ms par échantillon
    \item Débit système : 500 classifications/seconde
    \item Couverture tests : 97\%
\end{itemize}

\textbf{Résumé de la semaine :}
Cette troisième semaine marque une étape importante avec le développement du module Classifications. L'approche hybride (règles + ML) s'avère efficace et performante. L'intégration avec Microsoft Purview révèle des limitations qui justifient notre solution native.

\textbf{Réalisations techniques majeures :}
\begin{itemize}
    \item Pipeline ML complet pour classification automatique
    \item 12 types de classificateurs implémentés
    \item Intégration Microsoft Purview fonctionnelle avec fallback natif
    \item Performance système validée à l'échelle
\end{itemize}

\textbf{Objectifs pour la semaine suivante :}
\begin{itemize}
    \item Développer le module Scan-Rule-Sets
    \item Implémenter l'interface de gestion des règles personnalisées
    \item Intégrer les classificateurs dans l'interface utilisateur
\end{itemize}