\subsection{Semaine 4 : 10-14 Mars 2025 - Module Scan-Rule-Sets et Personnalisation}

\subsubsection{Lundi 10 Mars 2025 - Architecture Scan-Rule-Sets}

\textbf{Activités réalisées :}
\begin{itemize}
    \item Conception de l'architecture du module Scan-Rule-Sets
    \item Analyse des besoins utilisateur pour les règles personnalisées
    \item Développement des modèles de données dans \texttt{scan\_rule\_set\_models.py}
    \item Création de l'interface de gestion des règles
    \item Définition du format JSON pour les règles personnalisées
\end{itemize}

\textbf{Apprentissages :}
La gestion des règles de scan nécessite une approche flexible permettant aux utilisateurs de définir des critères complexes. J'ai appris l'importance de créer des abstractions simples pour des logiques complexes.

\textbf{Compétences mobilisées :}
Conception d'APIs flexibles, modélisation de données complexes, UX design pour utilisateurs techniques.

\textbf{Défis rencontrés :}
Équilibrage entre simplicité d'utilisation et puissance d'expression des règles. Les utilisateurs métier ont besoin d'interfaces intuitives pour des logiques sophistiquées.

\subsubsection{Mardi 11 Mars 2025 - Moteur de Règles}

\textbf{Activités réalisées :}
\begin{itemize}
    \item Implémentation du moteur d'exécution des règles personnalisées
    \item Développement du système de validation des règles
    \item Création des services dans \texttt{custom\_scan\_rule\_service.py}
    \item Intégration avec le module Classifications
    \item Tests de performance du moteur de règles
\end{itemize}

\textbf{Apprentissages :}
Le développement d'un moteur de règles performant nécessite une attention particulière à l'optimisation des expressions régulières et à la mise en cache des résultats. J'ai appris les techniques d'optimisation des parsers de règles.

\textbf{Compétences mobilisées :}
Développement de moteurs de règles, optimisation d'algorithmes, expressions régulières avancées.

\textbf{Code développé :}
\begin{lstlisting}[language=Python, caption=Moteur de règles personnalisées]
class ScanRuleEngine:
    def __init__(self):
        self.compiled_rules = {}
        self.performance_cache = LRUCache(maxsize=1000)
    
    async def execute_rule_set(self, data: Any, rule_set: ScanRuleSet):
        rule_key = self.get_rule_cache_key(rule_set)
        if rule_key in self.compiled_rules:
            compiled_rule = self.compiled_rules[rule_key]
        else:
            compiled_rule = self.compile_rule_set(rule_set)
            self.compiled_rules[rule_key] = compiled_rule
        
        return await compiled_rule.execute(data)
\end{lstlisting}

\subsubsection{Mercredi 12 Mars 2025 - Interface Utilisateur Avancée}

\textbf{Activités réalisées :}
\begin{itemize}
    \item Développement de l'interface de création de règles dans le frontend
    \item Implémentation du drag-and-drop pour la composition de règles
    \item Création des composants de validation en temps réel
    \item Intégration avec les APIs backend
    \item Tests utilisateur avec l'équipe métier
\end{itemize}

\textbf{Apprentissages :}
Le développement d'interfaces pour la création de règles complexes nécessite une approche UX innovante. J'ai appris l'importance du feedback visuel immédiat et de la validation en temps réel.

\textbf{Compétences mobilisées :}
React.js avancé, UX design, composants interactifs, validation côté client.

\textbf{Réflexions :}
L'interface développée permet aux utilisateurs métier de créer des règles sophistiquées sans connaissances techniques approfondies. Cette approche démocratise l'utilisation du système.

\subsubsection{Jeudi 13 Mars 2025 - Intégration et Tests}

\textbf{Activités réalisées :}
\begin{itemize}
    \item Intégration complète du module Scan-Rule-Sets avec les autres modules
    \item Tests d'intégration avec le module Classifications
    \item Validation des performances système avec règles personnalisées
    \item Développement des tests automatisés pour le moteur de règles
    \item Documentation technique du module
\end{itemize}

\textbf{Apprentissages :}
L'intégration de modules complexes révèle l'importance d'une architecture bien conçue. Les interfaces bien définies facilitent grandement l'intégration et la maintenance.

\textbf{Compétences mobilisées :}
Tests d'intégration, architecture modulaire, documentation technique.

\textbf{Métriques obtenues :}
\begin{itemize}
    \item Temps d'exécution règle simple : 5ms
    \item Temps d'exécution règle complexe : 45ms
    \item Capacité : 1000 règles simultanées
    \item Précision validation : 99.2\%
\end{itemize}

\subsubsection{Vendredi 14 Mars 2025 - Workshop Équipe}

\textbf{Activités réalisées :}
\begin{itemize}
    \item Workshop interne sur l'utilisation du module Scan-Rule-Sets
    \item Formation de l'équipe aux nouvelles fonctionnalités
    \item Collecte des retours utilisateur et suggestions d'amélioration
    \item Planification des développements futurs
    \item Démonstration complète du système jusqu'à présent
\end{itemize}

\textbf{Apprentissages :}
Les workshops internes sont essentiels pour valider l'utilisabilité du système et identifier les améliorations nécessaires. J'ai appris l'importance de l'adoption utilisateur dès les phases de développement.

\textbf{Compétences mobilisées :}
Formation utilisateur, collecte de feedback, amélioration continue.

\textbf{Retours équipe :}
\begin{itemize}
    \item Interface intuitive et puissante
    \item Performances excellentes
    \item Suggestion : ajout de templates de règles prédéfinies
    \item Demande : intégration avec outils de monitoring existants
\end{itemize}

\textbf{Résumé de la semaine :}
Cette quatrième semaine consolide l'approche modulaire avec le développement complet du module Scan-Rule-Sets. L'interface utilisateur avancée permet une adoption facilitée par les équipes métier.

\textbf{Réalisations techniques majeures :}
\begin{itemize}
    \item Moteur de règles personnalisées haute performance
    \item Interface drag-and-drop pour création de règles
    \item Intégration transparente avec les modules existants
    \item Validation en temps réel des règles créées
\end{itemize}

\textbf{Objectifs pour la semaine suivante :}
\begin{itemize}
    \item Développer le module ScanLogic
    \item Implémenter l'orchestration intelligente des scans
    \item Optimiser les performances pour les gros volumes de données
\end{itemize}