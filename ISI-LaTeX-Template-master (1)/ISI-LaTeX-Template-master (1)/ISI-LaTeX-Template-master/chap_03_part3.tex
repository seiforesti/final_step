% CHAPITRE 3 - PARTIE 3 : MODULES 5-7
% Scan Logic, Compliance System, et RBAC

\section{Module Scan Logic : Orchestration Distribuée}

\subsection{Workflow Engine Multi-Étapes}

Le module Scan Logic constitue le moteur d'orchestration de DataWave, coordonnant l'exécution des scans sur une architecture distribuée avec gestion intelligente des ressources.

\subsubsection{Architecture du Workflow Engine}

Le workflow engine implémente une architecture sophistiquée permettant l'orchestration de workflows complexes avec logique conditionnelle et gestion des dépendances.

La figure \ref{fig:architecture_workflow} présente l'architecture du workflow engine.

\begin{figure}[htpb]
\centering
% TODO: Créer un diagramme de l'architecture workflow
\includegraphics[width=0.95\textwidth]{architecture_workflow}
\caption{Architecture du workflow engine avec orchestration multi-étapes}
\label{fig:architecture_workflow}
\end{figure}

\textbf{Phases du Workflow} : Le tableau \ref{tab:phases_workflow} détaille les phases d'exécution d'un scan.

\begin{table}[htpb]
\centering
\caption{Phases du workflow de scanning}
\label{tab:phases_workflow}
\begin{tabular}{|p{0.15\textwidth}|p{0.3\textwidth}|p{0.25\textwidth}|p{0.15\textwidth}|}
\hline
\textbf{Phase} & \textbf{Description} & \textbf{Actions} & \textbf{Durée Moy.} \\
\hline
INITIALIZATION & Préparation du scan & Validation config, allocation ressources & 5-10s \\
\hline
DISCOVERY & Découverte des assets & Extraction métadonnées, catalogage & 1-5 min \\
\hline
CLASSIFICATION & Classification des données & Application règles, ML, IA & 5-15 min \\
\hline
COMPLIANCE & Évaluation conformité & Vérification frameworks, scoring & 2-5 min \\
\hline
REPORTING & Génération rapports & Agrégation résultats, notifications & 1-2 min \\
\hline
CLEANUP & Nettoyage & Libération ressources, archivage & 1-2 min \\
\hline
\end{tabular}
\end{table}

\textbf{Logique Conditionnelle} : Le workflow engine supporte des conditions complexes :
\begin{itemize}
    \item \textbf{IF-THEN-ELSE} : Exécution conditionnelle basée sur résultats précédents
    \item \textbf{RETRY} : Tentatives automatiques avec exponential backoff
    \item \textbf{TIMEOUT} : Timeouts configurables par phase
    \item \textbf{FALLBACK} : Stratégies de fallback en cas d'échec
    \item \textbf{PARALLEL} : Exécution parallèle de branches indépendantes
\end{itemize}

\textbf{Gestion des Dépendances} : Le système gère automatiquement les dépendances entre étapes avec un DAG (Directed Acyclic Graph), garantissant l'ordre d'exécution correct et la parallélisation optimale.

\subsection{Orchestration Distribuée sur Edge Nodes}

L'innovation majeure de DataWave réside dans son architecture d'orchestration distribuée sur edge nodes, permettant une scalabilité illimitée.

\subsubsection{Architecture Distribuée}

La figure \ref{fig:orchestration_distribuee} illustre l'architecture d'orchestration distribuée.

\begin{figure}[htpb]
\centering
% TODO: Créer un diagramme de l'orchestration distribuée
\includegraphics[width=0.95\textwidth]{orchestration_distribuee}
\caption{Architecture d'orchestration distribuée sur edge nodes}
\label{fig:orchestration_distribuee}
\end{figure}

\textbf{Composants de l'Architecture} :
\begin{itemize}
    \item \textbf{Central Orchestrator} : Coordonne les edge nodes via Kafka
    \item \textbf{Edge Nodes} : Exécutent les scans localement près des sources
    \item \textbf{Resource Manager} : Alloue dynamiquement les ressources
    \item \textbf{Load Balancer} : Distribue intelligemment la charge
    \item \textbf{Health Monitor} : Surveille l'état des nodes en temps réel
\end{itemize}

\subsubsection{Allocation Dynamique de Ressources}

Le système implémente une allocation dynamique de ressources basée sur la charge et les priorités.

La figure \ref{fig:allocation_dynamique} montre le processus d'allocation dynamique.

\begin{figure}[htpb]
\centering
% TODO: Créer un diagramme de l'allocation dynamique
\includegraphics[width=0.85\textwidth]{allocation_dynamique}
\caption{Allocation dynamique de ressources avec load balancing intelligent}
\label{fig:allocation_dynamique}
\end{figure}

\textbf{Algorithme d'Allocation} :
\begin{enumerate}
    \item Évaluation de la charge actuelle de chaque edge node
    \item Calcul des ressources requises pour le scan (CPU, mémoire, I/O)
    \item Sélection du node optimal selon critères multiples :
    \begin{itemize}
        \item Proximité à la source de données (latence réseau)
        \item Disponibilité des ressources (CPU, mémoire, I/O)
        \item Charge actuelle (nombre de scans en cours)
        \item Historique de performance (succès, échecs, vitesse)
    \end{itemize}
    \item Allocation des ressources avec réservation
    \item Monitoring continu et réallocation si nécessaire
\end{enumerate}

\textbf{Load Balancing Intelligent} : Le système utilise un algorithme de load balancing pondéré qui considère :
\begin{itemize}
    \item Capacité du node (CPU cores, RAM, I/O bandwidth)
    \item Charge actuelle (utilisation CPU/mémoire/I/O)
    \item Priorité du scan (URGENT, HIGH, NORMAL, LOW)
    \item SLA du client (temps de réponse garanti)
\end{itemize}

\textbf{Résultat Mesurable} : L'allocation dynamique a permis d'augmenter l'utilisation des ressources de 45\% à 82\%, réduisant les coûts d'infrastructure de 40\% tout en améliorant les performances.

\subsection{Monitoring en Temps Réel}

Le monitoring en temps réel est essentiel pour garantir la visibilité et la réactivité du système.

\subsubsection{Dashboard de Monitoring}

La figure \ref{fig:dashboard_monitoring} présente le dashboard de monitoring en temps réel.

\begin{figure}[htpb]
\centering
% TODO: Ajouter capture d'écran du dashboard
\includegraphics[width=0.95\textwidth]{dashboard_monitoring}
\caption{Dashboard de monitoring en temps réel avec métriques de performance}
\label{fig:dashboard_monitoring}
\end{figure}

\textbf{Métriques Monitorées} :
\begin{itemize}
    \item \textbf{Progression} : Pourcentage de complétion par phase
    \item \textbf{Throughput} : Lignes/seconde, tables/minute
    \item \textbf{Latence} : Temps de réponse par opération
    \item \textbf{Ressources} : CPU, mémoire, I/O, réseau
    \item \textbf{Erreurs} : Taux d'erreur, types d'erreurs
    \item \textbf{Queue} : Scans en attente, temps d'attente
\end{itemize}

\subsubsection{Progression des Scans}

La figure \ref{fig:progression_scans} montre la visualisation de la progression des scans.

\begin{figure}[htpb]
\centering
% TODO: Ajouter capture d'écran de la progression
\includegraphics[width=0.9\textwidth]{progression_scans}
\caption{Visualisation de la progression des scans avec timeline détaillée}
\label{fig:progression_scans}
\end{figure}

\textbf{Informations de Progression} :
\begin{itemize}
    \item Timeline des phases avec durées
    \item Nombre d'assets traités / total
    \item Vitesse de traitement actuelle
    \item Temps estimé de complétion (ETA)
    \item Alertes et warnings en temps réel
\end{itemize}

\subsection{Alerting et Gestion des Erreurs}

Le système implémente un système d'alerting multi-niveaux avec gestion intelligente des erreurs.

\subsubsection{Système d'Alerting}

La figure \ref{fig:systeme_alerting} présente le système d'alerting.

\begin{figure}[htpb]
\centering
% TODO: Ajouter capture d'écran du système d'alerting
\includegraphics[width=0.85\textwidth]{systeme_alerting}
\caption{Système d'alerting multi-niveaux avec escalation automatique}
\label{fig:systeme_alerting}
\end{figure}

\textbf{Niveaux d'Alertes} :
\begin{itemize}
    \item \textbf{INFO} : Événements normaux (début/fin scan)
    \item \textbf{WARNING} : Situations anormales non critiques (latence élevée)
    \item \textbf{ERROR} : Erreurs nécessitant attention (échec connexion)
    \item \textbf{CRITICAL} : Erreurs critiques (perte de node, corruption données)
\end{itemize}

\textbf{Canaux de Notification} :
\begin{itemize}
    \item Email avec détails et recommandations
    \item Slack/Teams avec liens directs vers dashboard
    \item SMS pour alertes critiques
    \item Webhooks pour intégrations SIEM/ticketing
    \item In-app notifications en temps réel
\end{itemize}

\textbf{Retry Automatique} : Le système implémente une stratégie de retry avec exponential backoff :
\begin{itemize}
    \item Tentative 1 : Immédiate
    \item Tentative 2 : Après 10 secondes
    \item Tentative 3 : Après 30 secondes
    \item Tentative 4 : Après 1 minute
    \item Tentative 5 : Après 5 minutes
    \item Échec final : Alerte CRITICAL et escalation
\end{itemize}

Le tableau \ref{tab:metriques_orchestration} présente les métriques d'orchestration.

\begin{table}[htpb]
\centering
\caption{Métriques d'orchestration et de performance}
\label{tab:metriques_orchestration}
\begin{tabular}{|p{0.3\textwidth}|p{0.25\textwidth}|p{0.3\textwidth}|}
\hline
\textbf{Métrique} & \textbf{Valeur} & \textbf{Objectif} \\
\hline
Scans parallèles max & 50+ & > 50 \\
\hline
Temps de failover & < 5 secondes & < 10 secondes \\
\hline
Taux de succès scans & 99.2\% & > 99\% \\
\hline
Utilisation ressources & 82\% & 70-85\% \\
\hline
Temps moyen scan (1000 tables) & 3 minutes & < 5 minutes \\
\hline
Latence orchestration & 50ms & < 100ms \\
\hline
\end{tabular}
\end{table}

\section{Module Compliance System : Conformité Automatisée}

\subsection{Support Multi-Frameworks}

Le module Compliance System automatise la conformité réglementaire en supportant 6 frameworks majeurs avec évaluation automatique et reporting avancé.

\subsubsection{Frameworks Supportés}

La figure \ref{fig:architecture_compliance} présente l'architecture du système de conformité.

\begin{figure}[htpb]
\centering
% TODO: Créer un diagramme de l'architecture compliance
\includegraphics[width=0.9\textwidth]{architecture_compliance}
\caption{Architecture du système de conformité multi-frameworks}
\label{fig:architecture_compliance}
\end{figure}

Le tableau \ref{tab:frameworks_compliance} détaille les 6 frameworks supportés.

\begin{table}[htpb]
\centering
\caption{Frameworks de conformité supportés avec exigences clés}
\label{tab:frameworks_compliance}
\begin{tabular}{|p{0.12\textwidth}|p{0.15\textwidth}|p{0.3\textwidth}|p{0.28\textwidth}|}
\hline
\textbf{Framework} & \textbf{Région} & \textbf{Domaine} & \textbf{Exigences Clés} \\
\hline
SOC2 & Global & Services cloud & Security, Availability, Processing Integrity, Confidentiality, Privacy \\
\hline
GDPR & UE & Données personnelles & Consentement, droit à l'oubli, portabilité, notification breaches < 72h \\
\hline
HIPAA & USA & Santé & Protection PHI, audit trails, chiffrement, contrôles d'accès \\
\hline
PCI-DSS & Global & Paiement & Protection PAN, segmentation réseau, chiffrement, tests sécurité \\
\hline
SOX & USA & Finance & Contrôles internes, séparation des tâches, audit, reporting financier \\
\hline
CCPA & Californie & Consommateurs & Transparence, opt-out, non-discrimination, suppression données \\
\hline
\end{tabular}
\end{table}

\textbf{Règles Pré-Configurées} : Pour chaque framework, DataWave inclut des règles pré-configurées couvrant les exigences principales :
\begin{itemize}
    \item \textbf{GDPR} : 45+ règles (identification PII, consentement, encryption, retention)
    \item \textbf{HIPAA} : 38+ règles (PHI protection, access controls, audit logs, encryption)
    \item \textbf{PCI-DSS} : 32+ règles (PAN protection, network segmentation, encryption)
    \item \textbf{SOX} : 28+ règles (financial data controls, audit trails, segregation)
    \item \textbf{SOC2} : 52+ règles (security, availability, integrity, confidentiality, privacy)
    \item \textbf{CCPA} : 25+ règles (consumer data, opt-out, deletion, transparency)
\end{itemize}

\subsection{Évaluation Automatique}

Le système évalue automatiquement la conformité avec scoring détaillé par framework.

\subsubsection{Scopes de Règles}

Le tableau \ref{tab:scopes_regles} présente les scopes d'application des règles.

\begin{table}[htpb]
\centering
\caption{Scopes d'application des règles de conformité}
\label{tab:scopes_regles}
\begin{tabular}{|p{0.15\textwidth}|p{0.35\textwidth}|p{0.35\textwidth}|}
\hline
\textbf{Scope} & \textbf{Description} & \textbf{Exemple} \\
\hline
GLOBAL & S'applique à toute l'organisation & Politique de chiffrement globale \\
\hline
DATA\_SOURCE & S'applique à une source spécifique & Règles spécifiques à une BD production \\
\hline
SCHEMA & S'applique à un schéma & Règles pour schéma "customers" \\
\hline
TABLE & S'applique à une table & Règles pour table "credit\_cards" \\
\hline
COLUMN & S'applique à une colonne & Règles pour colonne "ssn" \\
\hline
\end{tabular}
\end{table}

\subsubsection{Processus d'Évaluation}

La figure \ref{fig:processus_evaluation} illustre le processus d'évaluation automatique.

\begin{figure}[htpb]
\centering
% TODO: Créer un diagramme du processus d'évaluation
\includegraphics[width=0.9\textwidth]{processus_evaluation}
\caption{Processus d'évaluation automatique de conformité}
\label{fig:processus_evaluation}
\end{figure}

\textbf{Étapes d'Évaluation} :
\begin{enumerate}
    \item Sélection des règles applicables selon scope
    \item Collecte des données nécessaires (classifications, métadonnées, configurations)
    \item Évaluation de chaque règle avec scoring (COMPLIANT, NON\_COMPLIANT, PARTIAL)
    \item Calcul du score global par framework (0-100\%)
    \item Identification des violations avec sévérité (CRITICAL, HIGH, MEDIUM, LOW)
    \item Génération de recommandations de remédiation
    \item Création d'issues avec workflows d'approbation
\end{enumerate}

\textbf{Scoring de Conformité} : Le score est calculé selon la formule :
\[
Score = \frac{\sum_{i=1}^{n} (w_i \times s_i)}{\sum_{i=1}^{n} w_i} \times 100
\]
où $w_i$ est le poids de la règle $i$ et $s_i$ son score (0 ou 1).

\subsection{Gestion des Issues et Remédiation}

Le système gère automatiquement les violations de conformité avec workflows de remédiation.

\subsubsection{Détection et Priorisation}

La figure \ref{fig:gestion_issues} présente l'interface de gestion des issues.

\begin{figure}[htpb]
\centering
% TODO: Ajouter capture d'écran de gestion des issues
\includegraphics[width=0.95\textwidth]{gestion_issues}
\caption{Gestion des issues de conformité avec workflows de remédiation}
\label{fig:gestion_issues}
\end{figure}

\textbf{Priorisation Automatique} : Les issues sont automatiquement priorisées selon :
\begin{itemize}
    \item Sévérité de la violation (CRITICAL > HIGH > MEDIUM > LOW)
    \item Framework concerné (GDPR, HIPAA > autres)
    \item Volume de données affectées
    \item Exposition (public, interne, confidentiel)
    \item Historique de violations similaires
\end{itemize}

\textbf{Plans de Remédiation} : Pour chaque type de violation, le système propose des plans de remédiation automatiques :
\begin{itemize}
    \item Actions recommandées (chiffrement, masking, suppression, etc.)
    \item Estimation du temps et des ressources nécessaires
    \item Impact sur les systèmes et utilisateurs
    \item Procédures de validation
\end{itemize}

\subsection{Reporting et Audit}

Le système génère automatiquement des rapports de conformité détaillés.

\subsubsection{Dashboard de Conformité}

La figure \ref{fig:dashboard_conformite} présente le dashboard exécutif de conformité.

\begin{figure}[htpb]
\centering
% TODO: Ajouter capture d'écran du dashboard
\includegraphics[width=0.95\textwidth]{dashboard_conformite}
\caption{Dashboard exécutif de conformité multi-frameworks}
\label{fig:dashboard_conformite}
\end{figure}

\textbf{Métriques du Dashboard} :
\begin{itemize}
    \item Score de conformité par framework (0-100\%)
    \item Nombre de violations par sévérité
    \item Tendances de conformité (amélioration/dégradation)
    \item Issues ouvertes vs résolues
    \item Temps moyen de remédiation
    \item Couverture de l'évaluation (assets évalués / total)
\end{itemize}

\subsubsection{Rapports d'Audit}

La figure \ref{fig:rapport_audit_gdpr} montre un exemple de rapport d'audit GDPR.

\begin{figure}[htpb]
\centering
% TODO: Ajouter capture d'écran du rapport
\includegraphics[width=0.9\textwidth]{rapport_audit_gdpr}
\caption{Rapport d'audit GDPR détaillé avec recommandations}
\label{fig:rapport_audit_gdpr}
\end{figure}

\textbf{Contenu des Rapports} :
\begin{itemize}
    \item Executive summary avec score global
    \item Détail des règles évaluées (compliant, non-compliant, partial)
    \item Liste des violations avec sévérité et impact
    \item Recommandations de remédiation priorisées
    \item Timeline des évaluations précédentes
    \item Annexes techniques (logs, preuves, configurations)
\end{itemize}

Le tableau \ref{tab:metriques_conformite} présente les métriques de conformité.

\begin{table}[htpb]
\centering
\caption{Métriques de conformité par framework}
\label{tab:metriques_conformite}
\begin{tabular}{|p{0.15\textwidth}|p{0.15\textwidth}|p{0.15\textwidth}|p{0.15\textwidth}|p{0.25\textwidth}|}
\hline
\textbf{Framework} & \textbf{Score} & \textbf{Violations} & \textbf{Temps Remed.} & \textbf{Statut} \\
\hline
SOC2 & 94\% & 3 MEDIUM & 2 jours & COMPLIANT \\
\hline
GDPR & 91\% & 5 HIGH, 8 MEDIUM & 5 jours & PARTIAL \\
\hline
HIPAA & 96\% & 2 MEDIUM & 1 jour & COMPLIANT \\
\hline
PCI-DSS & 89\% & 1 CRITICAL, 4 HIGH & 7 jours & NON-COMPLIANT \\
\hline
SOX & 93\% & 4 MEDIUM & 3 jours & COMPLIANT \\
\hline
CCPA & 95\% & 3 LOW & 1 jour & COMPLIANT \\
\hline
\end{tabular}
\end{table}

\section{Module RBAC : Sécurité et Contrôle d'Accès}

\subsection{Contrôle d'Accès Granulaire}

Le module RBAC (Role-Based Access Control) implémente un système de contrôle d'accès granulaire au niveau ressource avec support ABAC (Attribute-Based Access Control).

\subsubsection{Architecture RBAC}

La figure \ref{fig:architecture_rbac} présente l'architecture du système RBAC.

\begin{figure}[htpb]
\centering
% TODO: Créer un diagramme de l'architecture RBAC
\includegraphics[width=0.85\textwidth]{architecture_rbac}
\caption{Architecture RBAC avec permissions granulaires}
\label{fig:architecture_rbac}
\end{figure}

\textbf{Modèle de Permissions} : Le système utilise un modèle hiérarchique de permissions avec héritage.

Le tableau \ref{tab:niveaux_permissions} détaille les niveaux de permissions.

\begin{table}[htpb]
\centering
\caption{Niveaux de permissions par type de ressource}
\label{tab:niveaux_permissions}
\begin{tabular}{|p{0.2\textwidth}|p{0.35\textwidth}|p{0.35\textwidth}|}
\hline
\textbf{Ressource} & \textbf{Permissions} & \textbf{Description} \\
\hline
Data Source & VIEW, EDIT, DELETE, SCAN, CONFIGURE & Gestion des sources de données \\
\hline
Schema/Table/Column & VIEW, EDIT, DELETE, CLASSIFY & Gestion des assets \\
\hline
Scan & VIEW, CREATE, EDIT, DELETE, EXECUTE & Gestion des scans \\
\hline
Rule & VIEW, CREATE, EDIT, DELETE, ACTIVATE & Gestion des règles \\
\hline
Report & VIEW, CREATE, EDIT, DELETE, EXPORT & Gestion des rapports \\
\hline
User & VIEW, CREATE, EDIT, DELETE, ASSIGN\_ROLE & Gestion des utilisateurs \\
\hline
\end{tabular}
\end{table}

\subsubsection{ABAC (Attribute-Based Access Control)}

En complément du RBAC, DataWave implémente l'ABAC pour des politiques d'accès dynamiques basées sur attributs.

\textbf{Attributs Contextuels} :
\begin{itemize}
    \item \textbf{Utilisateur} : Rôle, département, niveau sécurité, localisation
    \item \textbf{Ressource} : Type, sensibilité, propriétaire, tags
    \item \textbf{Environnement} : Heure, jour, localisation, réseau
    \item \textbf{Action} : Type d'opération, impact, risque
\end{itemize}

\textbf{Exemple de Politique ABAC} : "Autoriser l'accès aux données PII uniquement si l'utilisateur a le rôle 'Data Steward', est connecté depuis le réseau interne, pendant les heures de bureau, et a complété la formation GDPR dans les 12 derniers mois."

\subsection{Multi-Tenancy et Isolation}

Le système supporte le multi-tenancy avec isolation complète entre organisations.

\textbf{Isolation des Données} :
\begin{itemize}
    \item Séparation logique au niveau base de données (tenant\_id dans toutes les tables)
    \item Validation du tenant\_id à chaque requête (middleware)
    \item Chiffrement des données au repos par tenant
    \item Isolation des ressources compute (quotas par tenant)
\end{itemize}

\subsection{Audit et Traçabilité}

Le système maintient un audit trail complet de toutes les actions utilisateur.

\textbf{Événements Audités} : Le tableau \ref{tab:evenements_audites} liste les événements audités.

\begin{table}[htpb]
\centering
\caption{Événements audités avec retention policies}
\label{tab:evenements_audites}
\begin{tabular}{|p{0.25\textwidth}|p{0.35\textwidth}|p{0.25\textwidth}|}
\hline
\textbf{Catégorie} & \textbf{Événements} & \textbf{Retention} \\
\hline
Authentification & Login, logout, MFA, échecs & 2 ans \\
\hline
Accès données & View, export, modification & 7 ans \\
\hline
Configuration & Création, modification, suppression & 5 ans \\
\hline
Scans & Exécution, résultats, erreurs & 3 ans \\
\hline
Conformité & Violations, remédiation, rapports & 10 ans \\
\hline
\end{tabular}
\end{table}

\textbf{Correlation IDs} : Chaque action est tracée avec un correlation ID unique permettant de suivre une transaction complète à travers tous les microservices.

\section*{Conclusion}

Ce chapitre a présenté la réalisation complète des 7 modules de gouvernance de DataWave, démontrant une maîtrise technique exceptionnelle et des résultats mesurables impressionnants. Les innovations majeures incluent l'architecture edge computing, la classification multi-niveaux atteignant 96.3\% de précision, l'orchestration distribuée avec allocation dynamique de ressources, et la conformité automatisée multi-frameworks. Les résultats mesurables (62\% réduction latence, 70\% réduction temps scan, 99.99\% disponibilité, 96.3\% précision classification) démontrent la supériorité de DataWave face aux solutions concurrentes. Le chapitre suivant présentera les tests, le déploiement, et l'analyse comparative détaillée avec Azure Purview et Databricks Unity Catalog.
