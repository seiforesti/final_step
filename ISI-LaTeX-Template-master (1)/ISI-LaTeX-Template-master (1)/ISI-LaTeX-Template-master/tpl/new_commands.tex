
%=========== File containing some new commands ============%
%                                                          %
% Copyright (C) ISI - All Rights Reserved                  %
% Proprietary                                              %
% Written by Med Hossam <med.hossam@gmail.com>, April 2016 %
%                                                          %
% @author: HEDHILI Med Houssemeddine                       %
% @linkedin: http://tn.linkedin.com/in/medhossam           %
%==========================================================%

\newenvironment{changemargin}[2]{%
\begin{list}{}{%
\setlength{\leftmargin}{#1}%
\setlength{\rightmargin}{#2}%
}%
\item[]}
{\end{list}}

\makeatletter

%================= front cover variables =================%

\newcommand{\secondAuthor}[1]{\gdef\@secondAuthor{#1}}%
\newcommand{\@secondAuthor}{\@latex@warning@no@line{No \noexpand\secondAuthor given}}

\newcommand{\diplomaName}[1]{\gdef\@diplomaName{#1}}%
\newcommand{\@diplomaName}{\@latex@warning@no@line{No \noexpand\diplomaName given}}

\newcommand{\speciality}[1]{\gdef\@speciality{#1}}%
\newcommand{\@speciality}{\@latex@warning@no@line{No \noexpand\speciality given}}

\newcommand{\proFramerName}[1]{\gdef\@proFramerName{#1}}%
\newcommand{\@proFramerName}{\@latex@warning@no@line{No \noexpand\proFramerName given}}

\newcommand{\proFramerSpeciality}[1]{\gdef\@proFramerSpeciality{#1}}%
\newcommand{\@proFramerSpeciality}{\@latex@warning@no@line{No \noexpand\proFramerSpeciality given}}

\newcommand{\academicFramerName}[1]{\gdef\@academicFramerName{#1}}%
\newcommand{\@academicFramerName}{\@latex@warning@no@line{No \noexpand\academicFramerName given}}

\newcommand{\academicFramerSpeciality}[1]{\gdef\@academicFramerSpeciality{#1}}%
\newcommand{\@academicFramerSpeciality}{\@latex@warning@no@line{No \noexpand\academicFramerSpeciality given}}

\newcommand{\collegeYear}[1]{\gdef\@collegeYear{#1}}%
\newcommand{\@collegeYear}{\@latex@warning@no@line{No \noexpand\collegeYear given}}

\newcommand{\companyName}[1]{\gdef\@companyName{#1}}%
\newcommand{\@companyName}{\@latex@warning@no@line{No \noexpand\companyName given}}

%================== Signatures variables ==================%

\newcommand{\proSignSentence}[1]{\gdef\@proSignSentence{#1}}%
\newcommand{\@proSignSentence}{\@latex@warning@no@line{No \noexpand\proSignSentence given}}

\newcommand{\academicSignSentence}[1]{\gdef\@academicSignSentence{#1}}%
\newcommand{\@academicSignSentence}{\@latex@warning@no@line{No \noexpand\academicSignSentence given}}

%================== Backcover variables ==================%

\newcommand{\arabicAbstract}[1]{\gdef\@arabicAbstract{#1}}%
\newcommand{\@arabicAbstract}{\@latex@warning@no@line{No \noexpand\arabicAbstract given}}

\newcommand{\arabicAbstractKeywords}[1]{\gdef\@arabicAbstractKeywords{#1}}%
\newcommand{\@arabicAbstractKeywords}{\@latex@warning@no@line{No \noexpand\arabicAbstractKeywords given}}

\newcommand{\frenchAbstract}[1]{\gdef\@frenchAbstract{#1}}%
\newcommand{\@frenchAbstract}{\@latex@warning@no@line{No \noexpand\frenchAbstract given}}

\newcommand{\frenchAbstractKeywords}[1]{\gdef\@frenchAbstractKeywords{#1}}%
\newcommand{\@frenchAbstractKeywords}{\@latex@warning@no@line{No \noexpand\frenchAbstractKeywords given}}

\newcommand{\englishAbstract}[1]{\gdef\@englishAbstract{#1}}%
\newcommand{\@englishAbstract}{\@latex@warning@no@line{No \noexpand\englishAbstract given}}

\newcommand{\englishAbstractKeywords}[1]{\gdef\@englishAbstractKeywords{#1}}%
\newcommand{\@englishAbstractKeywords}{\@latex@warning@no@line{No \noexpand\englishAbstractKeywords given}}

\newcommand{\companyEmail}[1]{\gdef\@companyEmail{#1}}%
\newcommand{\@companyEmail}{\@latex@warning@no@line{No \noexpand\companyEmail given}}

\newcommand{\companyTel}[1]{\gdef\@companyTel{#1}}%
\newcommand{\@companyTel}{\@latex@warning@no@line{No \noexpand\companyTel given}}

\newcommand{\companyFax}[1]{\gdef\@companyFax{#1}}%
\newcommand{\@companyFax}{\@latex@warning@no@line{No \noexpand\companyFax given}}

\newcommand{\companyAddressFR}[1]{\gdef\@companyAddressFR{#1}}%
\newcommand{\@companyAddressFR}{\@latex@warning@no@line{No \noexpand\companyAddressFR given}}

\newcommand{\companyAddressAR}[1]{\gdef\@companyAddressAR{#1}}%
\newcommand{\@companyAddressAR}{\@latex@warning@no@line{No \noexpand\companyAddressAR given}}

%============= cmd for inserting blank page =============%
\newcommand\blankpage{%
    \null
    \thispagestyle{empty}%
    \addtocounter{page}{-1}%
    \newpage}

%================ document main language ================%
%\selectlanguage{english}
\selectlanguage{french}

%================== required packages ===================%

\usepackage{tcolorbox}
\usepackage{afterpage}
\usepackage{array,longtable,multirow}% http://ctan.org/pkg/{array,longtable,multirow}
\usepackage{pifont}

\usepackage{pdflscape}
\usepackage{rotating}
\usepackage{wrapfig}

%================ TABLEAUX ADAPTATIFS PROFESSIONNELS =================%
% Utilise tabularx + ragged2e pour tableaux flexibles et propres

% Commande pour tableau 2 colonnes
\newcommand{\tabletwo}[3]{%
    \begin{table}[htbp]
    \centering
    \caption{#2}
    \label{#1}
    \renewcommand{\arraystretch}{1.3}
    \setlength{\tabcolsep}{8pt}
    \small
    \begin{tabularx}{\textwidth}{>{\RaggedRight\arraybackslash}X>{\RaggedRight\arraybackslash}X}
    \toprule
    #3
    \bottomrule
    \end{tabularx}
    \end{table}
}

% Commande pour tableau 3 colonnes
\newcommand{\tablethree}[3]{%
    \begin{table}[htbp]
    \centering
    \caption{#2}
    \label{#1}
    \renewcommand{\arraystretch}{1.3}
    \setlength{\tabcolsep}{8pt}
    \small
    \begin{tabularx}{\textwidth}{>{\RaggedRight\arraybackslash}X>{\RaggedRight\arraybackslash}X>{\RaggedRight\arraybackslash}X}
    \toprule
    #3
    \bottomrule
    \end{tabularx}
    \end{table}
}

% Commande pour tableau 4 colonnes
\newcommand{\tablefour}[3]{%
    \begin{table}[htbp]
    \centering
    \caption{#2}
    \label{#1}
    \renewcommand{\arraystretch}{1.3}
    \setlength{\tabcolsep}{8pt}
    \small
    \begin{tabularx}{\textwidth}{>{\RaggedRight\arraybackslash}X>{\RaggedRight\arraybackslash}X>{\RaggedRight\arraybackslash}X>{\RaggedRight\arraybackslash}X}
    \toprule
    #3
    \bottomrule
    \end{tabularx}
    \end{table}
}

% Alias pour compatibilité
\newcommand{\tablethreecustom}[3]{\tablethree{#1}{#2}{#3}}

% Commande pour ajouter une ligne avec espacement dans un tableau
\newcommand{\tablerow}[1]{#1 \\ \addlinespace}

% Commande pour ligne d'en-tête
\newcommand{\tableheader}[1]{#1 \\ \midrule}

%================ EXEMPLE D'UTILISATION =================%
% \tablethreecustom{tab:exemple}{Titre du tableau}{
%   \tableheader{\textbf{Colonne 1} & \textbf{Colonne 2} & \textbf{Colonne 3}}
%   \tablerow{Donnée 1 & Donnée 2 & Donnée 3}
%   \tablerow{Donnée 4 & Donnée 5 & Donnée 6}
%   Dernière ligne & Dernière donnée & Fin \\
% }