\documentclass[12pt,a4paper]{article}
\usepackage[a4paper,top=2.5cm,bottom=2.5cm,left=2.5cm,right=2.5cm]{geometry}
\usepackage{tabularx}
\usepackage{booktabs}
\usepackage{ragged2e}
\usepackage[utf8]{inputenc}
\usepackage[T1]{fontenc}
\usepackage{mathptmx}

\begin{document}

\section*{Test Tableau Adaptatif}

Largeur de page (textwidth): \the\textwidth

\vspace{1cm}

\begin{table}[h]
\centering
\caption{Test tableau avec tabularx et ragged2e}
\renewcommand{\arraystretch}{1.3}
\setlength{\tabcolsep}{8pt}
\small
\begin{tabularx}{\textwidth}{>{\RaggedRight\arraybackslash}X>{\RaggedRight\arraybackslash}X>{\RaggedRight\arraybackslash}X}
\toprule
\textbf{Défi Critique} & \textbf{Impact Quantifié} & \textbf{Conséquences Métier} \\
\midrule
Intégration multi-bases de données & 60\% d'entreprises en difficulté & Développement manuel 3-6 mois par connecteur, coûts prohibitifs \\
\midrule
Classification manuelle & 70\% données non classifiées & Violations de données sensibles, non-conformité, pénalités financières \\
\midrule
Orchestration fragmentée & 80\% processus manuels & Latence élevée, coûts opérationnels, incohérences \\
\bottomrule
\end{tabularx}
\end{table}

\vspace{1cm}

Si ce tableau s'affiche correctement avec les 3 colonnes complètes, alors le problème vient du template ISI.

Si ce tableau est aussi coupé, alors le problème vient de MiKTeX ou des packages manquants.

\end{document}
