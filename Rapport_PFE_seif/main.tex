% Options for packages loaded elsewhere
\PassOptionsToPackage{unicode}{hyperref}
\PassOptionsToPackage{hyphens}{url}
%
\documentclass[
]{article}
\usepackage{amsmath,amssymb}
\usepackage{lmodern}
\usepackage{iftex}
\ifPDFTeX
  \usepackage[T1]{fontenc}
  \usepackage[utf8]{inputenc}
  \usepackage{textcomp} % provide euro and other symbols
\else % if luatex or xetex
  \usepackage{unicode-math}
  \defaultfontfeatures{Scale=MatchLowercase}
  \defaultfontfeatures[\rmfamily]{Ligatures=TeX,Scale=1}
\fi
% Use upquote if available, for straight quotes in verbatim environments
\IfFileExists{upquote.sty}{\usepackage{upquote}}{}
\IfFileExists{microtype.sty}{% use microtype if available
  \usepackage[]{microtype}
  \UseMicrotypeSet[protrusion]{basicmath} % disable protrusion for tt fonts
}{}
\makeatletter
\@ifundefined{KOMAClassName}{% if non-KOMA class
  \IfFileExists{parskip.sty}{%
    \usepackage{parskip}
  }{% else
    \setlength{\parindent}{0pt}
    \setlength{\parskip}{6pt plus 2pt minus 1pt}}
}{% if KOMA class
  \KOMAoptions{parskip=half}}
\makeatother
\usepackage{xcolor}
\usepackage{longtable,booktabs,array}
\usepackage{multirow}
\usepackage{calc} % for calculating minipage widths
% Correct order of tables after \paragraph or \subparagraph
\usepackage{etoolbox}
\makeatletter
\patchcmd\longtable{\par}{\if@noskipsec\mbox{}\fi\par}{}{}
\makeatother
% Allow footnotes in longtable head/foot
\IfFileExists{footnotehyper.sty}{\usepackage{footnotehyper}}{\usepackage{footnote}}
\makesavenoteenv{longtable}
\usepackage{graphicx}
\makeatletter
\def\maxwidth{\ifdim\Gin@nat@width>\linewidth\linewidth\else\Gin@nat@width\fi}
\def\maxheight{\ifdim\Gin@nat@height>\textheight\textheight\else\Gin@nat@height\fi}
\makeatother
% Scale images if necessary, so that they will not overflow the page
% margins by default, and it is still possible to overwrite the defaults
% using explicit options in \includegraphics[width, height, ...]{}
\setkeys{Gin}{width=\maxwidth,height=\maxheight,keepaspectratio}
% Set default figure placement to htbp
\makeatletter
\def\fps@figure{htbp}
\makeatother
\usepackage[normalem]{ulem}
\setlength{\emergencystretch}{3em} % prevent overfull lines
\providecommand{\tightlist}{%
  \setlength{\itemsep}{0pt}\setlength{\parskip}{0pt}}
\setcounter{secnumdepth}{-\maxdimen} % remove section numbering
\ifLuaTeX
  \usepackage{selnolig}  % disable illegal ligatures
\fi
\IfFileExists{bookmark.sty}{\usepackage{bookmark}}{\usepackage{hyperref}}
\IfFileExists{xurl.sty}{\usepackage{xurl}}{} % add URL line breaks if available
\urlstyle{same} % disable monospaced font for URLs
\hypersetup{
  hidelinks,
  pdfcreator={LaTeX via pandoc}}

\author{}
\date{}

\begin{document}

\includegraphics[width=8.25833in,height=11.68252in]{vertopal_fa5b2edd966b48bfbc6a830f7abfbc35/media/image1.png}

\textbf{2022 - 2023}

\textbf{INFORMATIQUE}

\begin{quote}
\textbf{Implémentation d\textquotesingle une solution SOAR}
\end{quote}

\textbf{dans un SOC}

\textbf{Réalisé par: Dhia Abdelli}

\textbf{Encadré par:}

\textbf{Encadrant ESPRIT: Hend Ben Moussa}

\textbf{Encadrant Entreprise: Ramzi Ben Slimen}

\includegraphics[width=8.38056in,height=11.84958in]{vertopal_fa5b2edd966b48bfbc6a830f7abfbc35/media/image2.png}

\begin{quote}
Je valide le depot du rapport PFE relatif a I\textquotesingle{} etudiant
nomme ci-dessous:
\end{quote}

% Simplified: avoid longtable here
\noindent •\; Dhia Abdelli

% Simplified to avoid nested longtable causing errors
\paragraph{Encadrant Advancia IT System}
\textbf{Ramzi Ben Slimen}\\
Les Berges du Lac\\
Fax: 71 860706

\begin{longtable}[]{@{}
  >{\raggedright\arraybackslash}p{(\columnwidth - 2\tabcolsep) * \real{0.5000}}
  >{\raggedright\arraybackslash}p{(\columnwidth - 2\tabcolsep) * \real{0.5000}}@{}}
\toprule()
\begin{minipage}[b]{\linewidth}\raggedright
\textbf{•}
\end{minipage} & \begin{minipage}[b]{\linewidth}\raggedright
Encadrant Academique

\begin{quote}
\textbf{Hend Ben Moussa}
\end{quote}
\end{minipage} \\
\midrule()
\endhead
\bottomrule()
\end{longtable}

\begin{quote}
\textbf{Dédicace}
\end{quote}

\textbf{À ma chère mère Mahbouba}\\
Vous êtes bien plus qu'une source de gentillesse et d'amour, vous êtes
le phare qui guide mes pas, la lumière qui illumine ma vie. Votre
présence est une bénédiction, une symphonie de bonté et de bienveillance
qui emplit mon cœur de gratitude.

\textbf{À mon cher père Mohamed}\\
Ta guidance éclairée a toujours été mon ancre. Ta patience et ton
soutien inconditionnel ont étémes repères. Je souhaite saluer l'homme
extraordinaire que tu es, une source d'inspiration sans fin.Puisse le
destin te gratifier d'une santé florissante et d'une joie sans bornes.

\textbf{À mes frères et sœurs}\\
Vous êtes mes piliers, toujours là quand j'ai besoin de soutien.
Ensemble, nous formons uneéquipe solide, unis par un amour
inconditionnel. Chacun de vous apporte de la chaleur à mes journées et
de la force à mon cœur. Je suis reconnaissant pour chacun de vous et je
chéris nos moments ensemble.

\begin{quote}
\textbf{À tous mes amis}\\
Ceux qui m'ont offert un soutien moral et émotionnel inestimable durant
cette période de stage. Je vous suis profondément reconnaissant pour
chaque mot d'encouragement, chaque geste de soutien, et d'avoir été une
épaule solide sur laquelle m'appuyer.

À \textbf{tous ceux} qui m'ont aidé à réaliser ce travail, je vous dédie
ce travail à tous.
\end{quote}

% Simplified: avoid longtable footer
\noindent Dhia Abdelli\hfill Page iii

\begin{quote}
\textbf{REMERCIEMENT}

Au cours de cette année, nous sommes empreints d'une profonde gratitude
envers toutes les personnes qui, de près ou de loin, ont joué un rôle
crucial dans la réalisation de notre projet au cours de cette année
particulière et enrichissante.

Tout d'abord, nous tenons à exprimer notre sincère reconnaissance envers
Monsieur \textbf{Ramzi Ben Slimen}, Directeur Technique chez Advancia.
Son soutien indéfectible, ses conseils éclairés et ses remarques
constructives ont été des piliers fondamentaux pour notre progression,
illuminant notre chemin vers le succès.

Nous voudrions également rendre hommage à Madame \textbf{Hend Ben
Moussa}, notre encadrante académique. Elle a fait preuve d'une
disponibilité inébranlable et d'une patience inestimable, des qualités
qui ont été des atouts majeurs tout au long de ce projet. Ses conseils
avisés ont étéune source d'inspiration, propulsant notre travail vers de
nouveaux sommets.

Un profond remerciement s'adresse également à nos enseignants à ESPRIT
pour leur dévouement inlassable envers notre formation. Leur expertise
et leur engagement ont été des piliers essentiels dans notre
apprentissage.

Enfin, nous adressons nos salutations les plus chaleureuses aux membres
du jury. Leur intérêt et leur évaluation de notre projet nous sont
précieux, et nous les remercions pour leur contribution à notre
parcours.
\end{quote}

\begin{longtable}[]{@{}
  >{\raggedright\arraybackslash}p{(\columnwidth - 2\tabcolsep) * \real{0.5000}}
  >{\raggedright\arraybackslash}p{(\columnwidth - 2\tabcolsep) * \real{0.5000}}@{}}
\toprule()
\begin{minipage}[b]{\linewidth}\raggedright
\begin{quote}
Dhia Abdelli
\end{quote}
\end{minipage} & \begin{minipage}[b]{\linewidth}\raggedright
Page iv
\end{minipage} \\
\midrule()
\endhead
\bottomrule()
\end{longtable}

\textbf{Dhia Abdelli}

\textbf{\uline{Résumé :}}\\
En vue d'optimiser la sécurité de nos clients face aux cyberattaques,
Advancia propose la mise en place d'une solution SOAR (Security
Orchestration, Automation, and Response). Cette approche intègre des
mécanismes de détection avancés et une réponse automatisée, garantissant
une protection proactive et efficace de l'infrastructure de nos clients.
SOAR représente une avancée significative dans notre engagement envers
la sécurité informatique, fournissant une défense robuste contre les
menaces émergentes.

\textbf{\uline{Mots clés :}} cyberattaques, SOAR, Security
Orchestration, Automation, Response, mécanismes de détection, réponse
automatisée, défense robuste.

\textbf{\uline{Abstract :}}\\
In order to enhance the security of our clients against cyberattacks,
Advancia proposes the implementation of a SOAR (Security Orchestration,
Automation, and Response) solution. This approach integrates advanced
detection mechanisms and automated response, ensuring proactive and
effective protection of our clients' infrastructure. SOAR represents a
significant advancement in our commitment to cybersecurity, providing
robust defense against emerging threats.

\textbf{\uline{Key-words :}} cyberattacks, SOAR, Security Orchestration,
Automation, Response, detection mec\uline{hanisms, aut}omated response,
robust defense

\begin{quote}
\textbf{TABLE DES MATIÈRES}

\textbf{LISTE DES FIGURES} \textbf{xi}
\end{quote}

\begin{longtable}[]{@{}
  >{\raggedright\arraybackslash}p{(\columnwidth - 22\tabcolsep) * \real{0.0833}}
  >{\raggedright\arraybackslash}p{(\columnwidth - 22\tabcolsep) * \real{0.0833}}
  >{\raggedright\arraybackslash}p{(\columnwidth - 22\tabcolsep) * \real{0.0833}}
  >{\raggedright\arraybackslash}p{(\columnwidth - 22\tabcolsep) * \real{0.0833}}
  >{\raggedright\arraybackslash}p{(\columnwidth - 22\tabcolsep) * \real{0.0833}}
  >{\raggedright\arraybackslash}p{(\columnwidth - 22\tabcolsep) * \real{0.0833}}
  >{\raggedright\arraybackslash}p{(\columnwidth - 22\tabcolsep) * \real{0.0833}}
  >{\raggedright\arraybackslash}p{(\columnwidth - 22\tabcolsep) * \real{0.0833}}
  >{\raggedright\arraybackslash}p{(\columnwidth - 22\tabcolsep) * \real{0.0833}}
  >{\raggedright\arraybackslash}p{(\columnwidth - 22\tabcolsep) * \real{0.0833}}
  >{\raggedright\arraybackslash}p{(\columnwidth - 22\tabcolsep) * \real{0.0833}}
  >{\raggedright\arraybackslash}p{(\columnwidth - 22\tabcolsep) * \real{0.0833}}@{}}
\toprule()
\multicolumn{11}{@{}>{\raggedright\arraybackslash}p{(\columnwidth - 22\tabcolsep) * \real{0.9167} + 20\tabcolsep}}{%
\begin{minipage}[b]{\linewidth}\raggedright
\begin{quote}
\textbf{LISTE DES ACRONYMES}
\end{quote}
\end{minipage}} & \begin{minipage}[b]{\linewidth}\raggedright
\textbf{xiii}
\end{minipage} \\
\midrule()
\endhead
\multicolumn{11}{@{}>{\raggedright\arraybackslash}p{(\columnwidth - 22\tabcolsep) * \real{0.9167} + 20\tabcolsep}}{%
\begin{minipage}[t]{\linewidth}\raggedright
\begin{quote}
\textbf{INTRODUCTION GÉNÉRALE}
\end{quote}
\end{minipage}} & \textbf{xiv} \\
\begin{minipage}[t]{\linewidth}\raggedright
\begin{quote}
\textbf{1}
\end{quote}
\end{minipage} &
\multicolumn{10}{>{\raggedright\arraybackslash}p{(\columnwidth - 22\tabcolsep) * \real{0.8333} + 18\tabcolsep}}{%
\begin{minipage}[t]{\linewidth}\raggedright
\begin{quote}
\textbf{Contexte général du projet}
\end{quote}
\end{minipage}} & \textbf{2} \\
\multicolumn{2}{@{}>{\raggedright\arraybackslash}p{(\columnwidth - 22\tabcolsep) * \real{0.1667} + 2\tabcolsep}}{%
1.1} &
\multicolumn{9}{>{\raggedright\arraybackslash}p{(\columnwidth - 22\tabcolsep) * \real{0.7500} + 16\tabcolsep}}{%
Introduction . . . . . . . . . . . . . . . . . . . . . . . . . . . . . .
. . . . . .} & 3 \\
\multicolumn{2}{@{}>{\raggedright\arraybackslash}p{(\columnwidth - 22\tabcolsep) * \real{0.1667} + 2\tabcolsep}}{%
1.2} &
\multicolumn{8}{>{\raggedright\arraybackslash}p{(\columnwidth - 22\tabcolsep) * \real{0.6667} + 14\tabcolsep}}{%
Présentation de l'organisme d'accueil} & . . . . . . . . . . . . . . . .
. . . . . . & 3 \\
\multicolumn{3}{@{}>{\raggedright\arraybackslash}p{(\columnwidth - 22\tabcolsep) * \real{0.2500} + 4\tabcolsep}}{%
1.2.1} &
\multicolumn{8}{>{\raggedright\arraybackslash}p{(\columnwidth - 22\tabcolsep) * \real{0.6667} + 14\tabcolsep}}{%
\begin{minipage}[t]{\linewidth}\raggedright
\begin{quote}
Géneralités . . . . . . . . . . . . . . . . . . . . . . . . . . . . . .
. .
\end{quote}
\end{minipage}} & 3 \\
\multicolumn{3}{@{}>{\raggedright\arraybackslash}p{(\columnwidth - 22\tabcolsep) * \real{0.2500} + 4\tabcolsep}}{%
1.2.2} &
\multicolumn{8}{>{\raggedright\arraybackslash}p{(\columnwidth - 22\tabcolsep) * \real{0.6667} + 14\tabcolsep}}{%
\begin{minipage}[t]{\linewidth}\raggedright
\begin{quote}
Organisation Interne . . . . . . . . . . . . . . . . . . . . . . . . . .
.
\end{quote}
\end{minipage}} & 4 \\
\multicolumn{3}{@{}>{\raggedright\arraybackslash}p{(\columnwidth - 22\tabcolsep) * \real{0.2500} + 4\tabcolsep}}{%
1.2.3} &
\multicolumn{8}{>{\raggedright\arraybackslash}p{(\columnwidth - 22\tabcolsep) * \real{0.6667} + 14\tabcolsep}}{%
\begin{minipage}[t]{\linewidth}\raggedright
\begin{quote}
Services . . . . . . . . . . . . . . . . . . . . . . . . . . . . . . . .
. .
\end{quote}
\end{minipage}} & 4 \\
\multicolumn{3}{@{}>{\raggedright\arraybackslash}p{(\columnwidth - 22\tabcolsep) * \real{0.2500} + 4\tabcolsep}}{%
1.2.4} & Clients &
\multicolumn{7}{>{\raggedright\arraybackslash}p{(\columnwidth - 22\tabcolsep) * \real{0.5833} + 12\tabcolsep}}{%
. . . . . . . . . . . . . . . . . . . . . . . . . . . . . . . . . .} &
5 \\
\multicolumn{3}{@{}>{\raggedright\arraybackslash}p{(\columnwidth - 22\tabcolsep) * \real{0.2500} + 4\tabcolsep}}{%
1.2.5} &
\multicolumn{2}{>{\raggedright\arraybackslash}p{(\columnwidth - 22\tabcolsep) * \real{0.1667} + 2\tabcolsep}}{%
\begin{minipage}[t]{\linewidth}\raggedright
\begin{quote}
Partenaires
\end{quote}
\end{minipage}} &
\multicolumn{6}{>{\raggedright\arraybackslash}p{(\columnwidth - 22\tabcolsep) * \real{0.5000} + 10\tabcolsep}}{%
. . . . . . . . . . . . . . . . . . . . . . . . . . . . . . . .} & 6 \\
\multicolumn{2}{@{}>{\raggedright\arraybackslash}p{(\columnwidth - 22\tabcolsep) * \real{0.1667} + 2\tabcolsep}}{%
1.3} &
\multicolumn{9}{>{\raggedright\arraybackslash}p{(\columnwidth - 22\tabcolsep) * \real{0.7500} + 16\tabcolsep}}{%
Contexte général du Projet . . . . . . . . . . . . . . . . . . . . . . .
. . . . .} & 6 \\
\multicolumn{3}{@{}>{\raggedright\arraybackslash}p{(\columnwidth - 22\tabcolsep) * \real{0.2500} + 4\tabcolsep}}{%
1.3.1} &
\multicolumn{5}{>{\raggedright\arraybackslash}p{(\columnwidth - 22\tabcolsep) * \real{0.4167} + 8\tabcolsep}}{%
Présentation du projet} &
\multicolumn{3}{>{\raggedright\arraybackslash}p{(\columnwidth - 22\tabcolsep) * \real{0.2500} + 4\tabcolsep}}{%
. . . . . . . . . . . . . . . . . . . . . . . . . .} & 6 \\
\multicolumn{3}{@{}>{\raggedright\arraybackslash}p{(\columnwidth - 22\tabcolsep) * \real{0.2500} + 4\tabcolsep}}{%
1.3.2} &
\multicolumn{3}{>{\raggedright\arraybackslash}p{(\columnwidth - 22\tabcolsep) * \real{0.2500} + 4\tabcolsep}}{%
\begin{minipage}[t]{\linewidth}\raggedright
\begin{quote}
Problématique
\end{quote}
\end{minipage}} &
\multicolumn{5}{>{\raggedright\arraybackslash}p{(\columnwidth - 22\tabcolsep) * \real{0.4167} + 8\tabcolsep}}{%
. . . . . . . . . . . . . . . . . . . . . . . . . . . . . .} & 7 \\
\multicolumn{3}{@{}>{\raggedright\arraybackslash}p{(\columnwidth - 22\tabcolsep) * \real{0.2500} + 4\tabcolsep}}{%
1.3.3} &
\multicolumn{4}{>{\raggedright\arraybackslash}p{(\columnwidth - 22\tabcolsep) * \real{0.3333} + 6\tabcolsep}}{%
Solution proposée} &
\multicolumn{4}{>{\raggedright\arraybackslash}p{(\columnwidth - 22\tabcolsep) * \real{0.3333} + 6\tabcolsep}}{%
. . . . . . . . . . . . . . . . . . . . . . . . . . . .} & 8 \\
\multicolumn{2}{@{}>{\raggedright\arraybackslash}p{(\columnwidth - 22\tabcolsep) * \real{0.1667} + 2\tabcolsep}}{%
1.4} &
\multicolumn{9}{>{\raggedright\arraybackslash}p{(\columnwidth - 22\tabcolsep) * \real{0.7500} + 16\tabcolsep}}{%
Méthodologie de travail . . . . . . . . . . . . . . . . . . . . . . . .
. . . . . .} & 8 \\
\multicolumn{3}{@{}>{\raggedright\arraybackslash}p{(\columnwidth - 22\tabcolsep) * \real{0.2500} + 4\tabcolsep}}{%
1.4.1} &
\multicolumn{8}{>{\raggedright\arraybackslash}p{(\columnwidth - 22\tabcolsep) * \real{0.6667} + 14\tabcolsep}}{%
\begin{minipage}[t]{\linewidth}\raggedright
\begin{quote}
Méthode classique . . . . . . . . . . . . . . . . . . . . . . . . . . .
.
\end{quote}
\end{minipage}} & 8 \\
\multicolumn{3}{@{}>{\raggedright\arraybackslash}p{(\columnwidth - 22\tabcolsep) * \real{0.2500} + 4\tabcolsep}}{%
1.4.2} &
\multicolumn{8}{>{\raggedright\arraybackslash}p{(\columnwidth - 22\tabcolsep) * \real{0.6667} + 14\tabcolsep}}{%
\begin{minipage}[t]{\linewidth}\raggedright
\begin{quote}
Méthode AGILE . . . . . . . . . . . . . . . . . . . . . . . . . . . . .
\end{quote}
\end{minipage}} & 9 \\
\multicolumn{4}{@{}>{\raggedright\arraybackslash}p{(\columnwidth - 22\tabcolsep) * \real{0.3333} + 6\tabcolsep}}{%
1.4.2.1} &
\multicolumn{5}{>{\raggedright\arraybackslash}p{(\columnwidth - 22\tabcolsep) * \real{0.4167} + 8\tabcolsep}}{%
Méthode Kanban} &
\multicolumn{2}{>{\raggedright\arraybackslash}p{(\columnwidth - 22\tabcolsep) * \real{0.1667} + 2\tabcolsep}}{%
. . . . . . . . . . . . . . . . . . . . . . .} & 9 \\
\multicolumn{3}{@{}>{\raggedright\arraybackslash}p{(\columnwidth - 22\tabcolsep) * \real{0.2500} + 4\tabcolsep}}{%
1.4.3} &
\multicolumn{5}{>{\raggedright\arraybackslash}p{(\columnwidth - 22\tabcolsep) * \real{0.4167} + 8\tabcolsep}}{%
\begin{minipage}[t]{\linewidth}\raggedright
\begin{quote}
Planification du projet
\end{quote}
\end{minipage}} &
\multicolumn{3}{>{\raggedright\arraybackslash}p{(\columnwidth - 22\tabcolsep) * \real{0.2500} + 4\tabcolsep}}{%
. . . . . . . . . . . . . . . . . . . . . . . . . .} & 10 \\
\multicolumn{2}{@{}>{\raggedright\arraybackslash}p{(\columnwidth - 22\tabcolsep) * \real{0.1667} + 2\tabcolsep}}{%
1.5} &
\multicolumn{9}{>{\raggedright\arraybackslash}p{(\columnwidth - 22\tabcolsep) * \real{0.7500} + 16\tabcolsep}}{%
CONCLUSION . . . . . . . . . . . . . . . . . . . . . . . . . . . . . . .
. . .} & 11 \\
\begin{minipage}[t]{\linewidth}\raggedright
\begin{quote}
\textbf{2}
\end{quote}
\end{minipage} &
\multicolumn{10}{>{\raggedright\arraybackslash}p{(\columnwidth - 22\tabcolsep) * \real{0.8333} + 18\tabcolsep}}{%
\begin{minipage}[t]{\linewidth}\raggedright
\begin{quote}
\textbf{État de l'art}
\end{quote}
\end{minipage}} & \textbf{12} \\
\multicolumn{2}{@{}>{\raggedright\arraybackslash}p{(\columnwidth - 22\tabcolsep) * \real{0.1667} + 2\tabcolsep}}{%
2.1} &
\multicolumn{9}{>{\raggedright\arraybackslash}p{(\columnwidth - 22\tabcolsep) * \real{0.7500} + 16\tabcolsep}}{%
Les Piliers de la Cybersécurité . . . . . . . . . . . . . . . . . . . .
. . . . . .} & 13 \\
\multicolumn{3}{@{}>{\raggedright\arraybackslash}p{(\columnwidth - 22\tabcolsep) * \real{0.2500} + 4\tabcolsep}}{%
2.1.1} &
\multicolumn{8}{>{\raggedright\arraybackslash}p{(\columnwidth - 22\tabcolsep) * \real{0.6667} + 14\tabcolsep}}{%
\begin{minipage}[t]{\linewidth}\raggedright
\begin{quote}
Cybersécurité . . . . . . . . . . . . . . . . . . . . . . . . . . . . .
. .
\end{quote}
\end{minipage}} & 13 \\
\multicolumn{3}{@{}>{\raggedright\arraybackslash}p{(\columnwidth - 22\tabcolsep) * \real{0.2500} + 4\tabcolsep}}{%
2.1.2} &
\multicolumn{8}{>{\raggedright\arraybackslash}p{(\columnwidth - 22\tabcolsep) * \real{0.6667} + 14\tabcolsep}}{%
\begin{minipage}[t]{\linewidth}\raggedright
\begin{quote}
Cyberattaque . . . . . . . . . . . . . . . . . . . . . . . . . . . . . .
.
\end{quote}
\end{minipage}} & 13 \\
\multicolumn{3}{@{}>{\raggedright\arraybackslash}p{(\columnwidth - 22\tabcolsep) * \real{0.2500} + 4\tabcolsep}}{%
\begin{minipage}[t]{\linewidth}\raggedright
\begin{quote}
Dhia Abdelli
\end{quote}
\end{minipage}} &
\multicolumn{9}{>{\raggedright\arraybackslash}p{(\columnwidth - 22\tabcolsep) * \real{0.7500} + 16\tabcolsep}@{}}{%
Page vi} \\
\bottomrule()
\end{longtable}

\begin{quote}
TABLE DES MATIÈRES
\end{quote}

\begin{longtable}[]{@{}
  >{\raggedright\arraybackslash}p{(\columnwidth - 20\tabcolsep) * \real{0.0909}}
  >{\raggedright\arraybackslash}p{(\columnwidth - 20\tabcolsep) * \real{0.0909}}
  >{\raggedright\arraybackslash}p{(\columnwidth - 20\tabcolsep) * \real{0.0909}}
  >{\raggedright\arraybackslash}p{(\columnwidth - 20\tabcolsep) * \real{0.0909}}
  >{\raggedright\arraybackslash}p{(\columnwidth - 20\tabcolsep) * \real{0.0909}}
  >{\raggedright\arraybackslash}p{(\columnwidth - 20\tabcolsep) * \real{0.0909}}
  >{\raggedright\arraybackslash}p{(\columnwidth - 20\tabcolsep) * \real{0.0909}}
  >{\raggedright\arraybackslash}p{(\columnwidth - 20\tabcolsep) * \real{0.0909}}
  >{\raggedright\arraybackslash}p{(\columnwidth - 20\tabcolsep) * \real{0.0909}}
  >{\raggedright\arraybackslash}p{(\columnwidth - 20\tabcolsep) * \real{0.0909}}
  >{\raggedright\arraybackslash}p{(\columnwidth - 20\tabcolsep) * \real{0.0909}}@{}}
\toprule()
\multicolumn{3}{@{}>{\raggedright\arraybackslash}p{(\columnwidth - 20\tabcolsep) * \real{0.2727} + 4\tabcolsep}}{%
\begin{minipage}[b]{\linewidth}\raggedright
2.1.3
\end{minipage}} &
\multicolumn{7}{>{\raggedright\arraybackslash}p{(\columnwidth - 20\tabcolsep) * \real{0.6364} + 12\tabcolsep}}{%
\begin{minipage}[b]{\linewidth}\raggedright
Le modèle triade CIA . . . . . . . . . . . . . . . . . . . . . . . . . .
.
\end{minipage}} & \begin{minipage}[b]{\linewidth}\raggedright
14
\end{minipage} \\
\midrule()
\endhead
\multicolumn{2}{@{}>{\raggedright\arraybackslash}p{(\columnwidth - 20\tabcolsep) * \real{0.1818} + 2\tabcolsep}}{%
2.2} &
\multicolumn{8}{>{\raggedright\arraybackslash}p{(\columnwidth - 20\tabcolsep) * \real{0.7273} + 14\tabcolsep}}{%
SOC . . . . . . . . . . . . . . . . . . . . . . . . . . . . . . . . . .
. . . . . .} & 15 \\
\multicolumn{3}{@{}>{\raggedright\arraybackslash}p{(\columnwidth - 20\tabcolsep) * \real{0.2727} + 4\tabcolsep}}{%
2.2.1} &
\multicolumn{7}{>{\raggedright\arraybackslash}p{(\columnwidth - 20\tabcolsep) * \real{0.6364} + 12\tabcolsep}}{%
Processus de réponse aux incidents dans un SOC . . . . . . . . . . . .}
& 15 \\
\multicolumn{3}{@{}>{\raggedright\arraybackslash}p{(\columnwidth - 20\tabcolsep) * \real{0.2727} + 4\tabcolsep}}{%
2.2.2} &
\multicolumn{3}{>{\raggedright\arraybackslash}p{(\columnwidth - 20\tabcolsep) * \real{0.2727} + 4\tabcolsep}}{%
Défis SOC} &
\multicolumn{4}{>{\raggedright\arraybackslash}p{(\columnwidth - 20\tabcolsep) * \real{0.3636} + 6\tabcolsep}}{%
. . . . . . . . . . . . . . . . . . . . . . . . . . . . . . . .} & 16 \\
\multicolumn{2}{@{}>{\raggedright\arraybackslash}p{(\columnwidth - 20\tabcolsep) * \real{0.1818} + 2\tabcolsep}}{%
2.3} &
\multicolumn{8}{>{\raggedright\arraybackslash}p{(\columnwidth - 20\tabcolsep) * \real{0.7273} + 14\tabcolsep}}{%
SOAR . . . . . . . . . . . . . . . . . . . . . . . . . . . . . . . . . .
. . . . .} & 17 \\
\multicolumn{3}{@{}>{\raggedright\arraybackslash}p{(\columnwidth - 20\tabcolsep) * \real{0.2727} + 4\tabcolsep}}{%
2.3.1} &
\multicolumn{7}{>{\raggedright\arraybackslash}p{(\columnwidth - 20\tabcolsep) * \real{0.6364} + 12\tabcolsep}}{%
Que signifie "SOAR"? . . . . . . . . . . . . . . . . . . . . . . . . .
.} & 18 \\
\multicolumn{3}{@{}>{\raggedright\arraybackslash}p{(\columnwidth - 20\tabcolsep) * \real{0.2727} + 4\tabcolsep}}{%
2.3.2} &
\multicolumn{7}{>{\raggedright\arraybackslash}p{(\columnwidth - 20\tabcolsep) * \real{0.6364} + 12\tabcolsep}}{%
Pourquoi SOAR est-il important? . . . . . . . . . . . . . . . . . . . .}
& 18 \\
\multicolumn{3}{@{}>{\raggedright\arraybackslash}p{(\columnwidth - 20\tabcolsep) * \real{0.2727} + 4\tabcolsep}}{%
2.3.3} &
\multicolumn{7}{>{\raggedright\arraybackslash}p{(\columnwidth - 20\tabcolsep) * \real{0.6364} + 12\tabcolsep}}{%
Composants SOAR . . . . . . . . . . . . . . . . . . . . . . . . . . . .}
& 19 \\
\multicolumn{5}{@{}>{\raggedright\arraybackslash}p{(\columnwidth - 20\tabcolsep) * \real{0.4545} + 8\tabcolsep}}{%
2.3.3.1} &
\multicolumn{5}{>{\raggedright\arraybackslash}p{(\columnwidth - 20\tabcolsep) * \real{0.4545} + 8\tabcolsep}}{%
Tableau de bord unifié . . . . . . . . . . . . . . . . . . . . .} &
20 \\
\multicolumn{5}{@{}>{\raggedright\arraybackslash}p{(\columnwidth - 20\tabcolsep) * \real{0.4545} + 8\tabcolsep}}{%
2.3.3.2} &
\multicolumn{5}{>{\raggedright\arraybackslash}p{(\columnwidth - 20\tabcolsep) * \real{0.4545} + 8\tabcolsep}}{%
Playbooks . . . . . . . . . . . . . . . . . . . . . . . . . . .} & 20 \\
\multicolumn{5}{@{}>{\raggedright\arraybackslash}p{(\columnwidth - 20\tabcolsep) * \real{0.4545} + 8\tabcolsep}}{%
2.3.3.3} &
\multicolumn{5}{>{\raggedright\arraybackslash}p{(\columnwidth - 20\tabcolsep) * \real{0.4545} + 8\tabcolsep}}{%
Gestion des renseignements sur les menaces . . . . . . . . .} & 21 \\
\multicolumn{5}{@{}>{\raggedright\arraybackslash}p{(\columnwidth - 20\tabcolsep) * \real{0.4545} + 8\tabcolsep}}{%
2.3.3.4} &
\multicolumn{5}{>{\raggedright\arraybackslash}p{(\columnwidth - 20\tabcolsep) * \real{0.4545} + 8\tabcolsep}}{%
Gestion de cas . . . . . . . . . . . . . . . . . . . . . . . . .} &
21 \\
\multicolumn{5}{@{}>{\raggedright\arraybackslash}p{(\columnwidth - 20\tabcolsep) * \real{0.4545} + 8\tabcolsep}}{%
2.3.3.5} &
\multicolumn{5}{>{\raggedright\arraybackslash}p{(\columnwidth - 20\tabcolsep) * \real{0.4545} + 8\tabcolsep}}{%
Moteur d'orchestration et d'automatisation . . . . . . . . . .} & 21 \\
\multicolumn{3}{@{}>{\raggedright\arraybackslash}p{(\columnwidth - 20\tabcolsep) * \real{0.2727} + 4\tabcolsep}}{%
2.3.4} &
\multicolumn{5}{>{\raggedright\arraybackslash}p{(\columnwidth - 20\tabcolsep) * \real{0.4545} + 8\tabcolsep}}{%
Solution SOAR dans un SOC} &
\multicolumn{2}{>{\raggedright\arraybackslash}p{(\columnwidth - 20\tabcolsep) * \real{0.1818} + 2\tabcolsep}}{%
. . . . . . . . . . . . . . . . . . . . . .} & 22 \\
\multicolumn{2}{@{}>{\raggedright\arraybackslash}p{(\columnwidth - 20\tabcolsep) * \real{0.1818} + 2\tabcolsep}}{%
2.4} &
\multicolumn{8}{>{\raggedright\arraybackslash}p{(\columnwidth - 20\tabcolsep) * \real{0.7273} + 14\tabcolsep}}{%
Solutions existantes . . . . . . . . . . . . . . . . . . . . . . . . . .
. . . . . .} & 23 \\
\multicolumn{2}{@{}>{\raggedright\arraybackslash}p{(\columnwidth - 20\tabcolsep) * \real{0.1818} + 2\tabcolsep}}{%
2.5} &
\multicolumn{8}{>{\raggedright\arraybackslash}p{(\columnwidth - 20\tabcolsep) * \real{0.7273} + 14\tabcolsep}}{%
Choix des outils . . . . . . . . . . . . . . . . . . . . . . . . . . . .
. . . . . .} & 24 \\
\multicolumn{3}{@{}>{\raggedright\arraybackslash}p{(\columnwidth - 20\tabcolsep) * \real{0.2727} + 4\tabcolsep}}{%
2.5.1} &
\multicolumn{7}{>{\raggedright\arraybackslash}p{(\columnwidth - 20\tabcolsep) * \real{0.6364} + 12\tabcolsep}}{%
SIEM (Elastic Stack) . . . . . . . . . . . . . . . . . . . . . . . . . .
.} & 24 \\
\multicolumn{3}{@{}>{\raggedright\arraybackslash}p{(\columnwidth - 20\tabcolsep) * \real{0.2727} + 4\tabcolsep}}{%
2.5.2} &
\multicolumn{7}{>{\raggedright\arraybackslash}p{(\columnwidth - 20\tabcolsep) * \real{0.6364} + 12\tabcolsep}}{%
Outil de renseignement sur les menaces (MISP) . . . . . . . . . . . . .}
& 24 \\
\multicolumn{3}{@{}>{\raggedright\arraybackslash}p{(\columnwidth - 20\tabcolsep) * \real{0.2727} + 4\tabcolsep}}{%
2.5.3} &
\multicolumn{7}{>{\raggedright\arraybackslash}p{(\columnwidth - 20\tabcolsep) * \real{0.6364} + 12\tabcolsep}}{%
Solution de réponse aux incidents de sécurité (DFIR IRIS) . . . . . . .}
& 25 \\
\multicolumn{3}{@{}>{\raggedright\arraybackslash}p{(\columnwidth - 20\tabcolsep) * \real{0.2727} + 4\tabcolsep}}{%
2.5.4} &
\multicolumn{7}{>{\raggedright\arraybackslash}p{(\columnwidth - 20\tabcolsep) * \real{0.6364} + 12\tabcolsep}}{%
Outil d'orchestration (Shuffle) . . . . . . . . . . . . . . . . . . . .
. .} & 25 \\
\multicolumn{3}{@{}>{\raggedright\arraybackslash}p{(\columnwidth - 20\tabcolsep) * \real{0.2727} + 4\tabcolsep}}{%
2.5.5} &
\multicolumn{7}{>{\raggedright\arraybackslash}p{(\columnwidth - 20\tabcolsep) * \real{0.6364} + 12\tabcolsep}}{%
Plateforme Honeypot (T-POT) . . . . . . . . . . . . . . . . . . . . . .}
& 26 \\
\multicolumn{3}{@{}>{\raggedright\arraybackslash}p{(\columnwidth - 20\tabcolsep) * \real{0.2727} + 4\tabcolsep}}{%
2.5.6} &
\multicolumn{7}{>{\raggedright\arraybackslash}p{(\columnwidth - 20\tabcolsep) * \real{0.6364} + 12\tabcolsep}}{%
Outil de collecte de données T-POT GreedyBear (GreedyBear) . . . . .} &
27 \\
\multicolumn{3}{@{}>{\raggedright\arraybackslash}p{(\columnwidth - 20\tabcolsep) * \real{0.2727} + 4\tabcolsep}}{%
2.5.7} &
\multicolumn{6}{>{\raggedright\arraybackslash}p{(\columnwidth - 20\tabcolsep) * \real{0.5455} + 10\tabcolsep}}{%
Outil d'analyse des logiciels malveillants (Cuckoo Sanbox)} & . . . . .
. & 27 \\
\multicolumn{3}{@{}>{\raggedright\arraybackslash}p{(\columnwidth - 20\tabcolsep) * \real{0.2727} + 4\tabcolsep}}{%
2.5.8} &
\multicolumn{7}{>{\raggedright\arraybackslash}p{(\columnwidth - 20\tabcolsep) * \real{0.6364} + 12\tabcolsep}}{%
Outil d'alerte (Praeco) . . . . . . . . . . . . . . . . . . . . . . . .
. .} & 28 \\
\multicolumn{3}{@{}>{\raggedright\arraybackslash}p{(\columnwidth - 20\tabcolsep) * \real{0.2727} + 4\tabcolsep}}{%
2.5.9} &
\multicolumn{8}{>{\raggedright\arraybackslash}p{(\columnwidth - 20\tabcolsep) * \real{0.7273} + 14\tabcolsep}@{}}{%
\begin{minipage}[t]{\linewidth}\raggedright
\begin{quote}
Outil d'investigation numérique et de réponse aux incidents
(Velociraptor) 28
\end{quote}
\end{minipage}} \\
\multicolumn{2}{@{}>{\raggedright\arraybackslash}p{(\columnwidth - 20\tabcolsep) * \real{0.1818} + 2\tabcolsep}}{%
2.6} &
\multicolumn{8}{>{\raggedright\arraybackslash}p{(\columnwidth - 20\tabcolsep) * \real{0.7273} + 14\tabcolsep}}{%
CONCLUSION . . . . . . . . . . . . . . . . . . . . . . . . . . . . . . .
. . .} & 29 \\
\begin{minipage}[t]{\linewidth}\raggedright
\begin{quote}
\textbf{3}
\end{quote}
\end{minipage} &
\multicolumn{9}{>{\raggedright\arraybackslash}p{(\columnwidth - 20\tabcolsep) * \real{0.8182} + 16\tabcolsep}}{%
\begin{minipage}[t]{\linewidth}\raggedright
\begin{quote}
\textbf{Analyse et conception}
\end{quote}
\end{minipage}} & \textbf{30} \\
\multicolumn{2}{@{}>{\raggedright\arraybackslash}p{(\columnwidth - 20\tabcolsep) * \real{0.1818} + 2\tabcolsep}}{%
3.1} &
\multicolumn{8}{>{\raggedright\arraybackslash}p{(\columnwidth - 20\tabcolsep) * \real{0.7273} + 14\tabcolsep}}{%
Outil de modélisation . . . . . . . . . . . . . . . . . . . . . . . . .
. . . . . .} & 31 \\
\multicolumn{2}{@{}>{\raggedright\arraybackslash}p{(\columnwidth - 20\tabcolsep) * \real{0.1818} + 2\tabcolsep}}{%
3.2} &
\multicolumn{8}{>{\raggedright\arraybackslash}p{(\columnwidth - 20\tabcolsep) * \real{0.7273} + 14\tabcolsep}}{%
Identifier les acteurs . . . . . . . . . . . . . . . . . . . . . . . . .
. . . . . . .} & 31 \\
\multicolumn{2}{@{}>{\raggedright\arraybackslash}p{(\columnwidth - 20\tabcolsep) * \real{0.1818} + 2\tabcolsep}}{%
3.3} &
\multicolumn{8}{>{\raggedright\arraybackslash}p{(\columnwidth - 20\tabcolsep) * \real{0.7273} + 14\tabcolsep}}{%
Spécification des besoins . . . . . . . . . . . . . . . . . . . . . . .
. . . . . .} & 32 \\
\multicolumn{3}{@{}>{\raggedright\arraybackslash}p{(\columnwidth - 20\tabcolsep) * \real{0.2727} + 4\tabcolsep}}{%
3.3.1} &
\multicolumn{7}{>{\raggedright\arraybackslash}p{(\columnwidth - 20\tabcolsep) * \real{0.6364} + 12\tabcolsep}}{%
Exigences fonctionnelles . . . . . . . . . . . . . . . . . . . . . . . .
.} & 32 \\
\multicolumn{3}{@{}>{\raggedright\arraybackslash}p{(\columnwidth - 20\tabcolsep) * \real{0.2727} + 4\tabcolsep}}{%
3.3.2} &
\multicolumn{5}{>{\raggedright\arraybackslash}p{(\columnwidth - 20\tabcolsep) * \real{0.4545} + 8\tabcolsep}}{%
Exigences non fonctionnelles} &
\multicolumn{2}{>{\raggedright\arraybackslash}p{(\columnwidth - 20\tabcolsep) * \real{0.1818} + 2\tabcolsep}}{%
. . . . . . . . . . . . . . . . . . . . . .} & 33 \\
\multicolumn{2}{@{}>{\raggedright\arraybackslash}p{(\columnwidth - 20\tabcolsep) * \real{0.1818} + 2\tabcolsep}}{%
3.4} &
\multicolumn{2}{>{\raggedright\arraybackslash}p{(\columnwidth - 20\tabcolsep) * \real{0.1818} + 2\tabcolsep}}{%
Conception} &
\multicolumn{6}{>{\raggedright\arraybackslash}p{(\columnwidth - 20\tabcolsep) * \real{0.5455} + 10\tabcolsep}}{%
. . . . . . . . . . . . . . . . . . . . . . . . . . . . . . . . . . . .}
& 33 \\
\multicolumn{3}{@{}>{\raggedright\arraybackslash}p{(\columnwidth - 20\tabcolsep) * \real{0.2727} + 4\tabcolsep}}{%
3.4.1} &
\multicolumn{7}{>{\raggedright\arraybackslash}p{(\columnwidth - 20\tabcolsep) * \real{0.6364} + 12\tabcolsep}}{%
Diagramme de cas d'utilisation global . . . . . . . . . . . . . . . . .
.} & 34 \\
\multicolumn{3}{@{}>{\raggedright\arraybackslash}p{(\columnwidth - 20\tabcolsep) * \real{0.2727} + 4\tabcolsep}}{%
3.4.2} &
\multicolumn{4}{>{\raggedright\arraybackslash}p{(\columnwidth - 20\tabcolsep) * \real{0.3636} + 6\tabcolsep}}{%
Description détaillée du cas} &
\multicolumn{3}{>{\raggedright\arraybackslash}p{(\columnwidth - 20\tabcolsep) * \real{0.2727} + 4\tabcolsep}}{%
. . . . . . . . . . . . . . . . . . . . . . .} & 35 \\
\multicolumn{3}{@{}>{\raggedright\arraybackslash}p{(\columnwidth - 20\tabcolsep) * \real{0.2727} + 4\tabcolsep}}{%
\begin{minipage}[t]{\linewidth}\raggedright
\begin{quote}
Dhia Abdelli
\end{quote}
\end{minipage}} &
\multicolumn{8}{>{\raggedright\arraybackslash}p{(\columnwidth - 20\tabcolsep) * \real{0.7273} + 14\tabcolsep}@{}}{%
Page vii} \\
\bottomrule()
\end{longtable}

\begin{quote}
TABLE DES MATIÈRES
\end{quote}

\begin{longtable}[]{@{}
  >{\raggedright\arraybackslash}p{(\columnwidth - 30\tabcolsep) * \real{0.0625}}
  >{\raggedright\arraybackslash}p{(\columnwidth - 30\tabcolsep) * \real{0.0625}}
  >{\raggedright\arraybackslash}p{(\columnwidth - 30\tabcolsep) * \real{0.0625}}
  >{\raggedright\arraybackslash}p{(\columnwidth - 30\tabcolsep) * \real{0.0625}}
  >{\raggedright\arraybackslash}p{(\columnwidth - 30\tabcolsep) * \real{0.0625}}
  >{\raggedright\arraybackslash}p{(\columnwidth - 30\tabcolsep) * \real{0.0625}}
  >{\raggedright\arraybackslash}p{(\columnwidth - 30\tabcolsep) * \real{0.0625}}
  >{\raggedright\arraybackslash}p{(\columnwidth - 30\tabcolsep) * \real{0.0625}}
  >{\raggedright\arraybackslash}p{(\columnwidth - 30\tabcolsep) * \real{0.0625}}
  >{\raggedright\arraybackslash}p{(\columnwidth - 30\tabcolsep) * \real{0.0625}}
  >{\raggedright\arraybackslash}p{(\columnwidth - 30\tabcolsep) * \real{0.0625}}
  >{\raggedright\arraybackslash}p{(\columnwidth - 30\tabcolsep) * \real{0.0625}}
  >{\raggedright\arraybackslash}p{(\columnwidth - 30\tabcolsep) * \real{0.0625}}
  >{\raggedright\arraybackslash}p{(\columnwidth - 30\tabcolsep) * \real{0.0625}}
  >{\raggedright\arraybackslash}p{(\columnwidth - 30\tabcolsep) * \real{0.0625}}
  >{\raggedright\arraybackslash}p{(\columnwidth - 30\tabcolsep) * \real{0.0625}}@{}}
\toprule()
\multicolumn{5}{@{}>{\raggedright\arraybackslash}p{(\columnwidth - 30\tabcolsep) * \real{0.3125} + 8\tabcolsep}}{%
\begin{minipage}[b]{\linewidth}\raggedright
3.4.2.1
\end{minipage}} &
\multicolumn{6}{>{\raggedright\arraybackslash}p{(\columnwidth - 30\tabcolsep) * \real{0.3750} + 10\tabcolsep}}{%
\begin{minipage}[b]{\linewidth}\raggedright
La gestion de cas
\end{minipage}} &
\multicolumn{4}{>{\raggedright\arraybackslash}p{(\columnwidth - 30\tabcolsep) * \real{0.2500} + 6\tabcolsep}}{%
\begin{minipage}[b]{\linewidth}\raggedright
. . . . . . . . . . . . . . . . . . . . . . .
\end{minipage}} & \begin{minipage}[b]{\linewidth}\raggedright
35
\end{minipage} \\
\midrule()
\endhead
\multicolumn{5}{@{}>{\raggedright\arraybackslash}p{(\columnwidth - 30\tabcolsep) * \real{0.3125} + 8\tabcolsep}}{%
3.4.2.2} &
\multicolumn{10}{>{\raggedright\arraybackslash}p{(\columnwidth - 30\tabcolsep) * \real{0.6250} + 18\tabcolsep}}{%
Gestion des flux de travail . . . . . . . . . . . . . . . . . . .} &
37 \\
\multicolumn{5}{@{}>{\raggedright\arraybackslash}p{(\columnwidth - 30\tabcolsep) * \real{0.3125} + 8\tabcolsep}}{%
3.4.2.3} &
\multicolumn{10}{>{\raggedright\arraybackslash}p{(\columnwidth - 30\tabcolsep) * \real{0.6250} + 18\tabcolsep}}{%
Authentification . . . . . . . . . . . . . . . . . . . . . . . .} &
39 \\
\multicolumn{5}{@{}>{\raggedright\arraybackslash}p{(\columnwidth - 30\tabcolsep) * \real{0.3125} + 8\tabcolsep}}{%
3.4.2.4} &
\multicolumn{10}{>{\raggedright\arraybackslash}p{(\columnwidth - 30\tabcolsep) * \real{0.6250} + 18\tabcolsep}}{%
Gestion des organisations . . . . . . . . . . . . . . . . . . .} & 39 \\
\multicolumn{2}{@{}>{\raggedright\arraybackslash}p{(\columnwidth - 30\tabcolsep) * \real{0.1250} + 2\tabcolsep}}{%
3.5} &
\multicolumn{13}{>{\raggedright\arraybackslash}p{(\columnwidth - 30\tabcolsep) * \real{0.8125} + 24\tabcolsep}}{%
Architecture du projet . . . . . . . . . . . . . . . . . . . . . . . . .
. . . . . .} & 41 \\
\multicolumn{2}{@{}>{\raggedright\arraybackslash}p{(\columnwidth - 30\tabcolsep) * \real{0.1250} + 2\tabcolsep}}{%
3.6} &
\multicolumn{13}{>{\raggedright\arraybackslash}p{(\columnwidth - 30\tabcolsep) * \real{0.8125} + 24\tabcolsep}}{%
CONCLUSION . . . . . . . . . . . . . . . . . . . . . . . . . . . . . . .
. . .} & 42 \\
\begin{minipage}[t]{\linewidth}\raggedright
\begin{quote}
\textbf{4}
\end{quote}
\end{minipage} &
\multicolumn{14}{>{\raggedright\arraybackslash}p{(\columnwidth - 30\tabcolsep) * \real{0.8750} + 26\tabcolsep}}{%
\begin{minipage}[t]{\linewidth}\raggedright
\begin{quote}
\textbf{Réalisation}
\end{quote}
\end{minipage}} & \textbf{43} \\
\multicolumn{2}{@{}>{\raggedright\arraybackslash}p{(\columnwidth - 30\tabcolsep) * \real{0.1250} + 2\tabcolsep}}{%
4.1} &
\multicolumn{13}{>{\raggedright\arraybackslash}p{(\columnwidth - 30\tabcolsep) * \real{0.8125} + 24\tabcolsep}}{%
Inroduction . . . . . . . . . . . . . . . . . . . . . . . . . . . . . .
. . . . . .} & 44 \\
\multicolumn{2}{@{}>{\raggedright\arraybackslash}p{(\columnwidth - 30\tabcolsep) * \real{0.1250} + 2\tabcolsep}}{%
4.2} &
\multicolumn{13}{>{\raggedright\arraybackslash}p{(\columnwidth - 30\tabcolsep) * \real{0.8125} + 24\tabcolsep}}{%
Environnement de travail . . . . . . . . . . . . . . . . . . . . . . . .
. . . . .} & 44 \\
\multicolumn{3}{@{}>{\raggedright\arraybackslash}p{(\columnwidth - 30\tabcolsep) * \real{0.1875} + 4\tabcolsep}}{%
4.2.1} &
\multicolumn{12}{>{\raggedright\arraybackslash}p{(\columnwidth - 30\tabcolsep) * \real{0.7500} + 22\tabcolsep}}{%
Composants logiciels . . . . . . . . . . . . . . . . . . . . . . . . . .
.} & 44 \\
\multicolumn{3}{@{}>{\raggedright\arraybackslash}p{(\columnwidth - 30\tabcolsep) * \real{0.1875} + 4\tabcolsep}}{%
4.2.2} &
\multicolumn{5}{>{\raggedright\arraybackslash}p{(\columnwidth - 30\tabcolsep) * \real{0.3125} + 8\tabcolsep}}{%
Composants matériels} &
\multicolumn{7}{>{\raggedright\arraybackslash}p{(\columnwidth - 30\tabcolsep) * \real{0.4375} + 12\tabcolsep}}{%
. . . . . . . . . . . . . . . . . . . . . . . . . .} & 44 \\
\multicolumn{2}{@{}>{\raggedright\arraybackslash}p{(\columnwidth - 30\tabcolsep) * \real{0.1250} + 2\tabcolsep}}{%
4.3} &
\multicolumn{13}{>{\raggedright\arraybackslash}p{(\columnwidth - 30\tabcolsep) * \real{0.8125} + 24\tabcolsep}}{%
La solution SOAR . . . . . . . . . . . . . . . . . . . . . . . . . . . .
. . . . .} & 46 \\
\multicolumn{3}{@{}>{\raggedright\arraybackslash}p{(\columnwidth - 30\tabcolsep) * \real{0.1875} + 4\tabcolsep}}{%
4.3.1} &
\multicolumn{4}{>{\raggedright\arraybackslash}p{(\columnwidth - 30\tabcolsep) * \real{0.2500} + 6\tabcolsep}}{%
Interface des cas} &
\multicolumn{8}{>{\raggedright\arraybackslash}p{(\columnwidth - 30\tabcolsep) * \real{0.5000} + 14\tabcolsep}}{%
. . . . . . . . . . . . . . . . . . . . . . . . . . . . .} & 46 \\
\multicolumn{3}{@{}>{\raggedright\arraybackslash}p{(\columnwidth - 30\tabcolsep) * \real{0.1875} + 4\tabcolsep}}{%
4.3.2} &
\multicolumn{12}{>{\raggedright\arraybackslash}p{(\columnwidth - 30\tabcolsep) * \real{0.7500} + 22\tabcolsep}}{%
Interface d'alertes . . . . . . . . . . . . . . . . . . . . . . . . . .
. . .} & 46 \\
\multicolumn{3}{@{}>{\raggedright\arraybackslash}p{(\columnwidth - 30\tabcolsep) * \real{0.1875} + 4\tabcolsep}}{%
4.3.3} &
\multicolumn{12}{>{\raggedright\arraybackslash}p{(\columnwidth - 30\tabcolsep) * \real{0.7500} + 22\tabcolsep}}{%
Aperçu de cas . . . . . . . . . . . . . . . . . . . . . . . . . . . . .
. .} & 47 \\
\multicolumn{3}{@{}>{\raggedright\arraybackslash}p{(\columnwidth - 30\tabcolsep) * \real{0.1875} + 4\tabcolsep}}{%
4.3.4} &
\multicolumn{12}{>{\raggedright\arraybackslash}p{(\columnwidth - 30\tabcolsep) * \real{0.7500} + 22\tabcolsep}}{%
Interface gestion des organisations . . . . . . . . . . . . . . . . . .
. .} & 48 \\
\multicolumn{3}{@{}>{\raggedright\arraybackslash}p{(\columnwidth - 30\tabcolsep) * \real{0.1875} + 4\tabcolsep}}{%
4.3.5} &
\multicolumn{6}{>{\raggedright\arraybackslash}p{(\columnwidth - 30\tabcolsep) * \real{0.3750} + 10\tabcolsep}}{%
Interface d'organisation} &
\multicolumn{6}{>{\raggedright\arraybackslash}p{(\columnwidth - 30\tabcolsep) * \real{0.3750} + 10\tabcolsep}}{%
. . . . . . . . . . . . . . . . . . . . . . . . .} & 49 \\
\multicolumn{3}{@{}>{\raggedright\arraybackslash}p{(\columnwidth - 30\tabcolsep) * \real{0.1875} + 4\tabcolsep}}{%
4.3.6} &
\multicolumn{10}{>{\raggedright\arraybackslash}p{(\columnwidth - 30\tabcolsep) * \real{0.6250} + 18\tabcolsep}}{%
Interface de gestion des utilisateurs} &
\multicolumn{2}{>{\raggedright\arraybackslash}p{(\columnwidth - 30\tabcolsep) * \real{0.1250} + 2\tabcolsep}}{%
. . . . . . . . . . . . . . . . . . .} & 50 \\
\multicolumn{3}{@{}>{\raggedright\arraybackslash}p{(\columnwidth - 30\tabcolsep) * \real{0.1875} + 4\tabcolsep}}{%
4.3.7} &
\multicolumn{12}{>{\raggedright\arraybackslash}p{(\columnwidth - 30\tabcolsep) * \real{0.7500} + 22\tabcolsep}}{%
Interface de gestion des modules . . . . . . . . . . . . . . . . . . . .
.} & 50 \\
\multicolumn{3}{@{}>{\raggedright\arraybackslash}p{(\columnwidth - 30\tabcolsep) * \real{0.1875} + 4\tabcolsep}}{%
4.3.8} &
\multicolumn{11}{>{\raggedright\arraybackslash}p{(\columnwidth - 30\tabcolsep) * \real{0.6875} + 20\tabcolsep}}{%
Intégration de Vélociraptor sur DFIR IRIS} & . . . . . . . . . . . . . .
. & 51 \\
\multicolumn{3}{@{}>{\raggedright\arraybackslash}p{(\columnwidth - 30\tabcolsep) * \real{0.1875} + 4\tabcolsep}}{%
4.3.9} &
\multicolumn{7}{>{\raggedright\arraybackslash}p{(\columnwidth - 30\tabcolsep) * \real{0.4375} + 12\tabcolsep}}{%
Liste des fichiers analysés} &
\multicolumn{5}{>{\raggedright\arraybackslash}p{(\columnwidth - 30\tabcolsep) * \real{0.3125} + 8\tabcolsep}}{%
. . . . . . . . . . . . . . . . . . . . . . . .} & 51 \\
\multicolumn{15}{@{}>{\raggedright\arraybackslash}p{(\columnwidth - 30\tabcolsep) * \real{0.9375} + 28\tabcolsep}}{%
\begin{minipage}[t]{\linewidth}\raggedright
\begin{quote}
4.3.10 Aperçu du fichier analysé . . . . . . . . . . . . . . . . . . . .
. . . . .
\end{quote}
\end{minipage}} & 52 \\
\multicolumn{15}{@{}>{\raggedright\arraybackslash}p{(\columnwidth - 30\tabcolsep) * \real{0.9375} + 28\tabcolsep}}{%
\begin{minipage}[t]{\linewidth}\raggedright
\begin{quote}
4.3.11 Interface du Honeypot . . . . . . . . . . . . . . . . . . . . . .
. . . .
\end{quote}
\end{minipage}} & 53 \\
\multicolumn{15}{@{}>{\raggedright\arraybackslash}p{(\columnwidth - 30\tabcolsep) * \real{0.9375} + 28\tabcolsep}}{%
\begin{minipage}[t]{\linewidth}\raggedright
\begin{quote}
4.3.12 Interface des flux de travail . . . . . . . . . . . . . . . . . .
. . . . . .
\end{quote}
\end{minipage}} & 54 \\
\multicolumn{15}{@{}>{\raggedright\arraybackslash}p{(\columnwidth - 30\tabcolsep) * \real{0.9375} + 28\tabcolsep}}{%
\begin{minipage}[t]{\linewidth}\raggedright
\begin{quote}
4.3.13 Scénarios de flux de travail réels . . . . . . . . . . . . . . .
. . . . . .
\end{quote}
\end{minipage}} & 55 \\
\multicolumn{6}{@{}>{\raggedright\arraybackslash}p{(\columnwidth - 30\tabcolsep) * \real{0.3750} + 10\tabcolsep}}{%
4.3.13.1} &
\multicolumn{9}{>{\raggedright\arraybackslash}p{(\columnwidth - 30\tabcolsep) * \real{0.5625} + 16\tabcolsep}}{%
Flux de travail numéro 1 . . . . . . . . . . . . . . . . . . . .} &
55 \\
\multicolumn{6}{@{}>{\raggedright\arraybackslash}p{(\columnwidth - 30\tabcolsep) * \real{0.3750} + 10\tabcolsep}}{%
4.3.13.2} &
\multicolumn{9}{>{\raggedright\arraybackslash}p{(\columnwidth - 30\tabcolsep) * \real{0.5625} + 16\tabcolsep}}{%
Flux de travail numéro 2 . . . . . . . . . . . . . . . . . . . .} &
56 \\
\multicolumn{2}{@{}>{\raggedright\arraybackslash}p{(\columnwidth - 30\tabcolsep) * \real{0.1250} + 2\tabcolsep}}{%
4.4} &
\multicolumn{13}{>{\raggedright\arraybackslash}p{(\columnwidth - 30\tabcolsep) * \real{0.8125} + 24\tabcolsep}}{%
CONCLUSION . . . . . . . . . . . . . . . . . . . . . . . . . . . . . . .
. . .} & 57 \\
\multicolumn{15}{@{}>{\raggedright\arraybackslash}p{(\columnwidth - 30\tabcolsep) * \real{0.9375} + 28\tabcolsep}}{%
\begin{minipage}[t]{\linewidth}\raggedright
\begin{quote}
\textbf{CONCLUSION GÉNÉRALE}
\end{quote}
\end{minipage}} & \textbf{58} \\
\multicolumn{15}{@{}>{\raggedright\arraybackslash}p{(\columnwidth - 30\tabcolsep) * \real{0.9375} + 28\tabcolsep}}{%
\begin{minipage}[t]{\linewidth}\raggedright
\begin{quote}
\textbf{BIBLIOGRAPHIE}
\end{quote}
\end{minipage}} & \textbf{58} \\
\multicolumn{15}{@{}>{\raggedright\arraybackslash}p{(\columnwidth - 30\tabcolsep) * \real{0.9375} + 28\tabcolsep}}{%
\begin{minipage}[t]{\linewidth}\raggedright
\begin{quote}
\textbf{ANNEXES}
\end{quote}
\end{minipage}} & \textbf{60} \\
\multicolumn{3}{@{}>{\raggedright\arraybackslash}p{(\columnwidth - 30\tabcolsep) * \real{0.1875} + 4\tabcolsep}}{%
A.0.1} & Misp &
\multicolumn{11}{>{\raggedright\arraybackslash}p{(\columnwidth - 30\tabcolsep) * \real{0.6875} + 20\tabcolsep}}{%
. . . . . . . . . . . . . . . . . . . . . . . . . . . . . . . . . . .} &
60 \\
\multicolumn{5}{@{}>{\raggedright\arraybackslash}p{(\columnwidth - 30\tabcolsep) * \real{0.3125} + 8\tabcolsep}}{%
A.0.1.1} &
\multicolumn{10}{>{\raggedright\arraybackslash}p{(\columnwidth - 30\tabcolsep) * \real{0.6250} + 18\tabcolsep}}{%
Installation . . . . . . . . . . . . . . . . . . . . . . . . . . .} &
60 \\
\multicolumn{12}{@{}>{\raggedright\arraybackslash}p{(\columnwidth - 30\tabcolsep) * \real{0.7500} + 22\tabcolsep}}{%
\begin{minipage}[t]{\linewidth}\raggedright
\begin{quote}
Dhia Abdelli
\end{quote}
\end{minipage}} &
\multicolumn{4}{>{\raggedright\arraybackslash}p{(\columnwidth - 30\tabcolsep) * \real{0.2500} + 6\tabcolsep}@{}}{%
Page viii} \\
\bottomrule()
\end{longtable}

\begin{quote}
TABLE DES MATIÈRES
\end{quote}

\begin{longtable}[]{@{}
  >{\raggedright\arraybackslash}p{(\columnwidth - 12\tabcolsep) * \real{0.1429}}
  >{\raggedright\arraybackslash}p{(\columnwidth - 12\tabcolsep) * \real{0.1429}}
  >{\raggedright\arraybackslash}p{(\columnwidth - 12\tabcolsep) * \real{0.1429}}
  >{\raggedright\arraybackslash}p{(\columnwidth - 12\tabcolsep) * \real{0.1429}}
  >{\raggedright\arraybackslash}p{(\columnwidth - 12\tabcolsep) * \real{0.1429}}
  >{\raggedright\arraybackslash}p{(\columnwidth - 12\tabcolsep) * \real{0.1429}}
  >{\raggedright\arraybackslash}p{(\columnwidth - 12\tabcolsep) * \real{0.1429}}@{}}
\toprule()
\multirow{10}{*}{\begin{minipage}[b]{\linewidth}\raggedright
A.0.2

A.0.3

A.0.4
\end{minipage}} &
\multicolumn{2}{>{\raggedright\arraybackslash}p{(\columnwidth - 12\tabcolsep) * \real{0.2857} + 2\tabcolsep}}{%
\begin{minipage}[b]{\linewidth}\raggedright
A.0.1.2
\end{minipage}} &
\multicolumn{3}{>{\raggedright\arraybackslash}p{(\columnwidth - 12\tabcolsep) * \real{0.4286} + 4\tabcolsep}}{%
\begin{minipage}[b]{\linewidth}\raggedright
Importer des flux de données . . . . . . . . . . . . . . . . .
\end{minipage}} & \begin{minipage}[b]{\linewidth}\raggedright
61
\end{minipage} \\
&
\multicolumn{2}{>{\raggedright\arraybackslash}p{(\columnwidth - 12\tabcolsep) * \real{0.2857} + 2\tabcolsep}}{%
\begin{minipage}[b]{\linewidth}\raggedright
A.0.1.3
\end{minipage}} &
\multicolumn{3}{>{\raggedright\arraybackslash}p{(\columnwidth - 12\tabcolsep) * \real{0.4286} + 4\tabcolsep}}{%
\begin{minipage}[b]{\linewidth}\raggedright
Activation des flux . . . . . . . . . . . . . . . . . . . . . . .
\end{minipage}} & \begin{minipage}[b]{\linewidth}\raggedright
61
\end{minipage} \\
& \begin{minipage}[b]{\linewidth}\raggedright
\begin{quote}
Shuffle
\end{quote}
\end{minipage} &
\multicolumn{4}{>{\raggedright\arraybackslash}p{(\columnwidth - 12\tabcolsep) * \real{0.5714} + 6\tabcolsep}}{%
\begin{minipage}[b]{\linewidth}\raggedright
. . . . . . . . . . . . . . . . . . . . . . . . . . . . . . . . . .
\end{minipage}} & \begin{minipage}[b]{\linewidth}\raggedright
62
\end{minipage} \\
&
\multicolumn{2}{>{\raggedright\arraybackslash}p{(\columnwidth - 12\tabcolsep) * \real{0.2857} + 2\tabcolsep}}{%
\begin{minipage}[b]{\linewidth}\raggedright
A.0.2.1
\end{minipage}} &
\multicolumn{3}{>{\raggedright\arraybackslash}p{(\columnwidth - 12\tabcolsep) * \real{0.4286} + 4\tabcolsep}}{%
\begin{minipage}[b]{\linewidth}\raggedright
Installation . . . . . . . . . . . . . . . . . . . . . . . . . . .
\end{minipage}} & \begin{minipage}[b]{\linewidth}\raggedright
62
\end{minipage} \\
&
\multicolumn{5}{>{\raggedright\arraybackslash}p{(\columnwidth - 12\tabcolsep) * \real{0.7143} + 8\tabcolsep}}{%
\begin{minipage}[b]{\linewidth}\raggedright
DFIR Iris . . . . . . . . . . . . . . . . . . . . . . . . . . . . . . .
. .
\end{minipage}} & \begin{minipage}[b]{\linewidth}\raggedright
63
\end{minipage} \\
&
\multicolumn{2}{>{\raggedright\arraybackslash}p{(\columnwidth - 12\tabcolsep) * \real{0.2857} + 2\tabcolsep}}{%
\begin{minipage}[b]{\linewidth}\raggedright
A.0.3.1
\end{minipage}} &
\multicolumn{3}{>{\raggedright\arraybackslash}p{(\columnwidth - 12\tabcolsep) * \real{0.4286} + 4\tabcolsep}}{%
\begin{minipage}[b]{\linewidth}\raggedright
Installation . . . . . . . . . . . . . . . . . . . . . . . . . . .
\end{minipage}} & \begin{minipage}[b]{\linewidth}\raggedright
63
\end{minipage} \\
&
\multicolumn{3}{>{\raggedright\arraybackslash}p{(\columnwidth - 12\tabcolsep) * \real{0.4286} + 4\tabcolsep}}{%
\begin{minipage}[b]{\linewidth}\raggedright
ELK Stack
\end{minipage}} &
\multicolumn{2}{>{\raggedright\arraybackslash}p{(\columnwidth - 12\tabcolsep) * \real{0.2857} + 2\tabcolsep}}{%
\begin{minipage}[b]{\linewidth}\raggedright
. . . . . . . . . . . . . . . . . . . . . . . . . . . . . . . .
\end{minipage}} & \begin{minipage}[b]{\linewidth}\raggedright
65
\end{minipage} \\
&
\multicolumn{2}{>{\raggedright\arraybackslash}p{(\columnwidth - 12\tabcolsep) * \real{0.2857} + 2\tabcolsep}}{%
\begin{minipage}[b]{\linewidth}\raggedright
A.0.4.1
\end{minipage}} &
\multicolumn{2}{>{\raggedright\arraybackslash}p{(\columnwidth - 12\tabcolsep) * \real{0.2857} + 2\tabcolsep}}{%
\begin{minipage}[b]{\linewidth}\raggedright
Installation ElasicSearch
\end{minipage}} & \begin{minipage}[b]{\linewidth}\raggedright
. . . . . . . . . . . . . . . . . . .
\end{minipage} & \begin{minipage}[b]{\linewidth}\raggedright
65
\end{minipage} \\
&
\multicolumn{2}{>{\raggedright\arraybackslash}p{(\columnwidth - 12\tabcolsep) * \real{0.2857} + 2\tabcolsep}}{%
\begin{minipage}[b]{\linewidth}\raggedright
A.0.4.2
\end{minipage}} &
\multicolumn{3}{>{\raggedright\arraybackslash}p{(\columnwidth - 12\tabcolsep) * \real{0.4286} + 4\tabcolsep}}{%
\begin{minipage}[b]{\linewidth}\raggedright
Installation Kibana . . . . . . . . . . . . . . . . . . . . . . .
\end{minipage}} & \begin{minipage}[b]{\linewidth}\raggedright
66
\end{minipage} \\
&
\multicolumn{2}{>{\raggedright\arraybackslash}p{(\columnwidth - 12\tabcolsep) * \real{0.2857} + 2\tabcolsep}}{%
\begin{minipage}[b]{\linewidth}\raggedright
A.0.4.3
\end{minipage}} &
\multicolumn{3}{>{\raggedright\arraybackslash}p{(\columnwidth - 12\tabcolsep) * \real{0.4286} + 4\tabcolsep}}{%
\begin{minipage}[b]{\linewidth}\raggedright
Configuration Logstash . . . . . . . . . . . . . . . . . . . .
\end{minipage}} & \begin{minipage}[b]{\linewidth}\raggedright
67
\end{minipage} \\
\midrule()
\endhead
\begin{minipage}[t]{\linewidth}\raggedright
\begin{quote}
Dhia Abdelli
\end{quote}
\end{minipage} &
\multicolumn{6}{>{\raggedright\arraybackslash}p{(\columnwidth - 12\tabcolsep) * \real{0.8571} + 10\tabcolsep}@{}}{%
Page ix} \\
\bottomrule()
\end{longtable}

\begin{quote}
\textbf{LISTE DES FIGURES}
\end{quote}

\begin{longtable}[]{@{}
  >{\raggedright\arraybackslash}p{(\columnwidth - 4\tabcolsep) * \real{0.3333}}
  >{\raggedright\arraybackslash}p{(\columnwidth - 4\tabcolsep) * \real{0.3333}}
  >{\raggedright\arraybackslash}p{(\columnwidth - 4\tabcolsep) * \real{0.3333}}@{}}
\toprule()
\begin{minipage}[b]{\linewidth}\raggedright
1.1
\end{minipage} & \begin{minipage}[b]{\linewidth}\raggedright
Advancia IT . . . . . . . . . . . . . . . . . . . . . . . . . . . . . .
. . . . . .
\end{minipage} & \begin{minipage}[b]{\linewidth}\raggedright
3
\end{minipage} \\
\midrule()
\endhead
\bottomrule()
\end{longtable}

\begin{longtable}[]{@{}
  >{\raggedright\arraybackslash}p{(\columnwidth - 16\tabcolsep) * \real{0.1111}}
  >{\raggedright\arraybackslash}p{(\columnwidth - 16\tabcolsep) * \real{0.1111}}
  >{\raggedright\arraybackslash}p{(\columnwidth - 16\tabcolsep) * \real{0.1111}}
  >{\raggedright\arraybackslash}p{(\columnwidth - 16\tabcolsep) * \real{0.1111}}
  >{\raggedright\arraybackslash}p{(\columnwidth - 16\tabcolsep) * \real{0.1111}}
  >{\raggedright\arraybackslash}p{(\columnwidth - 16\tabcolsep) * \real{0.1111}}
  >{\raggedright\arraybackslash}p{(\columnwidth - 16\tabcolsep) * \real{0.1111}}
  >{\raggedright\arraybackslash}p{(\columnwidth - 16\tabcolsep) * \real{0.1111}}
  >{\raggedright\arraybackslash}p{(\columnwidth - 16\tabcolsep) * \real{0.1111}}@{}}
\toprule()
\begin{minipage}[b]{\linewidth}\raggedright
1.2
\end{minipage} &
\multicolumn{7}{>{\raggedright\arraybackslash}p{(\columnwidth - 16\tabcolsep) * \real{0.7778} + 12\tabcolsep}}{%
\begin{minipage}[b]{\linewidth}\raggedright
Schéma organisationnel d'advancia . . . . . . . . . . . . . . . . . . .
. . . . .
\end{minipage}} & \begin{minipage}[b]{\linewidth}\raggedright
4
\end{minipage} \\
\midrule()
\endhead
1.3 &
\multicolumn{7}{>{\raggedright\arraybackslash}p{(\columnwidth - 16\tabcolsep) * \real{0.7778} + 12\tabcolsep}}{%
Services Proximity d'Avancia IT . . . . . . . . . . . . . . . . . . . .
. . . . .} & 5 \\
1.4 &
\multicolumn{7}{>{\raggedright\arraybackslash}p{(\columnwidth - 16\tabcolsep) * \real{0.7778} + 12\tabcolsep}}{%
Clients Advancia IT . . . . . . . . . . . . . . . . . . . . . . . . . .
. . . . . .} & 5 \\
1.5 &
\multicolumn{5}{>{\raggedright\arraybackslash}p{(\columnwidth - 16\tabcolsep) * \real{0.5556} + 8\tabcolsep}}{%
Partenaires d'Advancia IT} &
\multicolumn{2}{>{\raggedright\arraybackslash}p{(\columnwidth - 16\tabcolsep) * \real{0.2222} + 2\tabcolsep}}{%
. . . . . . . . . . . . . . . . . . . . . . . . . . . .} & 6 \\
1.6 &
\multicolumn{7}{>{\raggedright\arraybackslash}p{(\columnwidth - 16\tabcolsep) * \real{0.7778} + 12\tabcolsep}}{%
Trello . . . . . . . . . . . . . . . . . . . . . . . . . . . . . . . . .
. . . . . .} & 10 \\
1.7 &
\multicolumn{3}{>{\raggedright\arraybackslash}p{(\columnwidth - 16\tabcolsep) * \real{0.3333} + 4\tabcolsep}}{%
Diagramme de Gantt} &
\multicolumn{4}{>{\raggedright\arraybackslash}p{(\columnwidth - 16\tabcolsep) * \real{0.4444} + 6\tabcolsep}}{%
. . . . . . . . . . . . . . . . . . . . . . . . . . . . . . .} & 11 \\
2.1 & Triade CIA &
\multicolumn{6}{>{\raggedright\arraybackslash}p{(\columnwidth - 16\tabcolsep) * \real{0.6667} + 10\tabcolsep}}{%
. . . . . . . . . . . . . . . . . . . . . . . . . . . . . . . . . . . .}
& 14 \\
2.2 &
\multicolumn{7}{>{\raggedright\arraybackslash}p{(\columnwidth - 16\tabcolsep) * \real{0.7778} + 12\tabcolsep}}{%
Logo de Elastic Stack . . . . . . . . . . . . . . . . . . . . . . . . .
. . . . . .} & 24 \\
2.3 &
\multicolumn{7}{>{\raggedright\arraybackslash}p{(\columnwidth - 16\tabcolsep) * \real{0.7778} + 12\tabcolsep}}{%
Logo de Misp . . . . . . . . . . . . . . . . . . . . . . . . . . . . . .
. . . . .} & 25 \\
2.4 &
\multicolumn{7}{>{\raggedright\arraybackslash}p{(\columnwidth - 16\tabcolsep) * \real{0.7778} + 12\tabcolsep}}{%
Logo de DFIR IRIS . . . . . . . . . . . . . . . . . . . . . . . . . . .
. . . . .} & 25 \\
2.5 &
\multicolumn{7}{>{\raggedright\arraybackslash}p{(\columnwidth - 16\tabcolsep) * \real{0.7778} + 12\tabcolsep}}{%
Logo de Shuffle . . . . . . . . . . . . . . . . . . . . . . . . . . . .
. . . . . .} & 26 \\
2.6 &
\multicolumn{7}{>{\raggedright\arraybackslash}p{(\columnwidth - 16\tabcolsep) * \real{0.7778} + 12\tabcolsep}}{%
Logo de T-pot . . . . . . . . . . . . . . . . . . . . . . . . . . . . .
. . . . . .} & 26 \\
2.7 &
\multicolumn{3}{>{\raggedright\arraybackslash}p{(\columnwidth - 16\tabcolsep) * \real{0.3333} + 4\tabcolsep}}{%
Logo de GreedyBear} &
\multicolumn{4}{>{\raggedright\arraybackslash}p{(\columnwidth - 16\tabcolsep) * \real{0.4444} + 6\tabcolsep}}{%
. . . . . . . . . . . . . . . . . . . . . . . . . . . . . . .} & 27 \\
2.8 &
\multicolumn{7}{>{\raggedright\arraybackslash}p{(\columnwidth - 16\tabcolsep) * \real{0.7778} + 12\tabcolsep}}{%
Logo Cuckoo Sanbox . . . . . . . . . . . . . . . . . . . . . . . . . . .
. . . .} & 27 \\
2.9 &
\multicolumn{7}{>{\raggedright\arraybackslash}p{(\columnwidth - 16\tabcolsep) * \real{0.7778} + 12\tabcolsep}}{%
Logo Praeco . . . . . . . . . . . . . . . . . . . . . . . . . . . . . .
. . . . . .} & 28 \\
\multicolumn{8}{@{}>{\raggedright\arraybackslash}p{(\columnwidth - 16\tabcolsep) * \real{0.8889} + 14\tabcolsep}}{%
\begin{minipage}[t]{\linewidth}\raggedright
\begin{quote}
2.10 Logo velociraptor . . . . . . . . . . . . . . . . . . . . . . . . .
. . . . . . . .
\end{quote}
\end{minipage}} & 28 \\
4.1 &
\multicolumn{6}{>{\raggedright\arraybackslash}p{(\columnwidth - 16\tabcolsep) * \real{0.6667} + 10\tabcolsep}}{%
Liste des cas sur DFIR IRIS} & . . . . . . . . . . . . . . . . . . . . .
. . . . . . & 46 \\
4.2 &
\multicolumn{7}{>{\raggedright\arraybackslash}p{(\columnwidth - 16\tabcolsep) * \real{0.7778} + 12\tabcolsep}}{%
Liste des alertes sur DFIR IRIS . . . . . . . . . . . . . . . . . . . .
. . . . . .} & 47 \\
4.3 &
\multicolumn{7}{>{\raggedright\arraybackslash}p{(\columnwidth - 16\tabcolsep) * \real{0.7778} + 12\tabcolsep}}{%
Aperçu de cas sur DFIR IRIS . . . . . . . . . . . . . . . . . . . . . .
. . . . .} & 48 \\
4.4 &
\multicolumn{7}{>{\raggedright\arraybackslash}p{(\columnwidth - 16\tabcolsep) * \real{0.7778} + 12\tabcolsep}}{%
Liste des IoCs d'un cas . . . . . . . . . . . . . . . . . . . . . . . .
. . . . . .} & 48 \\
4.5 &
\multicolumn{7}{>{\raggedright\arraybackslash}p{(\columnwidth - 16\tabcolsep) * \real{0.7778} + 12\tabcolsep}}{%
Liste des tâches d'un cas . . . . . . . . . . . . . . . . . . . . . . .
. . . . . .} & 48 \\
4.6 &
\multicolumn{7}{>{\raggedright\arraybackslash}p{(\columnwidth - 16\tabcolsep) * \real{0.7778} + 12\tabcolsep}}{%
Liste des organisations sur DFIR Iris . . . . . . . . . . . . . . . . .
. . . . . .} & 49 \\
4.7 &
\multicolumn{5}{>{\raggedright\arraybackslash}p{(\columnwidth - 16\tabcolsep) * \real{0.5556} + 8\tabcolsep}}{%
Interface de l'organisation} &
\multicolumn{2}{>{\raggedright\arraybackslash}p{(\columnwidth - 16\tabcolsep) * \real{0.2222} + 2\tabcolsep}}{%
. . . . . . . . . . . . . . . . . . . . . . . . . . . .} & 49 \\
4.8 &
\multicolumn{2}{>{\raggedright\arraybackslash}p{(\columnwidth - 16\tabcolsep) * \real{0.2222} + 2\tabcolsep}}{%
\begin{minipage}[t]{\linewidth}\raggedright
\begin{quote}
Liste des utilisateurs
\end{quote}
\end{minipage}} &
\multicolumn{5}{>{\raggedright\arraybackslash}p{(\columnwidth - 16\tabcolsep) * \real{0.5556} + 8\tabcolsep}}{%
\begin{minipage}[t]{\linewidth}\raggedright
\begin{quote}
. . . . . . . . . . . . . . . . . . . . . . . . . . . . . . .
\end{quote}
\end{minipage}} & 50 \\
4.9 &
\multicolumn{7}{>{\raggedright\arraybackslash}p{(\columnwidth - 16\tabcolsep) * \real{0.7778} + 12\tabcolsep}}{%
Liste des modules . . . . . . . . . . . . . . . . . . . . . . . . . . .
. . . . . .} & 50 \\
\multicolumn{5}{@{}>{\raggedright\arraybackslash}p{(\columnwidth - 16\tabcolsep) * \real{0.5556} + 8\tabcolsep}}{%
\begin{minipage}[t]{\linewidth}\raggedright
\begin{quote}
4.10 Intégration Vélociraptor
\end{quote}
\end{minipage}} &
\multicolumn{3}{>{\raggedright\arraybackslash}p{(\columnwidth - 16\tabcolsep) * \real{0.3333} + 4\tabcolsep}}{%
. . . . . . . . . . . . . . . . . . . . . . . . . . . . .} & 51 \\
\multicolumn{3}{@{}>{\raggedright\arraybackslash}p{(\columnwidth - 16\tabcolsep) * \real{0.3333} + 4\tabcolsep}}{%
\begin{minipage}[t]{\linewidth}\raggedright
\begin{quote}
Dhia Abdelli
\end{quote}
\end{minipage}} &
\multicolumn{6}{>{\raggedright\arraybackslash}p{(\columnwidth - 16\tabcolsep) * \real{0.6667} + 10\tabcolsep}@{}}{%
Page x} \\
\bottomrule()
\end{longtable}

\begin{quote}
LISTE DES FIGURES
\end{quote}

\begin{longtable}[]{@{}
  >{\raggedright\arraybackslash}p{(\columnwidth - 16\tabcolsep) * \real{0.1111}}
  >{\raggedright\arraybackslash}p{(\columnwidth - 16\tabcolsep) * \real{0.1111}}
  >{\raggedright\arraybackslash}p{(\columnwidth - 16\tabcolsep) * \real{0.1111}}
  >{\raggedright\arraybackslash}p{(\columnwidth - 16\tabcolsep) * \real{0.1111}}
  >{\raggedright\arraybackslash}p{(\columnwidth - 16\tabcolsep) * \real{0.1111}}
  >{\raggedright\arraybackslash}p{(\columnwidth - 16\tabcolsep) * \real{0.1111}}
  >{\raggedright\arraybackslash}p{(\columnwidth - 16\tabcolsep) * \real{0.1111}}
  >{\raggedright\arraybackslash}p{(\columnwidth - 16\tabcolsep) * \real{0.1111}}
  >{\raggedright\arraybackslash}p{(\columnwidth - 16\tabcolsep) * \real{0.1111}}@{}}
\toprule()
\multicolumn{8}{@{}>{\raggedright\arraybackslash}p{(\columnwidth - 16\tabcolsep) * \real{0.8889} + 14\tabcolsep}}{%
\begin{minipage}[b]{\linewidth}\raggedright
\begin{quote}
4.11 Liste des fichiers analysés sur Cuckoo Sanbox . . . . . . . . . . .
. . . . . . .
\end{quote}
\end{minipage}} & \begin{minipage}[b]{\linewidth}\raggedright
51
\end{minipage} \\
\midrule()
\endhead
\multicolumn{3}{@{}>{\raggedright\arraybackslash}p{(\columnwidth - 16\tabcolsep) * \real{0.3333} + 4\tabcolsep}}{%
\begin{minipage}[t]{\linewidth}\raggedright
\begin{quote}
4.12 Résumé d'un fichier analysé
\end{quote}
\end{minipage}} &
\multicolumn{5}{>{\raggedright\arraybackslash}p{(\columnwidth - 16\tabcolsep) * \real{0.5556} + 8\tabcolsep}}{%
\begin{minipage}[t]{\linewidth}\raggedright
\begin{quote}
. . . . . . . . . . . . . . . . . . . . . . . . . . .
\end{quote}
\end{minipage}} & 52 \\
\multicolumn{5}{@{}>{\raggedright\arraybackslash}p{(\columnwidth - 16\tabcolsep) * \real{0.5556} + 8\tabcolsep}}{%
\begin{minipage}[t]{\linewidth}\raggedright
\begin{quote}
4.13 Les Captures d'écran d'un fichier analysé
\end{quote}
\end{minipage}} &
\multicolumn{3}{>{\raggedright\arraybackslash}p{(\columnwidth - 16\tabcolsep) * \real{0.3333} + 4\tabcolsep}}{%
. . . . . . . . . . . . . . . . . . . .} & 52 \\
\multicolumn{8}{@{}>{\raggedright\arraybackslash}p{(\columnwidth - 16\tabcolsep) * \real{0.8889} + 14\tabcolsep}}{%
\begin{minipage}[t]{\linewidth}\raggedright
\begin{quote}
4.14 Liste des flux de travail . . . . . . . . . . . . . . . . . . . . .
. . . . . . . . .
\end{quote}
\end{minipage}} & 53 \\
\multicolumn{8}{@{}>{\raggedright\arraybackslash}p{(\columnwidth - 16\tabcolsep) * \real{0.8889} + 14\tabcolsep}}{%
\begin{minipage}[t]{\linewidth}\raggedright
\begin{quote}
4.15 Liste des mots de passe et noms d'utilisateur collectés . . . . . .
. . . . . . . .
\end{quote}
\end{minipage}} & 53 \\
\multicolumn{8}{@{}>{\raggedright\arraybackslash}p{(\columnwidth - 16\tabcolsep) * \real{0.8889} + 14\tabcolsep}}{%
\begin{minipage}[t]{\linewidth}\raggedright
\begin{quote}
4.16 Liste des mots de passe et noms d'utilisateur collectés . . . . . .
. . . . . . . .
\end{quote}
\end{minipage}} & 54 \\
\multicolumn{8}{@{}>{\raggedright\arraybackslash}p{(\columnwidth - 16\tabcolsep) * \real{0.8889} + 14\tabcolsep}}{%
\begin{minipage}[t]{\linewidth}\raggedright
\begin{quote}
4.17 Liste des flux de travail . . . . . . . . . . . . . . . . . . . . .
. . . . . . . . .
\end{quote}
\end{minipage}} & 54 \\
\multicolumn{8}{@{}>{\raggedright\arraybackslash}p{(\columnwidth - 16\tabcolsep) * \real{0.8889} + 14\tabcolsep}}{%
\begin{minipage}[t]{\linewidth}\raggedright
\begin{quote}
4.18 Scénario de connexion à l'aide d'un port inhabituel . . . . . . . .
. . . . . . .
\end{quote}
\end{minipage}} & 55 \\
\multicolumn{8}{@{}>{\raggedright\arraybackslash}p{(\columnwidth - 16\tabcolsep) * \real{0.8889} + 14\tabcolsep}}{%
\begin{minipage}[t]{\linewidth}\raggedright
\begin{quote}
4.19 Cas créé sur DFIR IRIS . . . . . . . . . . . . . . . . . . . . . .
. . . . . . . .
\end{quote}
\end{minipage}} & 56 \\
\multicolumn{8}{@{}>{\raggedright\arraybackslash}p{(\columnwidth - 16\tabcolsep) * \real{0.8889} + 14\tabcolsep}}{%
\begin{minipage}[t]{\linewidth}\raggedright
\begin{quote}
4.20 L'équipement a été ajouté à la liste des actifs . . . . . . . . . .
. . . . . . . . .
\end{quote}
\end{minipage}} & 56 \\
\multicolumn{3}{@{}>{\raggedright\arraybackslash}p{(\columnwidth - 16\tabcolsep) * \real{0.3333} + 4\tabcolsep}}{%
\begin{minipage}[t]{\linewidth}\raggedright
\begin{quote}
4.21 Les IoCs a été ajouté à la liste
\end{quote}
\end{minipage}} &
\multicolumn{5}{>{\raggedright\arraybackslash}p{(\columnwidth - 16\tabcolsep) * \real{0.5556} + 8\tabcolsep}}{%
\begin{minipage}[t]{\linewidth}\raggedright
\begin{quote}
. . . . . . . . . . . . . . . . . . . . . . . . . .
\end{quote}
\end{minipage}} & 56 \\
\multicolumn{8}{@{}>{\raggedright\arraybackslash}p{(\columnwidth - 16\tabcolsep) * \real{0.8889} + 14\tabcolsep}}{%
\begin{minipage}[t]{\linewidth}\raggedright
\begin{quote}
4.22 Scénario de connexion à un honeypot local . . . . . . . . . . . . .
. . . . . . .
\end{quote}
\end{minipage}} & 57 \\
A.1 &
\multicolumn{7}{>{\raggedright\arraybackslash}p{(\columnwidth - 16\tabcolsep) * \real{0.7778} + 12\tabcolsep}}{%
Téléchargement du script d'installation . . . . . . . . . . . . . . . .
. . . . . .} & 60 \\
A.2 &
\multicolumn{7}{>{\raggedright\arraybackslash}p{(\columnwidth - 16\tabcolsep) * \real{0.7778} + 12\tabcolsep}}{%
Installation MISP . . . . . . . . . . . . . . . . . . . . . . . . . . .
. . . . . .} & 60 \\
A.3 &
\multicolumn{7}{>{\raggedright\arraybackslash}p{(\columnwidth - 16\tabcolsep) * \real{0.7778} + 12\tabcolsep}}{%
Page d'Authentification de MISP . . . . . . . . . . . . . . . . . . . .
. . . . .} & 61 \\
A.4 &
\multicolumn{7}{>{\raggedright\arraybackslash}p{(\columnwidth - 16\tabcolsep) * \real{0.7778} + 12\tabcolsep}}{%
MISP flux de données . . . . . . . . . . . . . . . . . . . . . . . . . .
. . . . .} & 61 \\
A.5 &
\multicolumn{7}{>{\raggedright\arraybackslash}p{(\columnwidth - 16\tabcolsep) * \real{0.7778} + 12\tabcolsep}}{%
MISP Activation des flux . . . . . . . . . . . . . . . . . . . . . . . .
. . . . .} & 62 \\
A.6 &
\multicolumn{7}{>{\raggedright\arraybackslash}p{(\columnwidth - 16\tabcolsep) * \real{0.7778} + 12\tabcolsep}}{%
Clonage du Projet Shuffle depuis GitHub . . . . . . . . . . . . . . . .
. . . . .} & 62 \\
A.7 &
\multicolumn{7}{>{\raggedright\arraybackslash}p{(\columnwidth - 16\tabcolsep) * \real{0.7778} + 12\tabcolsep}}{%
Prérequis pour la Base de Données OpenSearch . . . . . . . . . . . . . .
. . .} & 62 \\
A.8 & Déploiement de Shuffle &
\multicolumn{6}{>{\raggedright\arraybackslash}p{(\columnwidth - 16\tabcolsep) * \real{0.6667} + 10\tabcolsep}}{%
. . . . . . . . . . . . . . . . . . . . . . . . . . . . .} & 63 \\
A.9 &
\multicolumn{7}{>{\raggedright\arraybackslash}p{(\columnwidth - 16\tabcolsep) * \real{0.7778} + 12\tabcolsep}}{%
Shuffle création de compte . . . . . . . . . . . . . . . . . . . . . . .
. . . . .} & 63 \\
\multicolumn{7}{@{}>{\raggedright\arraybackslash}p{(\columnwidth - 16\tabcolsep) * \real{0.7778} + 12\tabcolsep}}{%
\begin{minipage}[t]{\linewidth}\raggedright
\begin{quote}
A.10 Clonage du Projet DFIR IRIS depuis GitHub
\end{quote}
\end{minipage}} & . . . . . . . . . . . . . . . . . . & 64 \\
\multicolumn{3}{@{}>{\raggedright\arraybackslash}p{(\columnwidth - 16\tabcolsep) * \real{0.3333} + 4\tabcolsep}}{%
\begin{minipage}[t]{\linewidth}\raggedright
\begin{quote}
A.11 Déploiement de DFIR IRIS
\end{quote}
\end{minipage}} &
\multicolumn{5}{>{\raggedright\arraybackslash}p{(\columnwidth - 16\tabcolsep) * \real{0.5556} + 8\tabcolsep}}{%
\begin{minipage}[t]{\linewidth}\raggedright
\begin{quote}
. . . . . . . . . . . . . . . . . . . . . . . . . . .
\end{quote}
\end{minipage}} & 64 \\
\multicolumn{3}{@{}>{\raggedright\arraybackslash}p{(\columnwidth - 16\tabcolsep) * \real{0.3333} + 4\tabcolsep}}{%
\begin{minipage}[t]{\linewidth}\raggedright
\begin{quote}
A.12 Déploiement de DFIR IRIS
\end{quote}
\end{minipage}} &
\multicolumn{5}{>{\raggedright\arraybackslash}p{(\columnwidth - 16\tabcolsep) * \real{0.5556} + 8\tabcolsep}}{%
\begin{minipage}[t]{\linewidth}\raggedright
\begin{quote}
. . . . . . . . . . . . . . . . . . . . . . . . . . .
\end{quote}
\end{minipage}} & 64 \\
\multicolumn{6}{@{}>{\raggedright\arraybackslash}p{(\columnwidth - 16\tabcolsep) * \real{0.6667} + 10\tabcolsep}}{%
\begin{minipage}[t]{\linewidth}\raggedright
\begin{quote}
A.13 l'installation de la clé de publique d'Elastic
\end{quote}
\end{minipage}} &
\multicolumn{2}{>{\raggedright\arraybackslash}p{(\columnwidth - 16\tabcolsep) * \real{0.2222} + 2\tabcolsep}}{%
. . . . . . . . . . . . . . . . . . .} & 65 \\
\multicolumn{8}{@{}>{\raggedright\arraybackslash}p{(\columnwidth - 16\tabcolsep) * \real{0.8889} + 14\tabcolsep}}{%
\begin{minipage}[t]{\linewidth}\raggedright
\begin{quote}
A.14 Téléchargement du package elasticSearch . . . . . . . . . . . . . .
. . . . . .
\end{quote}
\end{minipage}} & 65 \\
\multicolumn{8}{@{}>{\raggedright\arraybackslash}p{(\columnwidth - 16\tabcolsep) * \real{0.8889} + 14\tabcolsep}}{%
\begin{minipage}[t]{\linewidth}\raggedright
\begin{quote}
A.15 Installation du package elasticSearch . . . . . . . . . . . . . . .
. . . . . . . .
\end{quote}
\end{minipage}} & 65 \\
\multicolumn{8}{@{}>{\raggedright\arraybackslash}p{(\columnwidth - 16\tabcolsep) * \real{0.8889} + 14\tabcolsep}}{%
\begin{minipage}[t]{\linewidth}\raggedright
\begin{quote}
A.16 Fichier configuration elasticSearch . . . . . . . . . . . . . . . .
. . . . . . . .
\end{quote}
\end{minipage}} & 65 \\
\multicolumn{8}{@{}>{\raggedright\arraybackslash}p{(\columnwidth - 16\tabcolsep) * \real{0.8889} + 14\tabcolsep}}{%
\begin{minipage}[t]{\linewidth}\raggedright
\begin{quote}
A.17 Interface Web elasticSearch . . . . . . . . . . . . . . . . . . . .
. . . . . . . .
\end{quote}
\end{minipage}} & 66 \\
\multicolumn{4}{@{}>{\raggedright\arraybackslash}p{(\columnwidth - 16\tabcolsep) * \real{0.4444} + 6\tabcolsep}}{%
\begin{minipage}[t]{\linewidth}\raggedright
\begin{quote}
A.18 Téléchargement du package Kibana
\end{quote}
\end{minipage}} &
\multicolumn{4}{>{\raggedright\arraybackslash}p{(\columnwidth - 16\tabcolsep) * \real{0.4444} + 6\tabcolsep}}{%
. . . . . . . . . . . . . . . . . . . . . . .} & 66 \\
\multicolumn{8}{@{}>{\raggedright\arraybackslash}p{(\columnwidth - 16\tabcolsep) * \real{0.8889} + 14\tabcolsep}}{%
\begin{minipage}[t]{\linewidth}\raggedright
\begin{quote}
A.19 Installation du package Kibana . . . . . . . . . . . . . . . . . .
. . . . . . . .
\end{quote}
\end{minipage}} & 66 \\
\multicolumn{8}{@{}>{\raggedright\arraybackslash}p{(\columnwidth - 16\tabcolsep) * \real{0.8889} + 14\tabcolsep}}{%
\begin{minipage}[t]{\linewidth}\raggedright
\begin{quote}
A.20 Installation du package Kibana . . . . . . . . . . . . . . . . . .
. . . . . . . .
\end{quote}
\end{minipage}} & 67 \\
\multicolumn{8}{@{}>{\raggedright\arraybackslash}p{(\columnwidth - 16\tabcolsep) * \real{0.8889} + 14\tabcolsep}}{%
\begin{minipage}[t]{\linewidth}\raggedright
\begin{quote}
A.21 Interface Kibana . . . . . . . . . . . . . . . . . . . . . . . . .
. . . . . . . . .
\end{quote}
\end{minipage}} & 67 \\
\multicolumn{8}{@{}>{\raggedright\arraybackslash}p{(\columnwidth - 16\tabcolsep) * \real{0.8889} + 14\tabcolsep}}{%
\begin{minipage}[t]{\linewidth}\raggedright
\begin{quote}
A.22 Fichier de configuration logstash . . . . . . . . . . . . . . . . .
. . . . . . . .
\end{quote}
\end{minipage}} & 67 \\
\multicolumn{3}{@{}>{\raggedright\arraybackslash}p{(\columnwidth - 16\tabcolsep) * \real{0.3333} + 4\tabcolsep}}{%
\begin{minipage}[t]{\linewidth}\raggedright
\begin{quote}
Dhia Abdelli
\end{quote}
\end{minipage}} &
\multicolumn{6}{>{\raggedright\arraybackslash}p{(\columnwidth - 16\tabcolsep) * \real{0.6667} + 10\tabcolsep}@{}}{%
Page xi} \\
\bottomrule()
\end{longtable}

\begin{quote}
LISTE DES TABLEAUX

\textbf{Liste des tableaux}
\end{quote}

\begin{longtable}[]{@{}
  >{\raggedright\arraybackslash}p{(\columnwidth - 6\tabcolsep) * \real{0.2500}}
  >{\raggedright\arraybackslash}p{(\columnwidth - 6\tabcolsep) * \real{0.2500}}
  >{\raggedright\arraybackslash}p{(\columnwidth - 6\tabcolsep) * \real{0.2500}}
  >{\raggedright\arraybackslash}p{(\columnwidth - 6\tabcolsep) * \real{0.2500}}@{}}
\toprule()
\begin{minipage}[b]{\linewidth}\raggedright
3.1
\end{minipage} & \begin{minipage}[b]{\linewidth}\raggedright
Description textuelle du sous-cas d'utilisation « Création de cas 1 »
\end{minipage} & \begin{minipage}[b]{\linewidth}\raggedright
. . . . . .
\end{minipage} & \begin{minipage}[b]{\linewidth}\raggedright
35
\end{minipage} \\
\midrule()
\endhead
\bottomrule()
\end{longtable}

\begin{longtable}[]{@{}
  >{\raggedright\arraybackslash}p{(\columnwidth - 8\tabcolsep) * \real{0.2000}}
  >{\raggedright\arraybackslash}p{(\columnwidth - 8\tabcolsep) * \real{0.2000}}
  >{\raggedright\arraybackslash}p{(\columnwidth - 8\tabcolsep) * \real{0.2000}}
  >{\raggedright\arraybackslash}p{(\columnwidth - 8\tabcolsep) * \real{0.2000}}
  >{\raggedright\arraybackslash}p{(\columnwidth - 8\tabcolsep) * \real{0.2000}}@{}}
\toprule()
\begin{minipage}[b]{\linewidth}\raggedright
3.2
\end{minipage} &
\multicolumn{3}{>{\raggedright\arraybackslash}p{(\columnwidth - 8\tabcolsep) * \real{0.6000} + 4\tabcolsep}}{%
\begin{minipage}[b]{\linewidth}\raggedright
Description textuelle du sous-cas d'utilisation « Création de cas 2» . .
. . . . .
\end{minipage}} & \begin{minipage}[b]{\linewidth}\raggedright
36
\end{minipage} \\
\midrule()
\endhead
3.3 &
\multicolumn{3}{>{\raggedright\arraybackslash}p{(\columnwidth - 8\tabcolsep) * \real{0.6000} + 4\tabcolsep}}{%
Description textuelle du sous-cas d'utilisation « Clôture du cas » . . .
. . . . .} & 36 \\
3.4 & Description textuelle du sous-cas d'utilisation « Escalade du cas
» &
\multicolumn{2}{>{\raggedright\arraybackslash}p{(\columnwidth - 8\tabcolsep) * \real{0.4000} + 2\tabcolsep}}{%
. . . . . . .} & 37 \\
3.5 &
\multicolumn{2}{>{\raggedright\arraybackslash}p{(\columnwidth - 8\tabcolsep) * \real{0.4000} + 2\tabcolsep}}{%
Description textuelle du sous-cas d'utilisation «Création du flux de
travail»} & . . & 37 \\
3.6 &
\multicolumn{3}{>{\raggedright\arraybackslash}p{(\columnwidth - 8\tabcolsep) * \real{0.6000} + 4\tabcolsep}}{%
Description textuelle du sous-cas d'utilisation «Modification du flux de
travail»} & 38 \\
3.7 &
\multicolumn{3}{>{\raggedright\arraybackslash}p{(\columnwidth - 8\tabcolsep) * \real{0.6000} + 4\tabcolsep}}{%
\begin{minipage}[t]{\linewidth}\raggedright
\begin{quote}
Description textuelle du sous-cas d'utilisation «Suppression du flux de
travail»
\end{quote}
\end{minipage}} & 38 \\
3.8 &
\multicolumn{3}{>{\raggedright\arraybackslash}p{(\columnwidth - 8\tabcolsep) * \real{0.6000} + 4\tabcolsep}}{%
Description textuelle du sous-cas d'utilisation «Authentification» . . .
. . . . .} & 39 \\
3.9 &
\multicolumn{3}{>{\raggedright\arraybackslash}p{(\columnwidth - 8\tabcolsep) * \real{0.6000} + 4\tabcolsep}}{%
Description textuelle du sous-cas d'utilisation «Ajout d'une
Organisation» . . .} & 39 \\
\bottomrule()
\end{longtable}

\begin{quote}
3.10 Description textuelle du sous-cas d'utilisation «Suppression d'une
Organisation» 40

3.11 Description textuelle du sous-cas d'utilisation «Modification d'une
Organisation» 40
\end{quote}

\begin{longtable}[]{@{}
  >{\raggedright\arraybackslash}p{(\columnwidth - 6\tabcolsep) * \real{0.2500}}
  >{\raggedright\arraybackslash}p{(\columnwidth - 6\tabcolsep) * \real{0.2500}}
  >{\raggedright\arraybackslash}p{(\columnwidth - 6\tabcolsep) * \real{0.2500}}
  >{\raggedright\arraybackslash}p{(\columnwidth - 6\tabcolsep) * \real{0.2500}}@{}}
\toprule()
\begin{minipage}[b]{\linewidth}\raggedright
4.1
\end{minipage} &
\multicolumn{2}{>{\raggedright\arraybackslash}p{(\columnwidth - 6\tabcolsep) * \real{0.5000} + 2\tabcolsep}}{%
\begin{minipage}[b]{\linewidth}\raggedright
Environnement Logiciel du Travail . . . . . . . . . . . . . . . . . . .
. . . . .
\end{minipage}} & \begin{minipage}[b]{\linewidth}\raggedright
44
\end{minipage} \\
\midrule()
\endhead
4.2 &
\multicolumn{2}{>{\raggedright\arraybackslash}p{(\columnwidth - 6\tabcolsep) * \real{0.5000} + 2\tabcolsep}}{%
Environnement Matériel du Travail . . . . . . . . . . . . . . . . . . .
. . . . .} & 45 \\
\multicolumn{2}{@{}>{\raggedright\arraybackslash}p{(\columnwidth - 6\tabcolsep) * \real{0.5000} + 2\tabcolsep}}{%
\begin{minipage}[t]{\linewidth}\raggedright
\begin{quote}
Dhia Abdelli
\end{quote}
\end{minipage}} &
\multicolumn{2}{>{\raggedright\arraybackslash}p{(\columnwidth - 6\tabcolsep) * \real{0.5000} + 2\tabcolsep}@{}}{%
Page xii} \\
\bottomrule()
\end{longtable}

\begin{quote}
\textbf{LISTE DES ACRONYMES}

\textbf{SOAR} Security Orchestration, Automation, and Response
\textbf{SOC} Security Operations Center\\
\textbf{DOS} Denial of Service\\
\textbf{EDR} Endpoint Detection and Response\\
\textbf{SIEM} Security Information Event Management\\
\textbf{NIPS} Network Intrusion Prevention Systems\\
\textbf{NIDS} Network Intrusion Detection Systems\\
\textbf{VM} Virtual Machine\\
\textbf{IoC} Indicator of Compromise
\end{quote}

\begin{longtable}[]{@{}
  >{\raggedright\arraybackslash}p{(\columnwidth - 2\tabcolsep) * \real{0.5000}}
  >{\raggedright\arraybackslash}p{(\columnwidth - 2\tabcolsep) * \real{0.5000}}@{}}
\toprule()
\begin{minipage}[b]{\linewidth}\raggedright
\begin{quote}
Dhia Abdelli
\end{quote}
\end{minipage} & \begin{minipage}[b]{\linewidth}\raggedright
Page xiii
\end{minipage} \\
\midrule()
\endhead
\bottomrule()
\end{longtable}

\begin{quote}
\textbf{INTRODUCTION GÉNÉRALE}

Dans un monde de plus en plus connecté, les cyberattaques ont pris ces
dernières années une augmentation sans précédent. La pandémie de
COVID-19 a provoqué une transition massive vers le travail à distance et
a ouvert de nouvelles portes aux cyberattaquants, qui ont saisi
l'occasion pour intensifier leurs tentatives d'intrusion.

Alors que les bureaux fermaient et que les employés se dispersaient dans
des espaces de travail virtuels, les organisations se sont trouvées
contraintes de s'adapter à une nouvelle réalité. Cette transition, bien
qu'essentielle pour assurer la continuité des activités, a
malheureusement révélé des vulnérabilités imprévues. Les mécanismes de
défense traditionnels comme le pare-feu, NIDS et NIPS, initialement
conçus pour protéger les environnements en réseau local, se sont avérés
moins efficaces dans ce nouvel univers virtuel, ouvrant la voie à un
éventail de menaces inédites.

Parmi les défis majeurs auxquels les organisations sont confrontées, la
protection des données occupe une place centrale. Les informations
clients, les secrets d'entreprise, les données stratégiques tout ces
biens inestimables sont devenus des cibles privilégiées pour les
cyberattaquants. Cetteévolution a déclenché une véritable montée en
flèche des tentatives d'intrusion, toutes visant às'emparer de ces
trésors numériques.

Dans ce contexte, la société Advancia a spécialement choisi notre sujet
de fin d'études visant à renforcer la sécurité de ses clients face aux
menaces actuelles, en mettant en œuvre une solution avancée et
automatisée pour sécuriser leurs infrastructures.

Pour répondre aux différentes exigences du projet, il est essentiel
d'avoir une compréhension globale des différentes perspectives. Ainsi,
nous avons soigneusement conçu le plan suivant pour la structure de ce
rapport :

\emph{•} Le chapitre d'ouverture est le premier chapitre de notre
rapport qui porte sur la présentation de l'organisme d'accueil,
problématique , la solution proposée ainsi que la méthodologie de
travail.
\end{quote}

\begin{longtable}[]{@{}
  >{\raggedright\arraybackslash}p{(\columnwidth - 2\tabcolsep) * \real{0.5000}}
  >{\raggedright\arraybackslash}p{(\columnwidth - 2\tabcolsep) * \real{0.5000}}@{}}
\toprule()
\begin{minipage}[b]{\linewidth}\raggedright
\begin{quote}
Dhia Abdelli
\end{quote}
\end{minipage} & \begin{minipage}[b]{\linewidth}\raggedright
Page xiv
\end{minipage} \\
\midrule()
\endhead
\bottomrule()
\end{longtable}

\begin{quote}
INTRODUCTION GÉNÉRALE

\emph{•} Dans le deuxième chapitre nommé «État de l'art» nous
explorerons les fondements essentiels de la cybersécurité. Nous
éclaircirons également la signification de SOAR et mettrons en

avant son importance. Enfin, nous aborderons les choix des outils qui
seront mis en œuvre

dans le cadre de notre projet.

\emph{•} Pour le troisième chapitre, nommé «Analyse et Conceptiont» nous
présenterons en détail la spécification des besoins, ainsi que la
conception et l'architecture de notre projet.

\emph{•} Dans le dernier chapitre, nous examinerons notre solution
implémentée. Nous détaillerons l'environnement utilisé pour
l'installation et conclurons en montrant des scénarios de flux

de travail réels

Nous terminerons ce rapport par une conclusion générale qui résumera les
résultats du projet et

présentera les perspectives potentielles de notre projet.
\end{quote}

\begin{longtable}[]{@{}
  >{\raggedright\arraybackslash}p{(\columnwidth - 2\tabcolsep) * \real{0.5000}}
  >{\raggedright\arraybackslash}p{(\columnwidth - 2\tabcolsep) * \real{0.5000}}@{}}
\toprule()
\begin{minipage}[b]{\linewidth}\raggedright
\begin{quote}
Dhia Abdelli
\end{quote}
\end{minipage} & \begin{minipage}[b]{\linewidth}\raggedright
Page 1
\end{minipage} \\
\midrule()
\endhead
\bottomrule()
\end{longtable}

Chapitre

\begin{longtable}[]{@{}
  >{\raggedright\arraybackslash}p{(\columnwidth - 2\tabcolsep) * \real{0.5000}}
  >{\raggedright\arraybackslash}p{(\columnwidth - 2\tabcolsep) * \real{0.5000}}@{}}
\toprule()
\begin{minipage}[b]{\linewidth}\raggedright
\begin{longtable}[]{@{}
  >{\raggedright\arraybackslash}p{(\columnwidth - 0\tabcolsep) * \real{1.0000}}@{}}
\toprule()
\begin{minipage}[b]{\linewidth}\raggedright
\textbf{1}
\end{minipage} \\
\midrule()
\endhead
\bottomrule()
\end{longtable}
\end{minipage} & \begin{minipage}[b]{\linewidth}\raggedright
\begin{quote}
\textbf{Contexte général du projet}
\end{quote}
\end{minipage} \\
\midrule()
\endhead
\bottomrule()
\end{longtable}

\begin{quote}
\textbf{Sommaire}
\end{quote}

\begin{longtable}[]{@{}
  >{\raggedright\arraybackslash}p{(\columnwidth - 18\tabcolsep) * \real{0.1000}}
  >{\raggedright\arraybackslash}p{(\columnwidth - 18\tabcolsep) * \real{0.1000}}
  >{\raggedright\arraybackslash}p{(\columnwidth - 18\tabcolsep) * \real{0.1000}}
  >{\raggedright\arraybackslash}p{(\columnwidth - 18\tabcolsep) * \real{0.1000}}
  >{\raggedright\arraybackslash}p{(\columnwidth - 18\tabcolsep) * \real{0.1000}}
  >{\raggedright\arraybackslash}p{(\columnwidth - 18\tabcolsep) * \real{0.1000}}
  >{\raggedright\arraybackslash}p{(\columnwidth - 18\tabcolsep) * \real{0.1000}}
  >{\raggedright\arraybackslash}p{(\columnwidth - 18\tabcolsep) * \real{0.1000}}
  >{\raggedright\arraybackslash}p{(\columnwidth - 18\tabcolsep) * \real{0.1000}}
  >{\raggedright\arraybackslash}p{(\columnwidth - 18\tabcolsep) * \real{0.1000}}@{}}
\toprule()
\begin{minipage}[b]{\linewidth}\raggedright
\textbf{1.1}
\end{minipage} &
\multicolumn{2}{>{\raggedright\arraybackslash}p{(\columnwidth - 18\tabcolsep) * \real{0.2000} + 2\tabcolsep}}{%
\begin{minipage}[b]{\linewidth}\raggedright
\begin{quote}
\textbf{Introduction}
\end{quote}
\end{minipage}} &
\multicolumn{6}{>{\raggedright\arraybackslash}p{(\columnwidth - 18\tabcolsep) * \real{0.6000} + 10\tabcolsep}}{%
\begin{minipage}[b]{\linewidth}\raggedright
\textbf{. . . . . . . . . . . . . . . . . . . . . . . . . . . . .}
\end{minipage}} & \begin{minipage}[b]{\linewidth}\raggedright
\textbf{3}
\end{minipage} \\
\midrule()
\endhead
\textbf{1.2} &
\multicolumn{7}{>{\raggedright\arraybackslash}p{(\columnwidth - 18\tabcolsep) * \real{0.7000} + 12\tabcolsep}}{%
\textbf{Présentation de l'organisme d'accueil}} & \textbf{. . . . . . .
. . . . . . .} & \textbf{3} \\
\multirow{6}{*}{\textbf{1.3}} & 1.2.1 &
\begin{minipage}[t]{\linewidth}\raggedright
\begin{quote}
Géneralités
\end{quote}
\end{minipage} &
\multicolumn{6}{>{\raggedright\arraybackslash}p{(\columnwidth - 18\tabcolsep) * \real{0.6000} + 10\tabcolsep}}{%
. . . . . . . . . . . . . . . . . . . . . . . . . . . . .} & 3 \\
& 1.2.2 &
\multicolumn{7}{>{\raggedright\arraybackslash}p{(\columnwidth - 18\tabcolsep) * \real{0.7000} + 12\tabcolsep}}{%
Organisation Interne . . . . . . . . . . . . . . . . . . . . . . . .} &
4 \\
& 1.2.3 &
\multicolumn{7}{>{\raggedright\arraybackslash}p{(\columnwidth - 18\tabcolsep) * \real{0.7000} + 12\tabcolsep}}{%
Services . . . . . . . . . . . . . . . . . . . . . . . . . . . . . . .}
& 4 \\
& 1.2.4 &
\multicolumn{7}{>{\raggedright\arraybackslash}p{(\columnwidth - 18\tabcolsep) * \real{0.7000} + 12\tabcolsep}}{%
Clients . . . . . . . . . . . . . . . . . . . . . . . . . . . . . . . .}
& 5 \\
& 1.2.5 &
\multicolumn{2}{>{\raggedright\arraybackslash}p{(\columnwidth - 18\tabcolsep) * \real{0.2000} + 2\tabcolsep}}{%
Partenaires} &
\multicolumn{5}{>{\raggedright\arraybackslash}p{(\columnwidth - 18\tabcolsep) * \real{0.5000} + 8\tabcolsep}}{%
. . . . . . . . . . . . . . . . . . . . . . . . . . . . .} & 6 \\
&
\multicolumn{6}{>{\raggedright\arraybackslash}p{(\columnwidth - 18\tabcolsep) * \real{0.6000} + 10\tabcolsep}}{%
\textbf{Contexte général du Projet}} &
\multicolumn{2}{>{\raggedright\arraybackslash}p{(\columnwidth - 18\tabcolsep) * \real{0.2000} + 2\tabcolsep}}{%
\textbf{. . . . . . . . . . . . . . . . . . . .}} & \textbf{6} \\
\multirow{4}{*}{\textbf{1.4}} & 1.3.1 &
\multicolumn{7}{>{\raggedright\arraybackslash}p{(\columnwidth - 18\tabcolsep) * \real{0.7000} + 12\tabcolsep}}{%
Présentation du projet . . . . . . . . . . . . . . . . . . . . . . .} &
6 \\
& 1.3.2 &
\multicolumn{3}{>{\raggedright\arraybackslash}p{(\columnwidth - 18\tabcolsep) * \real{0.3000} + 4\tabcolsep}}{%
Problématique} &
\multicolumn{4}{>{\raggedright\arraybackslash}p{(\columnwidth - 18\tabcolsep) * \real{0.4000} + 6\tabcolsep}}{%
. . . . . . . . . . . . . . . . . . . . . . . . . . .} & 7 \\
& 1.3.3 &
\multicolumn{7}{>{\raggedright\arraybackslash}p{(\columnwidth - 18\tabcolsep) * \real{0.7000} + 12\tabcolsep}}{%
Solution proposée . . . . . . . . . . . . . . . . . . . . . . . . . .} &
8 \\
&
\multicolumn{8}{>{\raggedright\arraybackslash}p{(\columnwidth - 18\tabcolsep) * \real{0.8000} + 14\tabcolsep}}{%
\textbf{Méthodologie de travail . . . . . . . . . . . . . . . . . . . .
. . .}} & \textbf{8} \\
\multirow{4}{*}{\textbf{1.5}} & 1.4.1 &
\multicolumn{4}{>{\raggedright\arraybackslash}p{(\columnwidth - 18\tabcolsep) * \real{0.4000} + 6\tabcolsep}}{%
Méthode classique} &
\multicolumn{3}{>{\raggedright\arraybackslash}p{(\columnwidth - 18\tabcolsep) * \real{0.3000} + 4\tabcolsep}}{%
. . . . . . . . . . . . . . . . . . . . . . . . .} & 8 \\
& 1.4.2 &
\multicolumn{7}{>{\raggedright\arraybackslash}p{(\columnwidth - 18\tabcolsep) * \real{0.7000} + 12\tabcolsep}}{%
Méthode AGILE . . . . . . . . . . . . . . . . . . . . . . . . . .} &
9 \\
& 1.4.3 &
\multicolumn{7}{>{\raggedright\arraybackslash}p{(\columnwidth - 18\tabcolsep) * \real{0.7000} + 12\tabcolsep}}{%
Planification du projet . . . . . . . . . . . . . . . . . . . . . . .} &
\begin{minipage}[t]{\linewidth}\raggedright
\begin{quote}
10
\end{quote}
\end{minipage} \\
&
\multicolumn{8}{>{\raggedright\arraybackslash}p{(\columnwidth - 18\tabcolsep) * \real{0.8000} + 14\tabcolsep}}{%
\multirow{2}{*}{\textbf{CONCLUSION . . . . . . . . . . . . . . . . . . .
. . . . . . . . .}}} & \begin{minipage}[t]{\linewidth}\raggedright
\begin{quote}
\textbf{11}
\end{quote}
\end{minipage} \\
Dhia Abdelli & & & & & & & & & Page 2 \\
\bottomrule()
\end{longtable}

\begin{quote}
CONTEXTE GÉNÉRAL DU PROJET

\textbf{1.1 Introduction}

Dans ce chapitre, nous allons établir le cadre global de notre projet en
présentant l'entitéd'accueil "Advancia IT" au sein de laquelle notre
stage a été réalisé. Nous aborderons ensuite le contexte du projet et
une problématique qui servira de base à la définition de notre solution
proposée. Enfin, nous conclurons en exposant notre approche
méthodologique et en détaillant la planification des diverses tâches
réalisées.

\textbf{1.2 Présentation de l'organisme d'accueil}

\textbf{1.2.1} \textbf{Géneralités}

Advancia une filiale du groupe SPG (Software Productivity Group), occupe
une position prépondérante sur le marché de l'informatique en Tunisie
avec une spécialisation pointue dans les services Cloud et les services
IT managés. L'entreprise accompagne activement ses clients pour relever
des défis cruciaux tels que la disponibilité et la performance des
applications, la mobilité, l'optimisation de la productivité et la
sécurité.{[}1{]}

Grâce à des équipes techniques hautement qualifiées et certifiées sur
les solutions leaders du marché (Microsoft, Symantec, Veritas, VMware,
HPE, Fortinet, etc.), Advancia va au-delà du simple conseil en
architecture technique. Elle assure également une gestion globale et
méticuleuse du système d'information, offrant ainsi à ses clients un
soutien complet et sur mesure. Cette approche intégrée garantit une
réponse agile et efficace à leurs besoins informatiques les plus
exigeants.(voir firgure 1.1)
\end{quote}

\includegraphics[width=1.89028in,height=1.09861in]{vertopal_fa5b2edd966b48bfbc6a830f7abfbc35/media/image3.png}

\textbf{FIGURE 1.1 -- Advancia IT}

\begin{longtable}[]{@{}
  >{\raggedright\arraybackslash}p{(\columnwidth - 2\tabcolsep) * \real{0.5000}}
  >{\raggedright\arraybackslash}p{(\columnwidth - 2\tabcolsep) * \real{0.5000}}@{}}
\toprule()
\begin{minipage}[b]{\linewidth}\raggedright
\begin{quote}
Dhia Abdelli
\end{quote}
\end{minipage} & \begin{minipage}[b]{\linewidth}\raggedright
Page 3
\end{minipage} \\
\midrule()
\endhead
\bottomrule()
\end{longtable}

\begin{quote}
CONTEXTE GÉNÉRAL DU PROJET
\end{quote}

\begin{longtable}[]{@{}
  >{\raggedright\arraybackslash}p{(\columnwidth - 2\tabcolsep) * \real{0.5000}}
  >{\raggedright\arraybackslash}p{(\columnwidth - 2\tabcolsep) * \real{0.5000}}@{}}
\toprule()
\begin{minipage}[b]{\linewidth}\raggedright
\begin{quote}
\textbf{1.2.2}
\end{quote}
\end{minipage} & \begin{minipage}[b]{\linewidth}\raggedright
\begin{quote}
\textbf{Organisation Interne}
\end{quote}
\end{minipage} \\
\midrule()
\endhead
\bottomrule()
\end{longtable}

\begin{quote}
\textbf{Advancia IT} est constituée de trois principales unités :
Marketing, Technique et Vente. La stratégie de l'entreprise vise à
développer les compétences et à maximiser l'engagement du personnel,
plaçant ainsi les collaborateurs au cœur de la réussite.(voir figure
1.2)

\includegraphics[width=5.66944in,height=3.16805in]{vertopal_fa5b2edd966b48bfbc6a830f7abfbc35/media/image4.png}

\textbf{FIGURE 1.2 -- Schéma organisationnel d'advancia}

\textbf{1.2.3} \textbf{Services}

Le Centre de Service Proximity d'Advancia se démarque en tant que
plateforme avancée, hébergeant une gamme diversifiée de composants
technologiques. Cette plateforme est prise en charge par des experts
techniques et propose des services managés sous le modèle
IT-as-a-Service. Nous citons quelques services de Advancia (voir figure
1.3).
\end{quote}

\begin{longtable}[]{@{}
  >{\raggedright\arraybackslash}p{(\columnwidth - 2\tabcolsep) * \real{0.5000}}
  >{\raggedright\arraybackslash}p{(\columnwidth - 2\tabcolsep) * \real{0.5000}}@{}}
\toprule()
\begin{minipage}[b]{\linewidth}\raggedright
\begin{quote}
Dhia Abdelli
\end{quote}
\end{minipage} & \begin{minipage}[b]{\linewidth}\raggedright
Page 4
\end{minipage} \\
\midrule()
\endhead
\bottomrule()
\end{longtable}

\begin{quote}
CONTEXTE GÉNÉRAL DU PROJET
\end{quote}

\includegraphics[width=5.66944in,height=3.20694in]{vertopal_fa5b2edd966b48bfbc6a830f7abfbc35/media/image5.png}

\textbf{FIGURE 1.3 -- Services Proximity d'Avancia IT}

\begin{quote}
\textbf{1.2.4} \textbf{Clients}

Guidée par une vision stratégique ambitieuse, Advancia s'engage
pleinement à consolider et à établir des relations riches et durables
avec sa clientèle diversifiée. Parmi les entreprises qui ont choisi
Advancia comme partenaire de confiance, citons celles qui ont bénéficié
de solutions sur mesure pour leurs projets, comme illustré dans la
figure 1.4.
\end{quote}

\includegraphics[width=5.66944in,height=2.76528in]{vertopal_fa5b2edd966b48bfbc6a830f7abfbc35/media/image6.png}

\textbf{FIGURE 1.4 -- Clients Advancia IT}

\begin{longtable}[]{@{}
  >{\raggedright\arraybackslash}p{(\columnwidth - 2\tabcolsep) * \real{0.5000}}
  >{\raggedright\arraybackslash}p{(\columnwidth - 2\tabcolsep) * \real{0.5000}}@{}}
\toprule()
\begin{minipage}[b]{\linewidth}\raggedright
\begin{quote}
Dhia Abdelli
\end{quote}
\end{minipage} & \begin{minipage}[b]{\linewidth}\raggedright
Page 5
\end{minipage} \\
\midrule()
\endhead
\bottomrule()
\end{longtable}

\begin{quote}
CONTEXTE GÉNÉRAL DU PROJET
\end{quote}

\begin{longtable}[]{@{}
  >{\raggedright\arraybackslash}p{(\columnwidth - 2\tabcolsep) * \real{0.5000}}
  >{\raggedright\arraybackslash}p{(\columnwidth - 2\tabcolsep) * \real{0.5000}}@{}}
\toprule()
\begin{minipage}[b]{\linewidth}\raggedright
\begin{quote}
\textbf{1.2.5}
\end{quote}
\end{minipage} & \begin{minipage}[b]{\linewidth}\raggedright
\begin{quote}
\textbf{Partenaires}
\end{quote}
\end{minipage} \\
\midrule()
\endhead
\bottomrule()
\end{longtable}

\begin{quote}
Pour assurer un niveau de service de premier plan, Advancia a conclu
plusieurs accords de partenariat avec des leaders internationaux. La
figure ci-dessous met en lumière ces précieuses collaborations.

\includegraphics[width=5.66944in,height=2.96806in]{vertopal_fa5b2edd966b48bfbc6a830f7abfbc35/media/image7.png}
\end{quote}

\textbf{FIGURE 1.5 -- Partenaires d'Advancia IT}

\begin{quote}
\textbf{1.3 Contexte général du Projet}

\textbf{1.3.1} \textbf{Présentation du projet}

Le sujet abordé dans ce rapport se concentre sur le déploiement d'une
solution SOAR-as-a-Service. Celle-ci excelle dans la détection des
attaques et agit automatiquement en réponse à ces incidents. Dans la
suite, nous allons détailler le projet, en exposant ses spécificités,
ses détails ainsi que les objectifs visés par sa conception.
\end{quote}

\begin{longtable}[]{@{}
  >{\raggedright\arraybackslash}p{(\columnwidth - 2\tabcolsep) * \real{0.5000}}
  >{\raggedright\arraybackslash}p{(\columnwidth - 2\tabcolsep) * \real{0.5000}}@{}}
\toprule()
\begin{minipage}[b]{\linewidth}\raggedright
\begin{quote}
Dhia Abdelli
\end{quote}
\end{minipage} & \begin{minipage}[b]{\linewidth}\raggedright
Page 6
\end{minipage} \\
\midrule()
\endhead
\bottomrule()
\end{longtable}

\begin{quote}
CONTEXTE GÉNÉRAL DU PROJET
\end{quote}

\begin{longtable}[]{@{}
  >{\raggedright\arraybackslash}p{(\columnwidth - 2\tabcolsep) * \real{0.5000}}
  >{\raggedright\arraybackslash}p{(\columnwidth - 2\tabcolsep) * \real{0.5000}}@{}}
\toprule()
\begin{minipage}[b]{\linewidth}\raggedright
\begin{quote}
\textbf{1.3.2}
\end{quote}
\end{minipage} & \begin{minipage}[b]{\linewidth}\raggedright
\begin{quote}
\textbf{Problématique}
\end{quote}
\end{minipage} \\
\midrule()
\endhead
\bottomrule()
\end{longtable}

\begin{quote}
Dans les dernières années, les attaques en ligne ont enregistré une
augmentation, particulièrement après la pandémie de Covid-19. Cette
augmentation découle de plusieurs facteurs, notamment la généralisation
du télétravail et l'essor massif des activités numériques. Provoquant
une hausse notable des attaques touchant divers secteurs tels que les
entreprises, la santé, les institutions financières et les
gouvernements.

Les solutions traditionnelles telles que les antivirus, les pare-feu,
les NIDS et les NIPS ne sont plus suffisantes en raison de la
sophistication des attaques et de leur nombre écrasant. Durant ces
dernières années, le SOC (Security operations center) est devenu la
solution pour de nombreuses entreprises. Il constitue une fonction
centralisée au sein d'une organisation mobilisant des personnes, des
processus et des technologies pour surveiller en permanence et renforcer
la sécurité de l'organisation. Cela inclut la prévention, la détection
et la réponse aux incidents.

Mais, Le SOC a actuellement trois problèmes majeurs, qui sont les
suivants :

\emph{•} Le problème du personnel\\
\emph{•} Le problème du processus\\
\emph{•} Le problème des technologies\\
Donc, le principal défi auquel fait face le SOC concerne le
personnel.Cela englobe la pénurie de personnel formé en cybersécurité,
une formation insuffisante pour suivre l'évolution des menaces, et une
rotation élevée du personnel.

Un autre défi significatif est lié au processus. La latence de processus
se manifeste par l'exécution majoritairement manuelle des plans de
réponse aux incidents. De plus, la fatigue d'alerte résulte du grand
nombre d'alertes qui surcharge le personnel du SOC.

Le dernier défi pour le SOC concerne les technologies. Il ya un manque
d'outils appropriés ce qui affecte l'efficacité du SOC. De plus,
l'utilisation d'outils provenant de différents fournisseurs est
courante, ce qui complique l'intégration.

Ainsi, comment Advancia peut-elle efficacement déployer un Service
SOAR-as-a-service, intégrant des technologies avancées, pour détecter
avec précision les attaques, répondre automatiquement
\end{quote}

\begin{longtable}[]{@{}
  >{\raggedright\arraybackslash}p{(\columnwidth - 2\tabcolsep) * \real{0.5000}}
  >{\raggedright\arraybackslash}p{(\columnwidth - 2\tabcolsep) * \real{0.5000}}@{}}
\toprule()
\begin{minipage}[b]{\linewidth}\raggedright
\begin{quote}
Dhia Abdelli
\end{quote}
\end{minipage} & \begin{minipage}[b]{\linewidth}\raggedright
Page 7
\end{minipage} \\
\midrule()
\endhead
\bottomrule()
\end{longtable}

\begin{quote}
CONTEXTE GÉNÉRAL DU PROJET

à des incidents et garantir une protection fiable des infrastructures
cruciales de leurs clients contre les menaces?

\textbf{1.3.3} \textbf{Solution proposée}

Afin d'assurer la sécurité des infrastructures cruciales de leurs
clients, Advancia propose d'implémenter dans leur SOC (Security
operations center) la solution \textbf{SOAR-as-a-Service} entièrement
basée sur des logiciels Open Source. Cette solution sera intégrée dans
leurs services gérés proximity.

SOAR représente l'évolution naturelle du SOC portée par l'automatisation
des opérations de sécurité. Cette solution combine quatre types d'outils
bien connus : l'orchestration et l'automatisation de la sécurité, la
réponse aux incidents et les solutions de renseignement sur les menaces.

Les principaux avantages de SOAR sont :

\emph{•} Des processus SOC plus rapides\\
\emph{•} Réduction des opérations manuelles et normalisation des
processus\emph{•} Renseignements optimisés sur les menaces\\
\emph{•} Atténue la fatigue d'alerte\\
\emph{•} Rapports et mesures automatisés\\
\emph{•} Réduire les dégâts des attaques

\textbf{1.4 Méthodologie de travail}

Assurer le succès d'un projet repose en grande partie sur le choix
adéquat de la méthodologieà suivre. En effet, une planification
rigoureuse permet une répartition efficace des tâches, une organisation
optimale et une gestion précise du temps.

\textbf{1.4.1} \textbf{Méthode classique}

La méthode classique, illustrée par le \textbf{cycle de développement en
V}, adopte une approche linéaire. Cela implique une séquence de phases,
chacune devait être accomplie avant de passer
\end{quote}

\begin{longtable}[]{@{}
  >{\raggedright\arraybackslash}p{(\columnwidth - 2\tabcolsep) * \real{0.5000}}
  >{\raggedright\arraybackslash}p{(\columnwidth - 2\tabcolsep) * \real{0.5000}}@{}}
\toprule()
\begin{minipage}[b]{\linewidth}\raggedright
\begin{quote}
Dhia Abdelli
\end{quote}
\end{minipage} & \begin{minipage}[b]{\linewidth}\raggedright
Page 8
\end{minipage} \\
\midrule()
\endhead
\bottomrule()
\end{longtable}

\begin{quote}
CONTEXTE GÉNÉRAL DU PROJET

à la suivante.

Cette forme en V comporte deux étapes distinctes : une phase ascendante,
puis une phase descendante. Dans cette méthode de gestion de projet,
chaque phase de développement est systématiquement suivie d'une phase de
validation.

Cette méthode implique de fixer les spécifications, les fonctionnalités
et les attentes du client dès le début. Elle repose sur un processus
bien défini, une documentation détaillée et nécessite peu d'intervention
du client.{[}2{]}

\textbf{1.4.2} \textbf{Méthode AGILE}

La méthode Agile place le client au cœur du projet, en mettant en
lumière sa satisfaction totale. Elle se caractérise par sa légèreté et
son approche souple, découpant le projet en petites itérations. Dans
cette démarche, l'équipe définit des objectifs à court terme, créant
ainsi des sous-projets. À chaque réussite d'un objectif, l'équipe peut
passer au suivant progressant de manière incrémentielle vers l'objectif
global. L'aspect essentiel réside dans la validation de chaque étape par
le client, favorisant ainsi une communication continue avec l'équipe.
Pour notre projet de fin d'études, nous avons opté pour la méthode
\textbf{kanban}, une pierre précieuse de l'approche Agile. Notre
décision découle de sa parfaite harmonie avec notre contexte.{[}2{]}

\textbf{1.4.2.1} \textbf{Méthode Kanban}

Kanban vient du japonais et signifie "étiquettes". C'est une méthode
Agile qui aide à améliorer la production en suivant les tâches à faire
pour répondre aux besoins du client. On l'utilise en utilisant un
tableau divisé en trois parties : \textbf{"A faire"}, \textbf{"En
cours"} , \textbf{"Réalisé"}. Chaque section sait ce qu'elle doit faire,
comment et quand.{[}3{]}

\emph{•} \textbf{Tableau Kanban}\\
Trello se présente comme un outil de collaboration et de gestion de
projet gratuit. Il puise son inspiration dans la méthodologie Kanban de
l'Agile. En réalité, il s'agit d'une application web qui s'apparente à
un tableau virtuel. Ce tableau est constitué de cartes distinctes,
chacune symbolisant une tâche spécifique, savamment organisées pour
assurer un suivi efficace de l'évolution
\end{quote}

\begin{longtable}[]{@{}
  >{\raggedright\arraybackslash}p{(\columnwidth - 2\tabcolsep) * \real{0.5000}}
  >{\raggedright\arraybackslash}p{(\columnwidth - 2\tabcolsep) * \real{0.5000}}@{}}
\toprule()
\begin{minipage}[b]{\linewidth}\raggedright
\begin{quote}
Dhia Abdelli
\end{quote}
\end{minipage} & \begin{minipage}[b]{\linewidth}\raggedright
Page 9
\end{minipage} \\
\midrule()
\endhead
\bottomrule()
\end{longtable}

\begin{quote}
CONTEXTE GÉNÉRAL DU PROJET

de notre projet. Dans l'illustration ci-dessous, nous découvrons le
tableau Kanban généré par l'outil de gestion de projet Trello.

\includegraphics[width=6.29861in,height=4.24722in]{vertopal_fa5b2edd966b48bfbc6a830f7abfbc35/media/image8.png}
\end{quote}

\textbf{FIGURE 1.6 -- Trello}

\begin{quote}
\textbf{1.4.3} \textbf{Planification du projet}

Afin de garantir une gestion efficace de notre projet et d'assurer un
équilibre entre le temps et la progression des tâches durant notre
période de stage, il est impératif de concevoir un diagramme
représentant la répartition des activités. Notre planification est
visuellement présentée dans le diagramme de Gantt, comme illustré dans
la figure ci-après.
\end{quote}

\begin{longtable}[]{@{}
  >{\raggedright\arraybackslash}p{(\columnwidth - 2\tabcolsep) * \real{0.5000}}
  >{\raggedright\arraybackslash}p{(\columnwidth - 2\tabcolsep) * \real{0.5000}}@{}}
\toprule()
\begin{minipage}[b]{\linewidth}\raggedright
\begin{quote}
Dhia Abdelli
\end{quote}
\end{minipage} & \begin{minipage}[b]{\linewidth}\raggedright
Page 10
\end{minipage} \\
\midrule()
\endhead
\bottomrule()
\end{longtable}

\begin{quote}
CONTEXTE GÉNÉRAL DU PROJET
\end{quote}

\includegraphics[width=5.66944in,height=1.9625in]{vertopal_fa5b2edd966b48bfbc6a830f7abfbc35/media/image9.png}

\textbf{FIGURE 1.7 -- Diagramme de Gantt}

\begin{quote}
\textbf{1.5 CONCLUSION}

Dans ce premier chapitre, nous avons présenté l'organisme d'accueil,
évoqué le contexte de notre projet et exposé notre approche en matière
de planification. Cette base nous prépare à une analyse plus approfondie
dans le chapitre suivant où nous examinerons en détail l'état actuel de
la sécurité informatique. De plus, nous présenterons en profondeur notre
solution SOAR.
\end{quote}

\begin{longtable}[]{@{}
  >{\raggedright\arraybackslash}p{(\columnwidth - 2\tabcolsep) * \real{0.5000}}
  >{\raggedright\arraybackslash}p{(\columnwidth - 2\tabcolsep) * \real{0.5000}}@{}}
\toprule()
\begin{minipage}[b]{\linewidth}\raggedright
\begin{quote}
Dhia Abdelli
\end{quote}
\end{minipage} & \begin{minipage}[b]{\linewidth}\raggedright
Page 11
\end{minipage} \\
\midrule()
\endhead
\bottomrule()
\end{longtable}

Chapitre

\begin{longtable}[]{@{}
  >{\raggedright\arraybackslash}p{(\columnwidth - 0\tabcolsep) * \real{1.0000}}@{}}
\toprule()
\begin{minipage}[b]{\linewidth}\raggedright
\textbf{2}
\end{minipage} \\
\midrule()
\endhead
\bottomrule()
\end{longtable}

\begin{quote}
\textbf{État de l'art}

\textbf{Sommaire}
\end{quote}

\begin{longtable}[]{@{}
  >{\raggedright\arraybackslash}p{(\columnwidth - 16\tabcolsep) * \real{0.1111}}
  >{\raggedright\arraybackslash}p{(\columnwidth - 16\tabcolsep) * \real{0.1111}}
  >{\raggedright\arraybackslash}p{(\columnwidth - 16\tabcolsep) * \real{0.1111}}
  >{\raggedright\arraybackslash}p{(\columnwidth - 16\tabcolsep) * \real{0.1111}}
  >{\raggedright\arraybackslash}p{(\columnwidth - 16\tabcolsep) * \real{0.1111}}
  >{\raggedright\arraybackslash}p{(\columnwidth - 16\tabcolsep) * \real{0.1111}}
  >{\raggedright\arraybackslash}p{(\columnwidth - 16\tabcolsep) * \real{0.1111}}
  >{\raggedright\arraybackslash}p{(\columnwidth - 16\tabcolsep) * \real{0.1111}}
  >{\raggedright\arraybackslash}p{(\columnwidth - 16\tabcolsep) * \real{0.1111}}@{}}
\toprule()
\multirow{24}{*}{\begin{minipage}[b]{\linewidth}\raggedright
\textbf{2.1}

\textbf{2.2}

\textbf{2.3}

\textbf{2.4}

\textbf{2.5}

\textbf{2.6}
\end{minipage}} &
\multicolumn{5}{>{\raggedright\arraybackslash}p{(\columnwidth - 16\tabcolsep) * \real{0.5556} + 8\tabcolsep}}{%
\begin{minipage}[b]{\linewidth}\raggedright
\textbf{Les Piliers de la Cybersécurité}
\end{minipage}} &
\multicolumn{2}{>{\raggedright\arraybackslash}p{(\columnwidth - 16\tabcolsep) * \real{0.2222} + 2\tabcolsep}}{%
\begin{minipage}[b]{\linewidth}\raggedright
\begin{quote}
\textbf{. . . . . . . . . . . . . . . . . .}
\end{quote}
\end{minipage}} & \begin{minipage}[b]{\linewidth}\raggedright
\begin{quote}
\textbf{13}
\end{quote}
\end{minipage} \\
& \begin{minipage}[b]{\linewidth}\raggedright
\begin{quote}
2.1.1
\end{quote}
\end{minipage} &
\multicolumn{6}{>{\raggedright\arraybackslash}p{(\columnwidth - 16\tabcolsep) * \real{0.6667} + 10\tabcolsep}}{%
\begin{minipage}[b]{\linewidth}\raggedright
Cybersécurité . . . . . . . . . . . . . . . . . . . . . . . . . . . .
\end{minipage}} & \begin{minipage}[b]{\linewidth}\raggedright
13
\end{minipage} \\
& \begin{minipage}[b]{\linewidth}\raggedright
\begin{quote}
2.1.2
\end{quote}
\end{minipage} &
\multicolumn{6}{>{\raggedright\arraybackslash}p{(\columnwidth - 16\tabcolsep) * \real{0.6667} + 10\tabcolsep}}{%
\begin{minipage}[b]{\linewidth}\raggedright
Cyberattaque . . . . . . . . . . . . . . . . . . . . . . . . . . . .
\end{minipage}} & \begin{minipage}[b]{\linewidth}\raggedright
13
\end{minipage} \\
& \begin{minipage}[b]{\linewidth}\raggedright
\begin{quote}
2.1.3
\end{quote}
\end{minipage} &
\multicolumn{3}{>{\raggedright\arraybackslash}p{(\columnwidth - 16\tabcolsep) * \real{0.3333} + 4\tabcolsep}}{%
\begin{minipage}[b]{\linewidth}\raggedright
Le modèle triade CIA
\end{minipage}} &
\multicolumn{3}{>{\raggedright\arraybackslash}p{(\columnwidth - 16\tabcolsep) * \real{0.3333} + 4\tabcolsep}}{%
\begin{minipage}[b]{\linewidth}\raggedright
. . . . . . . . . . . . . . . . . . . . . . .
\end{minipage}} & \begin{minipage}[b]{\linewidth}\raggedright
14
\end{minipage} \\
& \begin{minipage}[b]{\linewidth}\raggedright
\begin{quote}
\textbf{SOC}
\end{quote}
\end{minipage} &
\multicolumn{6}{>{\raggedright\arraybackslash}p{(\columnwidth - 16\tabcolsep) * \real{0.6667} + 10\tabcolsep}}{%
\begin{minipage}[b]{\linewidth}\raggedright
\begin{quote}
\textbf{. . . . . . . . . . . . . . . . . . . . . . . . . . . . . . . .
. .}
\end{quote}
\end{minipage}} & \begin{minipage}[b]{\linewidth}\raggedright
\begin{quote}
\textbf{15}
\end{quote}
\end{minipage} \\
& \begin{minipage}[b]{\linewidth}\raggedright
\begin{quote}
2.2.1
\end{quote}
\end{minipage} &
\multicolumn{6}{>{\raggedright\arraybackslash}p{(\columnwidth - 16\tabcolsep) * \real{0.6667} + 10\tabcolsep}}{%
\begin{minipage}[b]{\linewidth}\raggedright
Processus de réponse aux incidents dans un SOC . . . . . . . .
\end{minipage}} & \begin{minipage}[b]{\linewidth}\raggedright
15
\end{minipage} \\
& \begin{minipage}[b]{\linewidth}\raggedright
\begin{quote}
2.2.2
\end{quote}
\end{minipage} &
\multicolumn{6}{>{\raggedright\arraybackslash}p{(\columnwidth - 16\tabcolsep) * \real{0.6667} + 10\tabcolsep}}{%
\begin{minipage}[b]{\linewidth}\raggedright
Défis SOC . . . . . . . . . . . . . . . . . . . . . . . . . . . . . .
\end{minipage}} & \begin{minipage}[b]{\linewidth}\raggedright
16
\end{minipage} \\
&
\multicolumn{2}{>{\raggedright\arraybackslash}p{(\columnwidth - 16\tabcolsep) * \real{0.2222} + 2\tabcolsep}}{%
\begin{minipage}[b]{\linewidth}\raggedright
\textbf{SOAR}
\end{minipage}} &
\multicolumn{5}{>{\raggedright\arraybackslash}p{(\columnwidth - 16\tabcolsep) * \real{0.5556} + 8\tabcolsep}}{%
\begin{minipage}[b]{\linewidth}\raggedright
\begin{quote}
\textbf{. . . . . . . . . . . . . . . . . . . . . . . . . . . . . . . .
.}
\end{quote}
\end{minipage}} & \begin{minipage}[b]{\linewidth}\raggedright
\begin{quote}
\textbf{17}
\end{quote}
\end{minipage} \\
& \begin{minipage}[b]{\linewidth}\raggedright
\begin{quote}
2.3.1
\end{quote}
\end{minipage} &
\multicolumn{6}{>{\raggedright\arraybackslash}p{(\columnwidth - 16\tabcolsep) * \real{0.6667} + 10\tabcolsep}}{%
\begin{minipage}[b]{\linewidth}\raggedright
Que signifie "SOAR" ? . . . . . . . . . . . . . . . . . . . . . . .
\end{minipage}} & \begin{minipage}[b]{\linewidth}\raggedright
18
\end{minipage} \\
& \begin{minipage}[b]{\linewidth}\raggedright
\begin{quote}
2.3.2
\end{quote}
\end{minipage} &
\multicolumn{5}{>{\raggedright\arraybackslash}p{(\columnwidth - 16\tabcolsep) * \real{0.5556} + 8\tabcolsep}}{%
\begin{minipage}[b]{\linewidth}\raggedright
Pourquoi SOAR est-il important ?
\end{minipage}} & \begin{minipage}[b]{\linewidth}\raggedright
\begin{quote}
. . . . . . . . . . . . . . . .
\end{quote}
\end{minipage} & \begin{minipage}[b]{\linewidth}\raggedright
18
\end{minipage} \\
& \begin{minipage}[b]{\linewidth}\raggedright
\begin{quote}
2.3.3
\end{quote}
\end{minipage} &
\multicolumn{6}{>{\raggedright\arraybackslash}p{(\columnwidth - 16\tabcolsep) * \real{0.6667} + 10\tabcolsep}}{%
\begin{minipage}[b]{\linewidth}\raggedright
Composants SOAR . . . . . . . . . . . . . . . . . . . . . . . . .
\end{minipage}} & \begin{minipage}[b]{\linewidth}\raggedright
19
\end{minipage} \\
& \begin{minipage}[b]{\linewidth}\raggedright
\begin{quote}
2.3.4
\end{quote}
\end{minipage} &
\multicolumn{6}{>{\raggedright\arraybackslash}p{(\columnwidth - 16\tabcolsep) * \real{0.6667} + 10\tabcolsep}}{%
\begin{minipage}[b]{\linewidth}\raggedright
Solution SOAR dans un SOC . . . . . . . . . . . . . . . . . . .
\end{minipage}} & \begin{minipage}[b]{\linewidth}\raggedright
22
\end{minipage} \\
&
\multicolumn{7}{>{\raggedright\arraybackslash}p{(\columnwidth - 16\tabcolsep) * \real{0.7778} + 12\tabcolsep}}{%
\begin{minipage}[b]{\linewidth}\raggedright
\begin{quote}
\textbf{Solutions existantes . . . . . . . . . . . . . . . . . . . . . .
. . .}
\end{quote}
\end{minipage}} & \begin{minipage}[b]{\linewidth}\raggedright
\begin{quote}
\textbf{23}
\end{quote}
\end{minipage} \\
&
\multicolumn{3}{>{\raggedright\arraybackslash}p{(\columnwidth - 16\tabcolsep) * \real{0.3333} + 4\tabcolsep}}{%
\begin{minipage}[b]{\linewidth}\raggedright
\textbf{Choix des outils}
\end{minipage}} &
\multicolumn{4}{>{\raggedright\arraybackslash}p{(\columnwidth - 16\tabcolsep) * \real{0.4444} + 6\tabcolsep}}{%
\begin{minipage}[b]{\linewidth}\raggedright
\begin{quote}
\textbf{. . . . . . . . . . . . . . . . . . . . . . . . . . .}
\end{quote}
\end{minipage}} & \begin{minipage}[b]{\linewidth}\raggedright
\begin{quote}
\textbf{24}
\end{quote}
\end{minipage} \\
& \begin{minipage}[b]{\linewidth}\raggedright
\begin{quote}
2.5.1
\end{quote}
\end{minipage} &
\multicolumn{6}{>{\raggedright\arraybackslash}p{(\columnwidth - 16\tabcolsep) * \real{0.6667} + 10\tabcolsep}}{%
\begin{minipage}[b]{\linewidth}\raggedright
SIEM (Elastic Stack) . . . . . . . . . . . . . . . . . . . . . . . .
\end{minipage}} & \begin{minipage}[b]{\linewidth}\raggedright
24
\end{minipage} \\
& \begin{minipage}[b]{\linewidth}\raggedright
\begin{quote}
2.5.2
\end{quote}
\end{minipage} &
\multicolumn{6}{>{\raggedright\arraybackslash}p{(\columnwidth - 16\tabcolsep) * \real{0.6667} + 10\tabcolsep}}{%
\begin{minipage}[b]{\linewidth}\raggedright
Outil de renseignement sur les menaces (MISP) . . . . . . . . .
\end{minipage}} & \begin{minipage}[b]{\linewidth}\raggedright
24
\end{minipage} \\
& \begin{minipage}[b]{\linewidth}\raggedright
\begin{quote}
2.5.3
\end{quote}
\end{minipage} &
\multicolumn{6}{>{\raggedright\arraybackslash}p{(\columnwidth - 16\tabcolsep) * \real{0.6667} + 10\tabcolsep}}{%
\begin{minipage}[b]{\linewidth}\raggedright
Solution de réponse aux incidents de sécurité (DFIR IRIS) . . .
\end{minipage}} & \begin{minipage}[b]{\linewidth}\raggedright
25
\end{minipage} \\
& \begin{minipage}[b]{\linewidth}\raggedright
\begin{quote}
2.5.4
\end{quote}
\end{minipage} &
\multicolumn{6}{>{\raggedright\arraybackslash}p{(\columnwidth - 16\tabcolsep) * \real{0.6667} + 10\tabcolsep}}{%
\begin{minipage}[b]{\linewidth}\raggedright
Outil d'orchestration (Shuffle) . . . . . . . . . . . . . . . . . . .
\end{minipage}} & \begin{minipage}[b]{\linewidth}\raggedright
25
\end{minipage} \\
& \begin{minipage}[b]{\linewidth}\raggedright
\begin{quote}
2.5.5
\end{quote}
\end{minipage} &
\multicolumn{6}{>{\raggedright\arraybackslash}p{(\columnwidth - 16\tabcolsep) * \real{0.6667} + 10\tabcolsep}}{%
\begin{minipage}[b]{\linewidth}\raggedright
Plateforme Honeypot (T-POT) . . . . . . . . . . . . . . . . . .
\end{minipage}} & \begin{minipage}[b]{\linewidth}\raggedright
26
\end{minipage} \\
& \begin{minipage}[b]{\linewidth}\raggedright
\begin{quote}
2.5.6
\end{quote}
\end{minipage} &
\multicolumn{6}{>{\raggedright\arraybackslash}p{(\columnwidth - 16\tabcolsep) * \real{0.6667} + 10\tabcolsep}}{%
\begin{minipage}[b]{\linewidth}\raggedright
Outil de collecte de données T-POT GreedyBear (GreedyBear)
\end{minipage}} & \begin{minipage}[b]{\linewidth}\raggedright
27
\end{minipage} \\
& \begin{minipage}[b]{\linewidth}\raggedright
\begin{quote}
2.5.7
\end{quote}
\end{minipage} &
\multicolumn{6}{>{\raggedright\arraybackslash}p{(\columnwidth - 16\tabcolsep) * \real{0.6667} + 10\tabcolsep}}{%
\begin{minipage}[b]{\linewidth}\raggedright
Outil d'analyse des logiciels malveillants (Cuckoo Sanbox) . . .
\end{minipage}} & \begin{minipage}[b]{\linewidth}\raggedright
27
\end{minipage} \\
& \begin{minipage}[b]{\linewidth}\raggedright
\begin{quote}
2.5.8
\end{quote}
\end{minipage} &
\multicolumn{6}{>{\raggedright\arraybackslash}p{(\columnwidth - 16\tabcolsep) * \real{0.6667} + 10\tabcolsep}}{%
\begin{minipage}[b]{\linewidth}\raggedright
Outil d'alerte (Praeco) . . . . . . . . . . . . . . . . . . . . . . .
\end{minipage}} & \begin{minipage}[b]{\linewidth}\raggedright
28
\end{minipage} \\
& \begin{minipage}[b]{\linewidth}\raggedright
\begin{quote}
2.5.9
\end{quote}
\end{minipage} &
\multicolumn{7}{>{\raggedright\arraybackslash}p{(\columnwidth - 16\tabcolsep) * \real{0.7778} + 12\tabcolsep}@{}}{%
\begin{minipage}[b]{\linewidth}\raggedright
Outil d'investigation numérique et de réponse aux incidents
(Velociraptor) 28
\end{minipage}} \\
&
\multicolumn{7}{>{\raggedright\arraybackslash}p{(\columnwidth - 16\tabcolsep) * \real{0.7778} + 12\tabcolsep}}{%
\begin{minipage}[b]{\linewidth}\raggedright
\begin{quote}
\textbf{CONCLUSION . . . . . . . . . . . . . . . . . . . . . . . . . . .
.}
\end{quote}
\end{minipage}} & \begin{minipage}[b]{\linewidth}\raggedright
\begin{quote}
\textbf{29}
\end{quote}
\end{minipage} \\
\midrule()
\endhead
Dhia Abdelli &
\multicolumn{8}{>{\raggedright\arraybackslash}p{(\columnwidth - 16\tabcolsep) * \real{0.8889} + 14\tabcolsep}@{}}{%
Page 12} \\
\bottomrule()
\end{longtable}

\begin{quote}
ÉTAT DE L'ART

\textbf{2.1 Les Piliers de la Cybersécurité}

\textbf{2.1.1} \textbf{Cybersécurité}

La cybersécurité est l'utilisation combinée de technologies avancées, de
méthodes organisées et de régulations strictes conçues pour renforcer
les systèmes, les réseaux, les logiciels, les appareils et les données
inestimables contre un paysage évolutif de menaces cybernétiques. Elle
englobe non seulement la protection, mais aussi la pratique visant à
garantir la confidentialité, l'intégrité et la disponibilité des
informations.

Au cœur de cette discipline, l'objectif est de réduire de manière
significative la vulnérabilitéaux intrusions cybernétiques, en assurant
des défenses solides contre tout accès non autorisé ou toute
exploitation potentielle des systèmes vitaux, des réseaux et des
fondements technologiques.

\textbf{2.1.2} \textbf{Cyberattaque}

Une cyberattaque se produit lorsqu'une entité malveillante, telle qu'un
hacker, une organisation criminelle, voire même un État-nation. visant à
perturber, endommager ou obtenir un accès non autorisé à un ordinateur,
un réseau ou un appareil. Ces attaques varient en complexité et
exploitent des vulnérabilités spécifiques pour atteindre leurs
objectifs. Chaque type d'attaque sert un but unique.

\emph{•} \textbf{Attaques par déni de service (Dos)} : Ces attaques
submergent un réseau avec un trafic massif qui le rend indisponible.
Cela peut entraîner des perturbations graves des services et des pertes
financières pour les entreprises.

\emph{•} \textbf{Attaques par ransomware} : Il s'agit de logiciels
malveillants qui chiffrent toutes les données disponibles et exigent le
paiement d'une rançon pour obtenir la clé de déchiffrement.

\emph{•} \textbf{Attaques de la chaîne d'approvisionnement} : Cette
tactique vise à exploiter les vulnérabilités chez les fournisseurs ou
sous-traitants tiers pour obtenir un accès non autorisé au réseau ou aux
systèmes de l'organisation ciblée. En compromettant une entité de
confiance dans
\end{quote}

\begin{longtable}[]{@{}
  >{\raggedright\arraybackslash}p{(\columnwidth - 2\tabcolsep) * \real{0.5000}}
  >{\raggedright\arraybackslash}p{(\columnwidth - 2\tabcolsep) * \real{0.5000}}@{}}
\toprule()
\begin{minipage}[b]{\linewidth}\raggedright
\begin{quote}
Dhia Abdelli
\end{quote}
\end{minipage} & \begin{minipage}[b]{\linewidth}\raggedright
Page 13
\end{minipage} \\
\midrule()
\endhead
\bottomrule()
\end{longtable}

\begin{quote}
ÉTAT DE L'ART

la chaîne d'approvisionnement, les attaquants peuvent contourner les
mesures de sécuritétraditionnelles, ce qui rend ce type d'attaque
particulièrement insidieux.

\emph{•} \textbf{Infections par logiciels malveillants} : Les logiciels
malveillants peuvent s'infiltrer dans les systèmes par divers moyens,
tels que les pièces jointes aux e-mails, les sites Web infectés ou les
clés USB. Une fois à l'intérieur d'un réseau, les logiciels malveillants
peuvent voler des informations sensibles, perturber les opérations ou
fournir un accès non autorisé aux attaquants.

\emph{•} \textbf{Attaques d'ingénierie sociale} : Ces attaques
manipulent la psychologie humaine pour obtenir un accès ou des
informations non autorisés. Les techniques peuvent inclure l'usurpation
d'identité, le prétexte ou l'appâtage. Les attaques d'ingénierie sociale
exploitent souvent la confiance ou l'autorité pour tromper les
individus.

\emph{•} \textbf{Attaques de l'homme du milieu} : Dans ce scénario, un
attaquant intercepte et potentiellement modifie la communication entre
deux parties. Cela peut se produire dans des environnements numériques
et physiques, constituant une menace significative pour les informations
confidentielles.

\textbf{2.1.3} \textbf{Le modèle triade CIA}

La Triade CIA représente un modèle de sécurité qui met l'accent sur la
confidentialité, l'intégrité et la disponibilité de l'information. Son
application s'avère essentielle pour une gestion sécurisée de
l'information dans de multiples domaines.{[}4{]}
\end{quote}

\includegraphics[width=2.51944in,height=2.47083in]{vertopal_fa5b2edd966b48bfbc6a830f7abfbc35/media/image10.png}

\textbf{FIGURE 2.1 -- Triade CIA}

\begin{longtable}[]{@{}
  >{\raggedright\arraybackslash}p{(\columnwidth - 2\tabcolsep) * \real{0.5000}}
  >{\raggedright\arraybackslash}p{(\columnwidth - 2\tabcolsep) * \real{0.5000}}@{}}
\toprule()
\begin{minipage}[b]{\linewidth}\raggedright
\begin{quote}
Dhia Abdelli
\end{quote}
\end{minipage} & \begin{minipage}[b]{\linewidth}\raggedright
Page 14
\end{minipage} \\
\midrule()
\endhead
\bottomrule()
\end{longtable}

\begin{quote}
ÉTAT DE L'ART

\emph{•} \textbf{Confidentialité} : garantir que les informations
sensibles sont accessibles exclusivement aux personnes autorisées. Ceci
est accompli grâce à des contrôles d'accès stricts, y compris des
systèmes de mots de passe robustes, formant une barrière imperméable
contre tout accès non autorisé.

\emph{•} \textbf{Intégrité} : Ce principe garantit la fiabilité et la
précision des données stockées, empêchant toute altération non
autorisée. Des techniques avancées telles que les empreintes digitales
et les signatures électroniques renforcent encore cet aspect critique.

\emph{•} \textbf{Accessibilité} : les informations doivent être
disponibles rapidement et facilement à tout moment et en tout lieu. Cela
garantit que les données remplissent leur objectif sans retard ni
obstruction.

\textbf{2.2 SOC}

Un Centre des Opérations de Sécurité, ou SOC, est comme le cerveau de la
sécurité d'une organisation. Il combine l'expertise de professionnels
avisés, des systèmes optimisés, et une technologie de pointe pour
constamment améliorer et renforcer la sécurité. Sa principale mission
est d'intercepter, localiser précisément, enquêter et gérer promptement
tout risque de cybermenace qui pourrait surgir.

\textbf{2.2.1} \textbf{Processus de réponse aux incidents dans un SOC}

Le SOC (Centre des Opérations de Sécurité) a établi un plan défini pour
faire face aux incidents de sécurité potentiels. Ce plan assure une
réponse rapide et efficace, protégeant ainsi les systèmes et les données
contre les menaces potentielles.

\emph{•} \textbf{1. Préparation} : Dans la phase de préparation, le SOC
établit un plan de réponse aux incidents complet qui définit des actions
et des responsabilités spécifiques. Une formation régulière garantit que
l'équipe maîtrise bien ses rôles. De plus, les équiper par des outils et
des ressources nécessaires pour une détection et une réponse rapides.
\end{quote}

\begin{longtable}[]{@{}
  >{\raggedright\arraybackslash}p{(\columnwidth - 2\tabcolsep) * \real{0.5000}}
  >{\raggedright\arraybackslash}p{(\columnwidth - 2\tabcolsep) * \real{0.5000}}@{}}
\toprule()
\begin{minipage}[b]{\linewidth}\raggedright
\begin{quote}
Dhia Abdelli
\end{quote}
\end{minipage} & \begin{minipage}[b]{\linewidth}\raggedright
Page 15
\end{minipage} \\
\midrule()
\endhead
\bottomrule()
\end{longtable}

\begin{quote}
ÉTAT DE L'ART

\emph{•} \textbf{2. Identification} : Pendant cette phase, le SOC
maintient une vigilance continue sur les activités du réseau et des
systèmes. Scrutent les alertes et les journaux avec précision,
recherchant toute déviation par rapport à la norme. Restant informés des
dernières informations sur les menaces provenant de sources fiables pour
renforcer leur perception de la situation.

\emph{•} \textbf{3. Endiguement} : Dans la phase de confinement, le SOC
met rapidement en place des mesures pour isoler les systèmes affectés,
freinant la propagation de l'incident. Renforcant leurs défenses en
appliquant des contrôles d'accès rigoureux et en appliquant rapidement
des correctifs ou des mises à jour pour corriger les vulnérabilités
connues.

\emph{•} \textbf{4. Éradication} : Cette phase se concentre sur le SOC
qui mène une enquête approfondie et systématique pour localiser et
éliminer la cause racine de l'incident. Fortifiant leur infrastructure
de sécurité en mettant en place des mesures supplémentaires pour
renforcer leurs défenses et prévenir les incidents futurs.

\emph{•} \textbf{5. Récupération} : Pendant la phase de récupération, le
SOC exécute un processus de récupération méticuleusement planifié,
restaurant méticuleusement les systèmes, les applications et les
services affectés à leur état opérationnel normal. Vérifiant
méthodiquement l'intégritédes données et des configurations, en
effectuant des vérifications complètes du système pour assurer un
fonctionnement optimal.

\emph{•} \textbf{6. Leçons apprises} : Le SOC se plonge dans les causes
sous-jacentes de l'incident, disséquant la séquence d'événements qui ont
conduit à son occurrence. Documentent ces précieuses observations, les
utilisant pour peaufiner et renforcer leur plan de réponse aux incidents
en vue d'une préparation future.

\textbf{2.2.2} \textbf{Défis SOC}

\emph{•} \textbf{Fatigue d'alerte} : Les analystes SOC peuvent être
inondés d'un volume élevé d'alertes, ce qui peut conduire à négliger des
menaces importantes.

\emph{•} \textbf{Pénurie de compétences} : Trouver et retenir des
professionnels qualifiés en cybersécuritépeut s'avérer un défi,
affectant l'efficacité du SOC.
\end{quote}

\begin{longtable}[]{@{}
  >{\raggedright\arraybackslash}p{(\columnwidth - 2\tabcolsep) * \real{0.5000}}
  >{\raggedright\arraybackslash}p{(\columnwidth - 2\tabcolsep) * \real{0.5000}}@{}}
\toprule()
\begin{minipage}[b]{\linewidth}\raggedright
\begin{quote}
Dhia Abdelli
\end{quote}
\end{minipage} & \begin{minipage}[b]{\linewidth}\raggedright
Page 16
\end{minipage} \\
\midrule()
\endhead
\bottomrule()
\end{longtable}

\begin{quote}
ÉTAT DE L'ART

\emph{•} \textbf{Menaces sophistiquées} : Les cybermenaces sont de plus
en plus avancées, nécessitant une adaptation constante et des défenses
mises à jour.

\emph{•} \textbf{Temps de réponse aux incidents} : Il est crucial
d'identifier, de contenir et d'atténuer rapidement les incidents, mais
cela peut être entravé par des retards ou des inefficacités.

\emph{•} \textbf{Faux positifs et négatifs} : Trier les alertes pour
différencier les fausses alarmes des menaces réelles peut s'avérer
difficile.

\emph{•} \textbf{Surcharge de données} : Gérer de grandes quantités de
données de sécurité générées par plusieurs sources peut s'avérer une
tâche ardue.

\textbf{2.3 SOAR}

Initialement formulé par Gartner, SOAR (Security Orchestration,
Automation, and Response) embrasse la fusion entre SOA (Security
Orchestration and Automation), TIP (Threat Intelligente platforms) et
IRP (Incident Response Platforms).Grâce à cette fusion, les
organisations peuvent gérer efficacement les menaces de sécurité et y
réagir rapidement, renforçant ainsi leur capacitédéfensive face à
l'évolution des cyber-risques.

Ce système harmonise diverses technologies avancées telles que
l'apprentissage automatique, l'intelligence artificielle et les flux de
travail automatisés pour affiner les opérations de sécuritéet accélérer
la réponse aux incidents.

Le SOAR intègre des outils et des technologies de sécurité dans une
plate-forme unique, tels que les pare-feu, les systèmes de détection
d'intrusion et les SIEM (systèmes de gestion des informations et des
événements de sécurité).

Cela donne aux équipes des opérations de sécurité une vue centralisée de
leur environnement, avec un accès à tout les outils et données
nécessaires pour détecter, enquêter et remédier aux menaces de sécurité.
\end{quote}

\begin{longtable}[]{@{}
  >{\raggedright\arraybackslash}p{(\columnwidth - 2\tabcolsep) * \real{0.5000}}
  >{\raggedright\arraybackslash}p{(\columnwidth - 2\tabcolsep) * \real{0.5000}}@{}}
\toprule()
\begin{minipage}[b]{\linewidth}\raggedright
\begin{quote}
Dhia Abdelli
\end{quote}
\end{minipage} & \begin{minipage}[b]{\linewidth}\raggedright
Page 17
\end{minipage} \\
\midrule()
\endhead
\bottomrule()
\end{longtable}

\begin{quote}
ÉTAT DE L'ART
\end{quote}

\begin{longtable}[]{@{}
  >{\raggedright\arraybackslash}p{(\columnwidth - 2\tabcolsep) * \real{0.5000}}
  >{\raggedright\arraybackslash}p{(\columnwidth - 2\tabcolsep) * \real{0.5000}}@{}}
\toprule()
\begin{minipage}[b]{\linewidth}\raggedright
\begin{quote}
\textbf{2.3.1}
\end{quote}
\end{minipage} & \begin{minipage}[b]{\linewidth}\raggedright
\begin{quote}
\textbf{Que signifie "SOAR"?}
\end{quote}
\end{minipage} \\
\midrule()
\endhead
\bottomrule()
\end{longtable}

\begin{quote}
SOAR, abréviation de Security Orchestration, Automation and Response,
est une stratégie de cybersécurité moderne qui combine des techniques
d'orchestration, d'automatisation et de réponse rapide. Il aide les
organisations à renforcer leurs défenses contre un large éventail de
menaces.. En mettant en place SOAR, on peut rapidement et précisément
réagir face aux incidents, ce qui en fait un outil inestimable dans le
paysage numérique actuel.{[}5{]}

\emph{•} \textbf{Orchestration} : L'orchestration rationalise la
collaboration de divers outils et systèmes de sécurité, renforçant ainsi
les mesures de sécurité d'une organisation. Il fournit un aperçu clair
de la configuration de sécurité, permettant une détection et une réponse
rapides aux menaces. Cette vue unifiée permet aux organisations
d'identifier et de traiter rapidement les incidents de sécurité.

\emph{•} \textbf{Automatisation} : L'automatisation consiste à utiliser
la technologie pour simplifier et automatiser les tâches de sécurité
courantes, telles que la collecte de données, l'analyse et
l'établissement de rapports. Les organisations peuvent en bénéficier en
réduisant les erreurs humaines et en augmentant la rapidité et la
précision de la détection et de la réponse aux menaces.

\emph{•} \textbf{Réponse} : La réponse est la réaction rapide et
efficace aux événements et menaces de sécurité. La mise en place d'un
système SOAR permet aux organisations de détecter rapidement les
incidents de sécurité et de les organiser en fonction de leur
importance. Il avertit automatiquement les équipes de sécurité et leur
fournit les informations essentielles nécessaires pour intervenir en cas
d'incident.

\textbf{2.3.2} \textbf{Pourquoi SOAR est-il important?}

SOAR joue un rôle central dans le domaine de la cybersécurité pour une
multitude de raisons impérieuses.

\emph{•} \textbf{Efficacité} : SOAR renforce principalement l'efficacité
des opérations de sécurité en automatisant les tâches répétitives et en
orchestrant les processus et les outils de sécurité. Il permet aux
\end{quote}

\begin{longtable}[]{@{}
  >{\raggedright\arraybackslash}p{(\columnwidth - 2\tabcolsep) * \real{0.5000}}
  >{\raggedright\arraybackslash}p{(\columnwidth - 2\tabcolsep) * \real{0.5000}}@{}}
\toprule()
\begin{minipage}[b]{\linewidth}\raggedright
\begin{quote}
Dhia Abdelli
\end{quote}
\end{minipage} & \begin{minipage}[b]{\linewidth}\raggedright
Page 18
\end{minipage} \\
\midrule()
\endhead
\bottomrule()
\end{longtable}

\begin{quote}
ÉTAT DE L'ART

analystes de la sécurité de concentrer leurs compétences sur des tâches
plus complexes et critiques qui nécessitent une prise de décision
humaine.

\emph{•} \textbf{Vitesse} : SOAR constitue une force formidable pour
réduire les délais de réponse aux incidents de sécurité. En effectuant
rapidement des interventions automatisées et orchestrées en temps réel,
cela réduit considérablement le risque d'un incident de sécurité causant
des dommages importants.

\emph{•} \textbf{Cohérence} : SOAR garantit que les processus de
sécurité sont exécutées sans failles. L'automatisation et
l'orchestration transparentes des flux de travail en matière de
sécuritépermettent d'éliminer les potentielles incohérences résultant
d'une erreur humaine ou d'une formation inadéquate dans le cadre des
opérations de sécurité.

\emph{•} \textbf{Évolutivité} : SOAR permet aux opérations de sécurité
d'élargir leurs horizons avec une efficacité inégalée. Face aux volumes
croissants et aux menaces de sécurité complexes, SOAR offre aux équipes
de sécurité la finesse nécessaire pour relever ces défis avec expertise
et sans avoir à recourir à une main-d'œuvre supplémentaire ou à une
allocation de ressources importantes.

\emph{•} \textbf{Intelligence améliorée sur les menaces} :Les
plateformes de renseignement sur les menaces peut être améliorées grâce
à SOAR en matière de précision et de qualité. En agrégeant et corrélant
de manière transparente des données provenant d'une variété d'outils et
de sources de sécurité, cela habilite les équipes de sécurité avec une
perspective globale et affinée sur les incidents de sécurité,
établissant ainsi une nouvelle norme en matière d'excellence
opérationnelle. .

\textbf{2.3.3} \textbf{Composants SOAR}

Pour atteindre une efficacité et une efficience optimales lors de
l'exécution des tâches, SOAR comporte cinq composants essentiels qui
fonctionnent en synergie :

\emph{•} Tableau de bord unifié\\
\emph{•} Playbooks\\
\emph{•} Gestion des renseignements sur les menaces
\end{quote}

\begin{longtable}[]{@{}
  >{\raggedright\arraybackslash}p{(\columnwidth - 2\tabcolsep) * \real{0.5000}}
  >{\raggedright\arraybackslash}p{(\columnwidth - 2\tabcolsep) * \real{0.5000}}@{}}
\toprule()
\begin{minipage}[b]{\linewidth}\raggedright
\begin{quote}
Dhia Abdelli
\end{quote}
\end{minipage} & \begin{minipage}[b]{\linewidth}\raggedright
Page 19
\end{minipage} \\
\midrule()
\endhead
\bottomrule()
\end{longtable}

\begin{quote}
ÉTAT DE L'ART

\emph{•} Gestion de cas\\
\emph{•} Moteur d'orchestration et d'automatisation

\textbf{2.3.3.1} \textbf{Tableau de bord unifié}

Le tableau de bord SOAR, souvent appelé interface centralisée ou
unifiée, rassemble de nombreuses informations importantes provenant de
différents endroits afin que les opérateurs de sécurité puissent tout
voir au même endroit.

Grâce à ce tableau de bord, les opérateurs peuvent recueillir des
informations de sécurité à la minute près et surveiller de près les
principaux signes de problèmes spécifiques aux problèmes de sécurité.
Cela leur donne une très bonne compréhension du niveau de sécurité
global de leur organisation. Le tableau de bord combine intelligemment
les données de différents services et systèmes de sécurité, pour que
tout semble clair et organisé.

Les avantages d'un tableau de bord SOAR sont énormes. Il permet de voir
facilement ce qui se passe avec les incidents de sécurité, aide à
prendre des décisions rapides concernant les menaces, rend la gestion
des problèmes de sécurité plus efficace et rend l'ensemble du processus
plus fluide. Les opérateurs peuvent rapidement repérer les nouvelles
tendances en matière de menaces de sécurité, remarquer tout ce qui est
étrange et prendre des décisions basées sur les faits affichés dans le
tableau de bord. Cela signifie qu'ils n'ont pas besoin de gérer de
nombreux systèmes de sécurité différents ni de lire de nombreux
rapports, ce qui permet d'économiser beaucoup de temps et de travail.

\textbf{2.3.3.2} \textbf{Playbooks}

Les playbooks sont comme des guides fiables pour l'orchestration,
l'automatisation et la réponse en matière de sécurité. Ils offrent des
instructions claires, étape par étape, pour traiter les différents
problèmes de cybersécurité. Ils aident les équipes de sécurité à réagir
rapidement et efficacement pour atténuer les menaces.

Chaque playbook est une solution personnalisée à un problème spécifique.
Il fournit des conseils simples sur la manière de gérer cette situation
particulière et il combine à la fois des
\end{quote}

\begin{longtable}[]{@{}
  >{\raggedright\arraybackslash}p{(\columnwidth - 2\tabcolsep) * \real{0.5000}}
  >{\raggedright\arraybackslash}p{(\columnwidth - 2\tabcolsep) * \real{0.5000}}@{}}
\toprule()
\begin{minipage}[b]{\linewidth}\raggedright
\begin{quote}
Dhia Abdelli
\end{quote}
\end{minipage} & \begin{minipage}[b]{\linewidth}\raggedright
Page 20
\end{minipage} \\
\midrule()
\endhead
\bottomrule()
\end{longtable}

\begin{quote}
ÉTAT DE L'ART

actions manuelles et des processus automatisés, tout en permettant aux
individus d'intervenir au besoin. Cela garantit que les réponses peuvent
s'adapter rapidement aux différents types de menaces.

\textbf{2.3.3.3} \textbf{Gestion des renseignements sur les menaces}

La gestion du renseignement sur les menaces implique la collecte,
l'analyse et l'enrichissement des données relatives aux risques de
sécurité, qui englobent les cybers menaces telles que les logiciels
malveillants et le phishing. Il s'agit d'un élément essentiel d'une
stratégie de sécuritésolide, qui permet aux organisations d'être
informées des menaces potentielles et de mettre en place des défenses
proactives et de meilleures stratégies de réponse.

Ce processus comprend la collecte d'informations telles que les adresses
IP, les hachages de fichiers, les URLs et d'autres données pertinentes
provenant de diverses sources, y compris les renseignements Open-source,
les bases de données propriétaires et même les réseaux sociaux. Les
renseignements exploitables sont partagés avec les parties concernées,
comme l'équipe de sécurité, pour la mise en œuvre de contre-mesures
appropriées.

\textbf{2.3.3.4} \textbf{Gestion de cas}

La gestion des cas consiste à superviser les incidents de sécurité du
début à la fin. Elle utilise un système central pour enregistrer toutes
les informations importantes relatives à un problème de sécurité en
veillant à ce que tout soit organisé. Elle détermine les menaces les
plus graves. Selon l'importance de celles-ci, le système SOAR peut les
traiter automatiquement ou demander l'aide d'analystes humains. Une
bonne gestion des cas permet également d'obtenir plus d'informations,
comme des alertes antérieures similaires.

\textbf{2.3.3.5} \textbf{Moteur d'orchestration et d'automatisation}

L'orchestration et l'automatisation sont des fonctionnalités
complémentaires qui agissent en synergie pour améliorer l'efficacité
opérationnelle.
\end{quote}

\begin{longtable}[]{@{}
  >{\raggedright\arraybackslash}p{(\columnwidth - 2\tabcolsep) * \real{0.5000}}
  >{\raggedright\arraybackslash}p{(\columnwidth - 2\tabcolsep) * \real{0.5000}}@{}}
\toprule()
\begin{minipage}[b]{\linewidth}\raggedright
\begin{quote}
Dhia Abdelli
\end{quote}
\end{minipage} & \begin{minipage}[b]{\linewidth}\raggedright
Page 21
\end{minipage} \\
\midrule()
\endhead
\bottomrule()
\end{longtable}

\begin{quote}
ÉTAT DE L'ART

\emph{•} \textbf{L'orchestration} consiste à séquencer intelligemment
les tâches, à diriger le flux sur la base de connexions logiques et à
proposer des itinéraires alternatifs pour atteindre les objectifs
opérationnels.

\emph{•} L'automatisation s'articule autour de la compréhension du
contexte, de l'exécution de mesures correctives et, surtout, de la
rationalisation des flux de travail conçus par les analystes à l'aide de
playbooks, éliminant ainsi le besoin d'interventions manuelles qui
entravent l'efficacité de l'orchestration. Les déclencheurs sont
utilisés pour lancer le moteur d'automatisation, conçu pour anticiper
l'événement pertinent.

\textbf{2.3.4} \textbf{Solution SOAR dans un SOC}

Un centre d'opération de sécurité (SOC) conventionnel s'appuie souvent
sur des procédures manuelles et une surveillance humaine pour
identifier, examiner et traiter les failles de sécurité.À l'inverse, un
SOC doté d'une plateforme d'orchestration, d'automatisation et de
réponse de sécurité (SOAR) possède la capacité de mécaniser une partie
substantielle de ces tâches. Cela conduit à des améliorations notables
de la vitesse de réponse aux incidents, à une efficacitéopérationnelle
accrue et à une coopération interministérielle renforcée.

\emph{•} L'intégration de la technologie SOAR au sein d'une équipe SOC
conduit à des réductions potentielles des délais de réponse aux
incidents d'environ 60\%.

\emph{•} L'intégration de la technologie SOAR au sein d'une équipe SOC
peut conduire à une réduction de 30 à 50\% du temps d'enquête et de
résolution des incidents..

\emph{•} L'intégration de la technologie SOAR au sein d'une équipe SOC
peut potentiellement réduire jusqu'à 50\% l'apparition de faux positifs.

\emph{•} L'intégration de la technologie SOAR au sein d'une équipe SOC
peut potentiellement entraîner une réduction de 25\% des coûts de
gestion des incidents de sécurité..
\end{quote}

\begin{longtable}[]{@{}
  >{\raggedright\arraybackslash}p{(\columnwidth - 2\tabcolsep) * \real{0.5000}}
  >{\raggedright\arraybackslash}p{(\columnwidth - 2\tabcolsep) * \real{0.5000}}@{}}
\toprule()
\begin{minipage}[b]{\linewidth}\raggedright
\begin{quote}
Dhia Abdelli
\end{quote}
\end{minipage} & \begin{minipage}[b]{\linewidth}\raggedright
Page 22
\end{minipage} \\
\midrule()
\endhead
\bottomrule()
\end{longtable}

\begin{quote}
ÉTAT DE L'ART

\textbf{2.4 Solutions existantes}

Aujourd'hui, les solutions SOAR sont essentiells pour automatiser la
sécurité et renforcer la réponse aux incidents. Évalué à 1,1 milliard de
dollars en 2022,Le marché du SOAR devrait atteindre 2,3 milliards de
dollars d'ici 2027 selon marketandmarkets {[}6{]} . le marché SOAR
affiche une croissance constante, portée par l'innovation et les
nouvelles technologies.

Les 6 meilleures solutions SOAR sur le marché pour 2022 sont les
suivantes :

1. \textbf{Splunk Phantom} est connu pour ses capacités avancées
d'automatisation et d'orchestration. Il excelle dans la fourniture d'une
large gamme d'intégrations avec divers outils de sécurité, permettant
des réponses automatisées et transparentes aux menaces. De plus, il
propose des playbooks robustes pour les flux de travail de réponse aux
incidents.

2. \textbf{IBM Resilient} propose une plateforme SOAR complète avec de
puissantes capacités d'automatisation, d'orchestration et de réponse. Il
se distingue par sa capacité à détecter, enquêter et réagir efficacement
aux incidents. Son tableau de bord centralisé fournit des informations
en temps réel sur les opérations de sécurité.

3. \textbf{Palo Alto Networks Cortex XSOAR} anciennement connu sous le
nom de Demisto, fournit une plateforme SOAR hautement personnalisable.
Il excelle dans la rationalisation de la réponse aux incidents grâce à
des playbooks automatisés et à des intégrations avec un large éventail
d'outils de sécurité. Son éditeur visuel de playbook simplifie la
création de flux de travail complexes.

4. \textbf{Siemplify} est reconnu pour son approche holistique de SOAR,
offrant des capacités avancées d'automatisation et d'orchestration. Il
se distingue par une interface conviviale qui centralise les alertes,
les incidents et les enquêtes. Ses tableaux de bord personnalisables
améliorent la visibilité et la collaboration entre les équipes de
sécurité\\
5. \textbf{Swimlane} est connu pour son éditeur visuel de flux de
travail, facilitant la conception et la mise en œuvre de flux de travail
de sécurité complexes. Il excelle dans l'automatisation et
l'orchestration des processus de sécurité, fournissant une plateforme
centrale pour gérer les alertes et les incidents.

6. \textbf{Fortinet FortiSOAR} se distingue par son intégration avec les
solutions de sécurité de Fortinet, améliorant ainsi la posture de
sécurité globale. Il offre des capacités avancées
\end{quote}

\begin{longtable}[]{@{}
  >{\raggedright\arraybackslash}p{(\columnwidth - 2\tabcolsep) * \real{0.5000}}
  >{\raggedright\arraybackslash}p{(\columnwidth - 2\tabcolsep) * \real{0.5000}}@{}}
\toprule()
\begin{minipage}[b]{\linewidth}\raggedright
\begin{quote}
Dhia Abdelli
\end{quote}
\end{minipage} & \begin{minipage}[b]{\linewidth}\raggedright
Page 23
\end{minipage} \\
\midrule()
\endhead
\bottomrule()
\end{longtable}

\begin{quote}
ÉTAT DE L'ART

d'automatisation et d'orchestration, rationalisant les opérations de
sécurité et la réponse aux incidents.

\textbf{2.5 Choix des outils}

\textbf{2.5.1} \textbf{SIEM (Elastic Stack)}

La Suite Elastic est une équipe d'outils open source spécialement conçus
pour gérer les données : Elasticsearch, Logstash, Kibana et Beats.
Elasticsearch agit comme un moteur de recherche exceptionnellement
puissant, capable de localiser et analyser divers types d'informations.

Logstash fait office d'organisateur de données, rassemblant et préparant
des informations en provenance de différentes sources avant de les
acheminer vers leur destination, tout comme Elasticsearch. Kibana sert
de tableau de bord, offrant une vue claire de vos données à travers des
graphiques et des diagrammes, facilitant ainsi leur compréhension. Quant
à Beats, ils jouent le rôle de messagers, explorant différents
ordinateurs et systèmes, collectant des données et les transmettant au
reste de l'équipe. Ensemble, ils forment une équipe solide pour gérer et
comprendre un large éventail d'informations.

\includegraphics[width=3.77917in,height=0.78333in]{vertopal_fa5b2edd966b48bfbc6a830f7abfbc35/media/image11.png}
\end{quote}

\textbf{FIGURE 2.2 -- Logo de Elastic Stack}

\begin{quote}
\textbf{2.5.2} \textbf{Outil de renseignement sur les menaces (MISP)}

MISP est une plateforme avancée de renseignement sur les menaces qui
transforme radicalement la manière dont les informations sont partagées
au sein d'une communauté de membres de confiance. Des indicateurs de
compromission (IoC) aux informations détaillées sur les logiciels
malveillants et les attaques ciblées, MISP propose un spectre complet.
Grâce à son infrastructure unique, MISP accélère non seulement la
détection des attaques et minimise les fausses alarmes,
\end{quote}

\begin{longtable}[]{@{}
  >{\raggedright\arraybackslash}p{(\columnwidth - 2\tabcolsep) * \real{0.5000}}
  >{\raggedright\arraybackslash}p{(\columnwidth - 2\tabcolsep) * \real{0.5000}}@{}}
\toprule()
\begin{minipage}[b]{\linewidth}\raggedright
\begin{quote}
Dhia Abdelli
\end{quote}
\end{minipage} & \begin{minipage}[b]{\linewidth}\raggedright
Page 24
\end{minipage} \\
\midrule()
\endhead
\bottomrule()
\end{longtable}

\begin{quote}
ÉTAT DE L'ART

mais renforce également les défenses et les contre-mesures contre les
menaces émergentes. Cela permet aux organisations de maintenir une
position proactive dans le paysage en constanteévolution de la
cybersécurité.
\end{quote}

\includegraphics[width=1.89028in,height=1.38472in]{vertopal_fa5b2edd966b48bfbc6a830f7abfbc35/media/image12.png}

\textbf{FIGURE 2.3 -- Logo de Misp}

\begin{quote}
\textbf{2.5.3} \textbf{Solution de réponse aux incidents de sécurité
(DFIR IRIS)}

DFIR IRIS est une solution open source révolutionnaire de réponse aux
incidents de sécuritéconçue pour simplifier le partage de détails
techniques entre les intervenants en cas d'incident, favorisant la
collaboration dans les enquêtes. Il excelle à organiser et rationaliser
les tâches, en respectant les flux de travail variés de chaque équipe
sans imposer de règles strictes. De plus, il sert de plateforme efficace
de gestion des alertes, guider une alerte incident depuis sa création
initiale jusqu'à sa résolution finale.
\end{quote}

\includegraphics[width=2.51944in,height=0.84583in]{vertopal_fa5b2edd966b48bfbc6a830f7abfbc35/media/image13.png}

\textbf{FIGURE 2.4 -- Logo de DFIR IRIS}

\begin{quote}
\textbf{2.5.4} \textbf{Outil d'orchestration (Shuffle)}

Shuffle révolutionne les opérations de sécurité grâce à sa plateforme
SOAR qui orchestre et automatise de manière transparente l'ensemble du
processus de traitement des incidents. Ce système de pointe repose sur
un cadre d'applications, d'intégrations, de déclencheurs et de flux de
travail. En outre, Shuffle est profondément ancré dans un écosystème
Open source, englobant
\end{quote}

\begin{longtable}[]{@{}
  >{\raggedright\arraybackslash}p{(\columnwidth - 2\tabcolsep) * \real{0.5000}}
  >{\raggedright\arraybackslash}p{(\columnwidth - 2\tabcolsep) * \real{0.5000}}@{}}
\toprule()
\begin{minipage}[b]{\linewidth}\raggedright
\begin{quote}
Dhia Abdelli
\end{quote}
\end{minipage} & \begin{minipage}[b]{\linewidth}\raggedright
Page 25
\end{minipage} \\
\midrule()
\endhead
\bottomrule()
\end{longtable}

\begin{quote}
ÉTAT DE L'ART

un large éventail de flux de travail, d'applications et de normes
industrielles. Cette plateforme offre une approche unique de SOAR, avec
une mission claire : faciliter le flux de données àtravers les
entreprises en utilisant des applications plug-and-play facilement
adaptables.
\end{quote}

\includegraphics[width=2.51944in,height=1.25972in]{vertopal_fa5b2edd966b48bfbc6a830f7abfbc35/media/image14.png}

\textbf{FIGURE 2.5 -- Logo de Shuffle}

\begin{quote}
\textbf{2.5.5} \textbf{Plateforme Honeypot (T-POT)}

T-Pot est un puissant framework de Honeypot open source conçu à des fins
de recherche et de sécurité. Essentiellement, il agit comme un système
leurre ou un réseau mis en place pour attirer des attaquants potentiels,
permettant de collecter des informations vitales sur leurs tactiques et
techniques.

L'une des forces de T-Pot réside dans sa capacité à émuler avec
précision divers services et systèmes couramment rencontrés dans les
configurations informatiques des organisations. Ce faisant, il collecte
efficacement des données sur les menaces et attaques potentielles. Ces
informations précieuses fournissent aux professionnels de la sécurité
les informations nécessaires pour analyser et comprendre le comportement
des attaquants.
\end{quote}

\includegraphics[width=1.25972in,height=1.25972in]{vertopal_fa5b2edd966b48bfbc6a830f7abfbc35/media/image15.png}

\textbf{FIGURE 2.6 -- Logo de T-pot}

\begin{longtable}[]{@{}
  >{\raggedright\arraybackslash}p{(\columnwidth - 2\tabcolsep) * \real{0.5000}}
  >{\raggedright\arraybackslash}p{(\columnwidth - 2\tabcolsep) * \real{0.5000}}@{}}
\toprule()
\begin{minipage}[b]{\linewidth}\raggedright
\begin{quote}
Dhia Abdelli
\end{quote}
\end{minipage} & \begin{minipage}[b]{\linewidth}\raggedright
Page 26
\end{minipage} \\
\midrule()
\endhead
\bottomrule()
\end{longtable}

\begin{quote}
ÉTAT DE L'ART
\end{quote}

\begin{longtable}[]{@{}
  >{\raggedright\arraybackslash}p{(\columnwidth - 2\tabcolsep) * \real{0.5000}}
  >{\raggedright\arraybackslash}p{(\columnwidth - 2\tabcolsep) * \real{0.5000}}@{}}
\toprule()
\begin{minipage}[b]{\linewidth}\raggedright
\begin{quote}
\textbf{2.5.6}
\end{quote}
\end{minipage} & \begin{minipage}[b]{\linewidth}\raggedright
\begin{quote}
\textbf{Outil de collecte de données T-POT GreedyBear (GreedyBear)}
\end{quote}
\end{minipage} \\
\midrule()
\endhead
\bottomrule()
\end{longtable}

\begin{quote}
Greedybear vise à extraire des données précises des attaques
enregistrées sur un honeypot T-POT ou un cluster de honeypots T-POT. Ces
données seront ensuite utilisées pour générer des flux qui jouent un
rôle vital dans le renforcement des mécanismes de prévention et de
détection des attaques. Par la suite, ces flux seront parfaitement
intégré à l'outil Threat Intelligence.
\end{quote}

\includegraphics[width=1.25972in,height=1.45417in]{vertopal_fa5b2edd966b48bfbc6a830f7abfbc35/media/image16.png}

\textbf{FIGURE 2.7 -- Logo de GreedyBear}

\begin{quote}
\textbf{2.5.7} \textbf{Outil d'analyse des logiciels malveillants
(Cuckoo Sanbox)}

Cuckoo Sandbox est un excellent outil de cybersécurité conçu pour lancer
des logiciels malveillants dans un environnement isolé. Le concept
central est basé sur sa capacité à tromper le malware en lui faisant
croire qu'il a réussi à infiltrer un hôte légitime. Ensuite, Cuckoo
Sandbox surveille méticuleusement les actions du malware et compile un
rapport complet décrivant en détail chaque effort entrepris dans cet
environnement contrôlé. Cela s'avère inestimable pour les équipes de
sécurité et les analystes de logiciels malveillants, offrant un accès
aux indicateurs de compromission "IOC" essentielle pour répondre aux
incidents de sécurité ou servant de point de base pour la collecte de
renseignements. Il fournit des informations rapides et complexes sur le
comportement anticipé du logiciel malveillant.
\end{quote}

\includegraphics[width=2.52083in,height=1.24028in]{vertopal_fa5b2edd966b48bfbc6a830f7abfbc35/media/image17.png}

\textbf{FIGURE 2.8 -- Logo Cuckoo Sanbox}

\begin{longtable}[]{@{}
  >{\raggedright\arraybackslash}p{(\columnwidth - 2\tabcolsep) * \real{0.5000}}
  >{\raggedright\arraybackslash}p{(\columnwidth - 2\tabcolsep) * \real{0.5000}}@{}}
\toprule()
\begin{minipage}[b]{\linewidth}\raggedright
\begin{quote}
Dhia Abdelli
\end{quote}
\end{minipage} & \begin{minipage}[b]{\linewidth}\raggedright
Page 27
\end{minipage} \\
\midrule()
\endhead
\bottomrule()
\end{longtable}

\begin{quote}
ÉTAT DE L'ART
\end{quote}

\begin{longtable}[]{@{}
  >{\raggedright\arraybackslash}p{(\columnwidth - 2\tabcolsep) * \real{0.5000}}
  >{\raggedright\arraybackslash}p{(\columnwidth - 2\tabcolsep) * \real{0.5000}}@{}}
\toprule()
\begin{minipage}[b]{\linewidth}\raggedright
\begin{quote}
\textbf{2.5.8}
\end{quote}
\end{minipage} & \begin{minipage}[b]{\linewidth}\raggedright
\begin{quote}
\textbf{Outil d'alerte (Praeco)}
\end{quote}
\end{minipage} \\
\midrule()
\endhead
\bottomrule()
\end{longtable}

\begin{quote}
Praeco est un outil d'alerte open-source robuste conçu pour les
utilisateurs d'Elasticsearch. Il fonctionne comme une interface
graphique intuitive pour ElastAlert, simplifiant la création d'alertes
grâce à un générateur de requêtes convivial. Cela permet aux
utilisateurs de surveiller de manière proactive les données
Elasticsearch. Notamment, Praeco excelle dans les notifications. Il
envoie des alertes de manière transparente via Slack, Email ou HTTP
POST, garantissant une livraison rapide aux bons destinataires.
\end{quote}

\includegraphics[width=0.62917in,height=1.28472in]{vertopal_fa5b2edd966b48bfbc6a830f7abfbc35/media/image18.png}

\textbf{FIGURE 2.9 -- Logo Praeco}

\begin{quote}
\textbf{2.5.9} \textbf{Outil d'investigation numérique et de réponse aux
incidents (Velociraptor)}

Velociraptor est une solution open-source avancée et distinguée, qui
intègre des capacités de surveillance des points finaux, de
criminalistique numérique et de cyber-réponse. Cet outil vous permet
d'améliorer votre réponse à un large éventail d'enquêtes judiciaires
numériques et d'incidents cybernétiques, y compris les violations de
données. Velociraptor vous permetégalement de mener une collecte précise
et rapide. En exploitant les fonctions sophistiquées de Velociraptor,
vous obtenez des informations extraordinaires sur les activités des
terminaux, ce qui vous permet d'apporter des réponses plus efficaces et
de mettre en place des stratégies robustes de résolution des incidents.
\end{quote}

\includegraphics[width=2.51944in,height=1.02639in]{vertopal_fa5b2edd966b48bfbc6a830f7abfbc35/media/image19.png}

\textbf{FIGURE 2.10 -- Logo velociraptor}

\begin{longtable}[]{@{}
  >{\raggedright\arraybackslash}p{(\columnwidth - 2\tabcolsep) * \real{0.5000}}
  >{\raggedright\arraybackslash}p{(\columnwidth - 2\tabcolsep) * \real{0.5000}}@{}}
\toprule()
\begin{minipage}[b]{\linewidth}\raggedright
\begin{quote}
Dhia Abdelli
\end{quote}
\end{minipage} & \begin{minipage}[b]{\linewidth}\raggedright
Page 28
\end{minipage} \\
\midrule()
\endhead
\bottomrule()
\end{longtable}

\begin{quote}
ÉTAT DE L'ART

\textbf{2.6 CONCLUSION}

Dans ce chapitre, nous avons approfondi SOAR, en mettant l'accent sur
ses composants et ses capacités. Nous avons démontré son efficacité pour
améliorer la réponse aux incidents au sein d'un SOC. De plus, nous avons
mis en évidence les solutions choisies qui ont créénotre plateforme.
Passons maintenant au chapitre suivant, où nous préciserons les
exigences et identifierons les acteurs clés.
\end{quote}

\begin{longtable}[]{@{}
  >{\raggedright\arraybackslash}p{(\columnwidth - 2\tabcolsep) * \real{0.5000}}
  >{\raggedright\arraybackslash}p{(\columnwidth - 2\tabcolsep) * \real{0.5000}}@{}}
\toprule()
\begin{minipage}[b]{\linewidth}\raggedright
\begin{quote}
Dhia Abdelli
\end{quote}
\end{minipage} & \begin{minipage}[b]{\linewidth}\raggedright
Page 29
\end{minipage} \\
\midrule()
\endhead
\bottomrule()
\end{longtable}

Chapitre

\begin{longtable}[]{@{}
  >{\raggedright\arraybackslash}p{(\columnwidth - 2\tabcolsep) * \real{0.5000}}
  >{\raggedright\arraybackslash}p{(\columnwidth - 2\tabcolsep) * \real{0.5000}}@{}}
\toprule()
\begin{minipage}[b]{\linewidth}\raggedright
\begin{longtable}[]{@{}
  >{\raggedright\arraybackslash}p{(\columnwidth - 0\tabcolsep) * \real{1.0000}}@{}}
\toprule()
\begin{minipage}[b]{\linewidth}\raggedright
\textbf{3}
\end{minipage} \\
\midrule()
\endhead
\bottomrule()
\end{longtable}
\end{minipage} & \begin{minipage}[b]{\linewidth}\raggedright
\begin{quote}
\textbf{Analyse et conception}
\end{quote}
\end{minipage} \\
\midrule()
\endhead
\bottomrule()
\end{longtable}

\begin{quote}
\textbf{Sommaire}
\end{quote}

\begin{longtable}[]{@{}
  >{\raggedright\arraybackslash}p{(\columnwidth - 12\tabcolsep) * \real{0.1429}}
  >{\raggedright\arraybackslash}p{(\columnwidth - 12\tabcolsep) * \real{0.1429}}
  >{\raggedright\arraybackslash}p{(\columnwidth - 12\tabcolsep) * \real{0.1429}}
  >{\raggedright\arraybackslash}p{(\columnwidth - 12\tabcolsep) * \real{0.1429}}
  >{\raggedright\arraybackslash}p{(\columnwidth - 12\tabcolsep) * \real{0.1429}}
  >{\raggedright\arraybackslash}p{(\columnwidth - 12\tabcolsep) * \real{0.1429}}
  >{\raggedright\arraybackslash}p{(\columnwidth - 12\tabcolsep) * \real{0.1429}}@{}}
\toprule()
\begin{minipage}[b]{\linewidth}\raggedright
\textbf{3.1}
\end{minipage} &
\multicolumn{5}{>{\raggedright\arraybackslash}p{(\columnwidth - 12\tabcolsep) * \real{0.7143} + 8\tabcolsep}}{%
\begin{minipage}[b]{\linewidth}\raggedright
\textbf{Outil de modélisation . . . . . . . . . . . . . . . . . . . . .
. . .}
\end{minipage}} & \begin{minipage}[b]{\linewidth}\raggedright
\begin{quote}
\textbf{31}
\end{quote}
\end{minipage} \\
\midrule()
\endhead
\textbf{3.2} &
\multicolumn{5}{>{\raggedright\arraybackslash}p{(\columnwidth - 12\tabcolsep) * \real{0.7143} + 8\tabcolsep}}{%
\textbf{Identifier les acteurs . . . . . . . . . . . . . . . . . . . . .
. . . .}} & \begin{minipage}[t]{\linewidth}\raggedright
\begin{quote}
\textbf{31}
\end{quote}
\end{minipage} \\
\textbf{3.3} &
\multicolumn{5}{>{\raggedright\arraybackslash}p{(\columnwidth - 12\tabcolsep) * \real{0.7143} + 8\tabcolsep}}{%
\textbf{Spécification des besoins . . . . . . . . . . . . . . . . . . .
. . .}} & \begin{minipage}[t]{\linewidth}\raggedright
\begin{quote}
\textbf{32}
\end{quote}
\end{minipage} \\
\multirow{3}{*}{\textbf{3.4}} & 3.3.1 &
\multicolumn{4}{>{\raggedright\arraybackslash}p{(\columnwidth - 12\tabcolsep) * \real{0.5714} + 6\tabcolsep}}{%
Exigences fonctionnelles . . . . . . . . . . . . . . . . . . . . . .} &
\begin{minipage}[t]{\linewidth}\raggedright
\begin{quote}
32
\end{quote}
\end{minipage} \\
& 3.3.2 &
\multicolumn{4}{>{\raggedright\arraybackslash}p{(\columnwidth - 12\tabcolsep) * \real{0.5714} + 6\tabcolsep}}{%
Exigences non fonctionnelles . . . . . . . . . . . . . . . . . . . .} &
\begin{minipage}[t]{\linewidth}\raggedright
\begin{quote}
33
\end{quote}
\end{minipage} \\
&
\multicolumn{2}{>{\raggedright\arraybackslash}p{(\columnwidth - 12\tabcolsep) * \real{0.2857} + 2\tabcolsep}}{%
\begin{minipage}[t]{\linewidth}\raggedright
\begin{quote}
\textbf{Conception}
\end{quote}
\end{minipage}} &
\multicolumn{3}{>{\raggedright\arraybackslash}p{(\columnwidth - 12\tabcolsep) * \real{0.4286} + 4\tabcolsep}}{%
\textbf{. . . . . . . . . . . . . . . . . . . . . . . . . . . . . .}} &
\begin{minipage}[t]{\linewidth}\raggedright
\begin{quote}
\textbf{33}
\end{quote}
\end{minipage} \\
\multirow{3}{*}{\textbf{3.5}} & 3.4.1 &
\multicolumn{3}{>{\raggedright\arraybackslash}p{(\columnwidth - 12\tabcolsep) * \real{0.4286} + 4\tabcolsep}}{%
Diagramme de cas d'utilisation global} & . . . . . . . . . . . . . . &
\begin{minipage}[t]{\linewidth}\raggedright
\begin{quote}
34
\end{quote}
\end{minipage} \\
& 3.4.2 &
\multicolumn{4}{>{\raggedright\arraybackslash}p{(\columnwidth - 12\tabcolsep) * \real{0.5714} + 6\tabcolsep}}{%
Description détaillée du cas . . . . . . . . . . . . . . . . . . . .} &
\begin{minipage}[t]{\linewidth}\raggedright
\begin{quote}
35
\end{quote}
\end{minipage} \\
&
\multicolumn{3}{>{\raggedright\arraybackslash}p{(\columnwidth - 12\tabcolsep) * \real{0.4286} + 4\tabcolsep}}{%
\textbf{Architecture du projet}} &
\multicolumn{2}{>{\raggedright\arraybackslash}p{(\columnwidth - 12\tabcolsep) * \real{0.2857} + 2\tabcolsep}}{%
\textbf{. . . . . . . . . . . . . . . . . . . . . . .}} &
\begin{minipage}[t]{\linewidth}\raggedright
\begin{quote}
\textbf{41}
\end{quote}
\end{minipage} \\
\textbf{3.6} &
\multicolumn{5}{>{\raggedright\arraybackslash}p{(\columnwidth - 12\tabcolsep) * \real{0.7143} + 8\tabcolsep}}{%
\multirow{2}{*}{\textbf{CONCLUSION . . . . . . . . . . . . . . . . . . .
. . . . . . . . .}}} & \begin{minipage}[t]{\linewidth}\raggedright
\begin{quote}
\textbf{42}
\end{quote}
\end{minipage} \\
Dhia Abdelli & & & & & & \begin{minipage}[t]{\linewidth}\raggedright
\begin{quote}
Page 30
\end{quote}
\end{minipage} \\
\bottomrule()
\end{longtable}

\begin{quote}
ANALYSE ET CONCEPTION

\textbf{3.1 Outil de modélisation}

Pour créer nos diagrammes, nous sommes tournés vers un outil en ligne
simple appelé(Visual Paradigm Online). Il s'agit d'une plateforme
intuitive qui rend la conception facile et claire.

\textbf{3.2 Identifier les acteurs}

Dans notre solution, quatre rôles clés ont des responsabilités
spécifiques, garantissant une collaboration et une exécution des tâches
fluides.

\emph{•} \textbf{Administrateur} : Cet utilisateur est en charge de la
gestion de l'ensemble de la solution, ce qui inclut la supervision des
clients, des analystes, et la création de modèles de rapports. Ils sont
également responsables de tâches telles que la création ou la
suppression de comptes analystes.

\emph{•} \textbf{Analyste niveau 1} : En tant qu'analyste de niveau 1,
leur rôle principal est d'examiner attentivement les alertes qui se
présentent et de procéder à des enquêtes préliminaires en suivant les
procédures établies

\emph{•} \textbf{Analyste niveau 2} : Malgré les analyses préalables du
système, cet utilisateur entreprend systématiquement les enquêtes
essentielles, élaborant une réponse sur mesure à l'incident en fonction
de ses conclusions.

\emph{•} \textbf{Analyste niveau 3} : En temps normal, cet analyste ne
s'immisce pas dans les opérations quotidiennes. Cependant, s'il est
sollicité par un analyste de niveau 2, il est prêt à mener une recherche
spécifique de menaces pour détecter d'éventuels risques.
\end{quote}

\begin{longtable}[]{@{}
  >{\raggedright\arraybackslash}p{(\columnwidth - 2\tabcolsep) * \real{0.5000}}
  >{\raggedright\arraybackslash}p{(\columnwidth - 2\tabcolsep) * \real{0.5000}}@{}}
\toprule()
\begin{minipage}[b]{\linewidth}\raggedright
\begin{quote}
Dhia Abdelli
\end{quote}
\end{minipage} & \begin{minipage}[b]{\linewidth}\raggedright
Page 31
\end{minipage} \\
\midrule()
\endhead
\bottomrule()
\end{longtable}

\begin{quote}
ANALYSE ET CONCEPTION

\textbf{3.3 Spécification des besoins}

Dans cette section, nous commencerons par identifier les besoins
spécifiques de la solution. Celles-ci seront classées en «exigences
fonctionnelles» et «exigences non fonctionnelles». Il est fondamental
d'approfondir ces détails pour garantir le fonctionnement efficace de la
solution.

\textbf{3.3.1} \textbf{Exigences fonctionnelles}

La solution proposée doit répondre à ces exigences spécifiées pour
permettre à la solution SOAR d'atteindre son plein potentiel.

\emph{•} \textbf{La gestion de cas} : Il est essentiel que la solution
propose une interface intuitive pour une gestion de cas fluide. Cela
inclut des options de filtrage avancées, des fonctionnalités de
priorisation et des analyses perspicaces, permettant un suivi rapide des
incidents, une résolution efficace et une surveillance globale.

\emph{•} \textbf{Gestion des IoCs} : La solution doit fournir aux
analystes SOC des outils efficaces pour une analyse et une gestion
rationalisées des IoC (Indicators of Compromise). Ces indicateurs sont
essentiels pour mener des enquêtes approfondies et complètes sur les
incidents de sécurité.

\emph{•} \textbf{Gestion d'actifs} : La solution doit exceller dans la
maintenance et la gestion efficaces d'un inventaire complet d'actifs
surveillés et protégés.

\emph{•} \textbf{Gestion des utilisateurs} : La solution doit s'assurer
que les personnes désignées mentionnées ci-dessus sont placées sous la
supervision de l'administrateur. Ce rôle est crucial pourétablir la
hiérarchie nécessaire à une gestion efficace des utilisateurs au sein du
SOC.

\emph{•} \textbf{Gestion du flux de travail} : La solution doit
permettre aux analystes de créer et de superviser les flux de travail,
en garantissant l'exécution transparente des playbooks et en promouvant
une collaboration optimale entre les analystes humains et la solution

\emph{•} \textbf{Gestion des rapports} : La solution doit faciliter une
gestion efficace des rapports en permettant aux analystes de prendre les
devants dans la création et la gestion des modèles de rapports, avec un
support supplémentaire fourni par des capacités de génération
automatisées.
\end{quote}

\begin{longtable}[]{@{}
  >{\raggedright\arraybackslash}p{(\columnwidth - 2\tabcolsep) * \real{0.5000}}
  >{\raggedright\arraybackslash}p{(\columnwidth - 2\tabcolsep) * \real{0.5000}}@{}}
\toprule()
\begin{minipage}[b]{\linewidth}\raggedright
\begin{quote}
Dhia Abdelli
\end{quote}
\end{minipage} & \begin{minipage}[b]{\linewidth}\raggedright
Page 32
\end{minipage} \\
\midrule()
\endhead
\bottomrule()
\end{longtable}

\begin{quote}
ANALYSE ET CONCEPTION
\end{quote}

\begin{longtable}[]{@{}
  >{\raggedright\arraybackslash}p{(\columnwidth - 2\tabcolsep) * \real{0.5000}}
  >{\raggedright\arraybackslash}p{(\columnwidth - 2\tabcolsep) * \real{0.5000}}@{}}
\toprule()
\begin{minipage}[b]{\linewidth}\raggedright
\begin{quote}
\textbf{3.3.2}
\end{quote}
\end{minipage} & \begin{minipage}[b]{\linewidth}\raggedright
\begin{quote}
\textbf{Exigences non fonctionnelles}
\end{quote}
\end{minipage} \\
\midrule()
\endhead
\bottomrule()
\end{longtable}

\begin{quote}
\emph{•} \textbf{Performance} : La solution doit fonctionner en temps
réel, garantissant des opérations rapides et efficaces tout en mettant
l'accent sur l'importance d'une allocation appropriée des ressources et
en minimisant la latence pour améliorer les performances.

\emph{•} \textbf{Sécurité} : La solution doit déployer et maintenir des
mesures de sécurité solides, garantissant la protection des données
sensibles. Cette action critique est fondamentale pour son rôle d'outil
de cybersécurité fiable.

\emph{•} \textbf{Prise en charge de plusieurs organisations} : La
plate-forme doit être conçue en mettant l'accent sur la prise en charge
efficace de plusieurs organisations, en harmonisant leurs opérations et
leurs flux de travail de manière transparente tout en garantissant une
isolation robuste des données.

\emph{•} \textbf{Authentification} : La plate-forme nécessite un
mécanisme d'authentification robuste pour protéger les données
sensibles, contrôler l'accès et garantir que seuls les utilisateurs
autorisés peuvent interagir avec le système, améliorant ainsi la
sécurité globale.

\textbf{3.4 Conception}

Regardons de plus près les fonctionnalités de la solution SOAR dans
cette section. En utilisant des scénarios de cas d'utilisation, nous
illustrerons clairement le fonctionnement de notre solution. Cela ouvre
la voie à une exploration approfondie, fournissant une compréhension
complète du noyau opérationnel de notre solution.
\end{quote}

\begin{longtable}[]{@{}
  >{\raggedright\arraybackslash}p{(\columnwidth - 2\tabcolsep) * \real{0.5000}}
  >{\raggedright\arraybackslash}p{(\columnwidth - 2\tabcolsep) * \real{0.5000}}@{}}
\toprule()
\begin{minipage}[b]{\linewidth}\raggedright
\begin{quote}
Dhia Abdelli
\end{quote}
\end{minipage} & \begin{minipage}[b]{\linewidth}\raggedright
Page 33
\end{minipage} \\
\midrule()
\endhead
\bottomrule()
\end{longtable}

\begin{quote}
ANALYSE ET CONCEPTION
\end{quote}

\begin{longtable}[]{@{}
  >{\raggedright\arraybackslash}p{(\columnwidth - 2\tabcolsep) * \real{0.5000}}
  >{\raggedright\arraybackslash}p{(\columnwidth - 2\tabcolsep) * \real{0.5000}}@{}}
\toprule()
\begin{minipage}[b]{\linewidth}\raggedright
\begin{quote}
\textbf{3.4.1}
\end{quote}
\end{minipage} & \begin{minipage}[b]{\linewidth}\raggedright
\begin{quote}
\textbf{Diagramme de cas d'utilisation global}
\end{quote}
\end{minipage} \\
\midrule()
\endhead
\bottomrule()
\end{longtable}

\includegraphics[width=7.81111in,height=8.41944in]{vertopal_fa5b2edd966b48bfbc6a830f7abfbc35/media/image20.png}

\begin{longtable}[]{@{}
  >{\raggedright\arraybackslash}p{(\columnwidth - 2\tabcolsep) * \real{0.5000}}
  >{\raggedright\arraybackslash}p{(\columnwidth - 2\tabcolsep) * \real{0.5000}}@{}}
\toprule()
\begin{minipage}[b]{\linewidth}\raggedright
\begin{quote}
Dhia Abdelli
\end{quote}
\end{minipage} & \begin{minipage}[b]{\linewidth}\raggedright
Page 34
\end{minipage} \\
\midrule()
\endhead
\bottomrule()
\end{longtable}

\begin{quote}
ANALYSE ET CONCEPTION
\end{quote}

\begin{longtable}[]{@{}
  >{\raggedright\arraybackslash}p{(\columnwidth - 2\tabcolsep) * \real{0.5000}}
  >{\raggedright\arraybackslash}p{(\columnwidth - 2\tabcolsep) * \real{0.5000}}@{}}
\toprule()
\begin{minipage}[b]{\linewidth}\raggedright
\begin{quote}
\textbf{3.4.2}
\end{quote}
\end{minipage} & \begin{minipage}[b]{\linewidth}\raggedright
\begin{quote}
\textbf{Description détaillée du cas}
\end{quote}
\end{minipage} \\
\midrule()
\endhead
\bottomrule()
\end{longtable}

\begin{quote}
Dans cette section dédiée, nous entamons une analyse approfondie de
notre diagramme de cas d'utilisation global. En examinant chaque cas
d'utilisation individuel, nous obtiendrons un aperçu des divers aspects
des capacités de notre système.

\textbf{3.4.2.1} \textbf{La gestion de cas}

\textbf{Création de cas}

Pour créer un cas, l'analyste peut le faire de deux manières différentes
:

Première façon :
\end{quote}

\begin{longtable}[]{@{}
  >{\raggedright\arraybackslash}p{(\columnwidth - 2\tabcolsep) * \real{0.5000}}
  >{\raggedright\arraybackslash}p{(\columnwidth - 2\tabcolsep) * \real{0.5000}}@{}}
\toprule()
\begin{minipage}[b]{\linewidth}\raggedright
\begin{quote}
Cas d'utilisation
\end{quote}
\end{minipage} & \begin{minipage}[b]{\linewidth}\raggedright
\begin{quote}
Création de cas
\end{quote}
\end{minipage} \\
\midrule()
\endhead
\begin{minipage}[t]{\linewidth}\raggedright
\begin{quote}
Description
\end{quote}
\end{minipage} & \begin{minipage}[t]{\linewidth}\raggedright
\begin{quote}
Permet à un analyste de créer un nouveau cas dans notre solution.
\end{quote}
\end{minipage} \\
\begin{minipage}[t]{\linewidth}\raggedright
\begin{quote}
Acteurs
\end{quote}
\end{minipage} & \begin{minipage}[t]{\linewidth}\raggedright
\begin{quote}
Analyste Niveau 1, Analyste Niveau 2, Analyste Niveau 3
\end{quote}
\end{minipage} \\
\begin{minipage}[t]{\linewidth}\raggedright
\begin{quote}
Pré Condition
\end{quote}
\end{minipage} & \begin{minipage}[t]{\linewidth}\raggedright
\begin{quote}
L'analyste doit être authentifié sur la solution et disposer de
l'autorisation "standard user"
\end{quote}
\end{minipage} \\
\begin{minipage}[t]{\linewidth}\raggedright
\begin{quote}
Post Condition
\end{quote}
\end{minipage} & \begin{minipage}[t]{\linewidth}\raggedright
\begin{quote}
Le système crée un cas
\end{quote}
\end{minipage} \\
\begin{minipage}[t]{\linewidth}\raggedright
\begin{quote}
Flux principal
\end{quote}
\end{minipage} & \begin{minipage}[t]{\linewidth}\raggedright
\begin{quote}
1.L'analyste accède à la page de création de cas
\end{quote}

\begin{longtable}[]{@{}
  >{\raggedright\arraybackslash}p{(\columnwidth - 10\tabcolsep) * \real{0.1667}}
  >{\raggedright\arraybackslash}p{(\columnwidth - 10\tabcolsep) * \real{0.1667}}
  >{\raggedright\arraybackslash}p{(\columnwidth - 10\tabcolsep) * \real{0.1667}}
  >{\raggedright\arraybackslash}p{(\columnwidth - 10\tabcolsep) * \real{0.1667}}
  >{\raggedright\arraybackslash}p{(\columnwidth - 10\tabcolsep) * \real{0.1667}}
  >{\raggedright\arraybackslash}p{(\columnwidth - 10\tabcolsep) * \real{0.1667}}@{}}
\toprule()
\begin{minipage}[b]{\linewidth}\raggedright
2.L'analyste
\end{minipage} & \begin{minipage}[b]{\linewidth}\raggedright
fournit
\end{minipage} & \begin{minipage}[b]{\linewidth}\raggedright
les
\end{minipage} & \begin{minipage}[b]{\linewidth}\raggedright
informations
\end{minipage} &
\multirow{2}{*}{\begin{minipage}[b]{\linewidth}\raggedright
nécessaires
\end{minipage}} &
\multirow{2}{*}{\begin{minipage}[b]{\linewidth}\raggedright
(organisation,
\end{minipage}} \\
\multicolumn{4}{@{}>{\raggedright\arraybackslash}p{(\columnwidth - 10\tabcolsep) * \real{0.6667} + 6\tabcolsep}}{%
\begin{minipage}[b]{\linewidth}\raggedright
\begin{quote}
classification, nom du cas, description)
\end{quote}
\end{minipage}} \\
\midrule()
\endhead
\bottomrule()
\end{longtable}

\begin{quote}
3.Le système valide les informations\\
4.Le système crée un nouveau cas\\
5.L'analyste ajoute des IoC, des alertes, des notes et des tâches au cas
6.Le cas d'utilisation se termine
\end{quote}\strut
\end{minipage} \\
\begin{minipage}[t]{\linewidth}\raggedright
\begin{quote}
Des exceptions
\end{quote}
\end{minipage} & \begin{minipage}[t]{\linewidth}\raggedright
\begin{quote}
Erreur système lors du processus de création de cas
\end{quote}
\end{minipage} \\
\bottomrule()
\end{longtable}

\textbf{TABLEAU 3.1 -- Description textuelle du sous-cas d'utilisation «
Création de cas 1 »}

\begin{longtable}[]{@{}
  >{\raggedright\arraybackslash}p{(\columnwidth - 2\tabcolsep) * \real{0.5000}}
  >{\raggedright\arraybackslash}p{(\columnwidth - 2\tabcolsep) * \real{0.5000}}@{}}
\toprule()
\begin{minipage}[b]{\linewidth}\raggedright
\begin{quote}
Dhia Abdelli
\end{quote}
\end{minipage} & \begin{minipage}[b]{\linewidth}\raggedright
Page 35
\end{minipage} \\
\midrule()
\endhead
\bottomrule()
\end{longtable}

\begin{quote}
ANALYSE ET CONCEPTION

Deuxième façon :
\end{quote}

\begin{longtable}[]{@{}
  >{\raggedright\arraybackslash}p{(\columnwidth - 2\tabcolsep) * \real{0.5000}}
  >{\raggedright\arraybackslash}p{(\columnwidth - 2\tabcolsep) * \real{0.5000}}@{}}
\toprule()
\begin{minipage}[b]{\linewidth}\raggedright
\begin{quote}
Cas d'utilisation
\end{quote}
\end{minipage} & \begin{minipage}[b]{\linewidth}\raggedright
\begin{quote}
Création de cas
\end{quote}
\end{minipage} \\
\midrule()
\endhead
\begin{minipage}[t]{\linewidth}\raggedright
\begin{quote}
Description
\end{quote}
\end{minipage} & \begin{minipage}[t]{\linewidth}\raggedright
\begin{quote}
Permet à un analyste de créer un nouveau cas dans notre solution
\end{quote}
\end{minipage} \\
\begin{minipage}[t]{\linewidth}\raggedright
\begin{quote}
Acteurs
\end{quote}
\end{minipage} & \begin{minipage}[t]{\linewidth}\raggedright
\begin{quote}
Analyste Niveau 1, Analyste Niveau 2, Analyste Niveau 3
\end{quote}
\end{minipage} \\
\begin{minipage}[t]{\linewidth}\raggedright
\begin{quote}
Pré Condition
\end{quote}
\end{minipage} & \begin{minipage}[t]{\linewidth}\raggedright
\begin{quote}
L'analyste doit être authentifié sur la solution et disposer de
l'autorisation "standard user"
\end{quote}
\end{minipage} \\
\begin{minipage}[t]{\linewidth}\raggedright
\begin{quote}
Post Condition
\end{quote}
\end{minipage} & \begin{minipage}[t]{\linewidth}\raggedright
\begin{quote}
Le système crée un cas
\end{quote}
\end{minipage} \\
\begin{minipage}[t]{\linewidth}\raggedright
\begin{quote}
Flux principal
\end{quote}
\end{minipage} & \begin{minipage}[t]{\linewidth}\raggedright
\begin{quote}
1. L'analyste accède à la page des alerts\\
2. l'analyste sélectionne une alerte\\
2.L'analyste fusionne l'alerte dans un nouveau ca.

3.Le système crée un nouveau cas\\
4.le système ajoute l'alerte dans le nouveau cas créé5.Le cas
d'utilisation se termine
\end{quote}\strut
\end{minipage} \\
\begin{minipage}[t]{\linewidth}\raggedright
\begin{quote}
Des exceptions
\end{quote}
\end{minipage} & \begin{minipage}[t]{\linewidth}\raggedright
\begin{quote}
Erreur système lors du processus de création de cas
\end{quote}
\end{minipage} \\
\bottomrule()
\end{longtable}

\textbf{TABLEAU 3.2 -- Description textuelle du sous-cas d'utilisation «
Création de cas 2»}

\begin{quote}
\textbf{Clôture du cas}
\end{quote}

\begin{longtable}[]{@{}
  >{\raggedright\arraybackslash}p{(\columnwidth - 2\tabcolsep) * \real{0.5000}}
  >{\raggedright\arraybackslash}p{(\columnwidth - 2\tabcolsep) * \real{0.5000}}@{}}
\toprule()
\begin{minipage}[b]{\linewidth}\raggedright
\begin{quote}
Cas d'utilisation
\end{quote}
\end{minipage} & \begin{minipage}[b]{\linewidth}\raggedright
\begin{quote}
Clôture du cas
\end{quote}
\end{minipage} \\
\midrule()
\endhead
\begin{minipage}[t]{\linewidth}\raggedright
\begin{quote}
Description
\end{quote}
\end{minipage} & \begin{minipage}[t]{\linewidth}\raggedright
\begin{quote}
Permet à l'analyste de clôturer un dossier une fois que toutes les
tâches ont été accomplies
\end{quote}
\end{minipage} \\
\begin{minipage}[t]{\linewidth}\raggedright
\begin{quote}
Acteurs
\end{quote}
\end{minipage} & \begin{minipage}[t]{\linewidth}\raggedright
\begin{quote}
Analyste Niveau 1, Analyste Niveau 2, Analyste Niveau 3
\end{quote}
\end{minipage} \\
\begin{minipage}[t]{\linewidth}\raggedright
\begin{quote}
Pré Condition
\end{quote}
\end{minipage} & \begin{minipage}[t]{\linewidth}\raggedright
\begin{quote}
L'analyste doit être authentifié sur la solution et disposer de
l'autorisation "standard user"
\end{quote}
\end{minipage} \\
\begin{minipage}[t]{\linewidth}\raggedright
\begin{quote}
Post Condition
\end{quote}
\end{minipage} & \begin{minipage}[t]{\linewidth}\raggedright
\begin{quote}
Le système clôture un cas et le marque comme terminé
\end{quote}
\end{minipage} \\
\begin{minipage}[t]{\linewidth}\raggedright
\begin{quote}
Flux principal
\end{quote}
\end{minipage} & \begin{minipage}[t]{\linewidth}\raggedright
\begin{quote}
1. L'analyste accède à la page des cas\\
2.L'analyste sélectionne un cas et accède à sa page\\
3.L'analyste clique ensuite sur modifier et change son statut en
terminé4.Le système modifie le statut du case\\
5.Le cas d'utilisation se termine
\end{quote}\strut
\end{minipage} \\
\begin{minipage}[t]{\linewidth}\raggedright
\begin{quote}
Des exceptions
\end{quote}
\end{minipage} & \begin{minipage}[t]{\linewidth}\raggedright
\begin{quote}
Erreur système lors du processus de clôturation
\end{quote}
\end{minipage} \\
\bottomrule()
\end{longtable}

\textbf{TABLEAU 3.3 -- Description textuelle du sous-cas d'utilisation «
Clôture du cas »}

\begin{longtable}[]{@{}
  >{\raggedright\arraybackslash}p{(\columnwidth - 2\tabcolsep) * \real{0.5000}}
  >{\raggedright\arraybackslash}p{(\columnwidth - 2\tabcolsep) * \real{0.5000}}@{}}
\toprule()
\begin{minipage}[b]{\linewidth}\raggedright
\begin{quote}
Dhia Abdelli
\end{quote}
\end{minipage} & \begin{minipage}[b]{\linewidth}\raggedright
Page 36
\end{minipage} \\
\midrule()
\endhead
\bottomrule()
\end{longtable}

\begin{quote}
ANALYSE ET CONCEPTION

\textbf{Escalade du cas}
\end{quote}

\begin{longtable}[]{@{}
  >{\raggedright\arraybackslash}p{(\columnwidth - 2\tabcolsep) * \real{0.5000}}
  >{\raggedright\arraybackslash}p{(\columnwidth - 2\tabcolsep) * \real{0.5000}}@{}}
\toprule()
\begin{minipage}[b]{\linewidth}\raggedright
\begin{quote}
Cas d'utilisation
\end{quote}
\end{minipage} & \begin{minipage}[b]{\linewidth}\raggedright
\begin{quote}
Escalader le cas
\end{quote}
\end{minipage} \\
\midrule()
\endhead
\begin{minipage}[t]{\linewidth}\raggedright
\begin{quote}
Description
\end{quote}
\end{minipage} & \begin{minipage}[t]{\linewidth}\raggedright
\begin{quote}
Permet à l'analyste de transmettre le cas à un analyste de niveau
supérieur si le cas l'exige.
\end{quote}
\end{minipage} \\
\begin{minipage}[t]{\linewidth}\raggedright
\begin{quote}
Acteurs
\end{quote}
\end{minipage} & \begin{minipage}[t]{\linewidth}\raggedright
\begin{quote}
Analyste Niveau 1, Analyste Niveau 2
\end{quote}
\end{minipage} \\
\begin{minipage}[t]{\linewidth}\raggedright
\begin{quote}
Pré Condition
\end{quote}
\end{minipage} & \begin{minipage}[t]{\linewidth}\raggedright
\begin{quote}
L'analyste doit être authentifié sur la solution et disposer de
l'autorisation "standard user"
\end{quote}
\end{minipage} \\
\begin{minipage}[t]{\linewidth}\raggedright
\begin{quote}
Post Condition
\end{quote}
\end{minipage} & \begin{minipage}[t]{\linewidth}\raggedright
\begin{quote}
Un analyste de niveau supérieur prendra la place de l'analyste du cas
\end{quote}
\end{minipage} \\
\begin{minipage}[t]{\linewidth}\raggedright
\begin{quote}
Flux principal
\end{quote}
\end{minipage} & \begin{minipage}[t]{\linewidth}\raggedright
\begin{quote}
1. L'analyste accède à la page des cas\\
2. L'analyste sélectionne un cas et accède à sa page\\
3.L'analyste clique sur "Modifier" et remplace "Assigned To" par le
nouvel analyste.

4.Le système met à jour la personne affectée au cas 5.Le cas
d'utilisation se termine
\end{quote}\strut
\end{minipage} \\
\begin{minipage}[t]{\linewidth}\raggedright
\begin{quote}
Des exceptions
\end{quote}
\end{minipage} & \begin{minipage}[t]{\linewidth}\raggedright
\begin{quote}
Erreur système lors du processus.
\end{quote}
\end{minipage} \\
\bottomrule()
\end{longtable}

\textbf{TABLEAU 3.4 -- Description textuelle du sous-cas d'utilisation «
Escalade du cas »}

\begin{quote}
\textbf{3.4.2.2} \textbf{Gestion des flux de travail}

\textbf{Création du flux de travail}
\end{quote}

\begin{longtable}[]{@{}
  >{\raggedright\arraybackslash}p{(\columnwidth - 2\tabcolsep) * \real{0.5000}}
  >{\raggedright\arraybackslash}p{(\columnwidth - 2\tabcolsep) * \real{0.5000}}@{}}
\toprule()
\begin{minipage}[b]{\linewidth}\raggedright
\begin{quote}
Cas d'utilisation
\end{quote}
\end{minipage} & \begin{minipage}[b]{\linewidth}\raggedright
\begin{quote}
Création du flux de travail
\end{quote}
\end{minipage} \\
\midrule()
\endhead
\begin{minipage}[t]{\linewidth}\raggedright
\begin{quote}
Description
\end{quote}
\end{minipage} & \begin{minipage}[t]{\linewidth}\raggedright
\begin{quote}
Permet à l'analyste de créer un nouveau flux de travail
\end{quote}
\end{minipage} \\
\begin{minipage}[t]{\linewidth}\raggedright
\begin{quote}
Acteurs
\end{quote}
\end{minipage} & \begin{minipage}[t]{\linewidth}\raggedright
\begin{quote}
Analyste Niveau 1, Analyste Niveau 2, Analyste Niveau 3
\end{quote}
\end{minipage} \\
\begin{minipage}[t]{\linewidth}\raggedright
\begin{quote}
Pré Condition
\end{quote}
\end{minipage} & \begin{minipage}[t]{\linewidth}\raggedright
\begin{quote}
L'analyste doit être authentifié sur la solution et disposer de
l'autorisation "standard user"
\end{quote}
\end{minipage} \\
\begin{minipage}[t]{\linewidth}\raggedright
\begin{quote}
Post Condition
\end{quote}
\end{minipage} & \begin{minipage}[t]{\linewidth}\raggedright
\begin{quote}
Un nouveau flux de travail sera ajouté
\end{quote}
\end{minipage} \\
\begin{minipage}[t]{\linewidth}\raggedright
\begin{quote}
Flux principal
\end{quote}
\end{minipage} & \begin{minipage}[t]{\linewidth}\raggedright
\begin{quote}
1. L'analyste accède à la page des flux de travail\\
2.L'analyste clique sur "nouveau"\\
3.L'analyste fournit les informations nécessaires (nom, description).

4.Le système valide les informations\\
5.Le système crée un nouveau flux de travail vide\\
6.L'analyste utilise l'interface pour créer un flux de travail 5.Le cas
d'utilisation se termine
\end{quote}\strut
\end{minipage} \\
\begin{minipage}[t]{\linewidth}\raggedright
\begin{quote}
Des exceptions
\end{quote}
\end{minipage} & \begin{minipage}[t]{\linewidth}\raggedright
\begin{quote}
Erreur système lors du processus de création de flux de travail
\end{quote}
\end{minipage} \\
\bottomrule()
\end{longtable}

\textbf{TABLEAU 3.5 -- Description textuelle du sous-cas d'utilisation
«Création du flux de travail»}

\begin{longtable}[]{@{}
  >{\raggedright\arraybackslash}p{(\columnwidth - 2\tabcolsep) * \real{0.5000}}
  >{\raggedright\arraybackslash}p{(\columnwidth - 2\tabcolsep) * \real{0.5000}}@{}}
\toprule()
\begin{minipage}[b]{\linewidth}\raggedright
\begin{quote}
Dhia Abdelli
\end{quote}
\end{minipage} & \begin{minipage}[b]{\linewidth}\raggedright
Page 37
\end{minipage} \\
\midrule()
\endhead
\bottomrule()
\end{longtable}

\begin{quote}
ANALYSE ET CONCEPTION

\textbf{Modification du flux de travail}
\end{quote}

\begin{longtable}[]{@{}
  >{\raggedright\arraybackslash}p{(\columnwidth - 2\tabcolsep) * \real{0.5000}}
  >{\raggedright\arraybackslash}p{(\columnwidth - 2\tabcolsep) * \real{0.5000}}@{}}
\toprule()
\begin{minipage}[b]{\linewidth}\raggedright
\begin{quote}
Cas d'utilisation
\end{quote}
\end{minipage} & \begin{minipage}[b]{\linewidth}\raggedright
\begin{quote}
Modification du flux de travai
\end{quote}
\end{minipage} \\
\midrule()
\endhead
\begin{minipage}[t]{\linewidth}\raggedright
\begin{quote}
Description
\end{quote}
\end{minipage} & \begin{minipage}[t]{\linewidth}\raggedright
\begin{quote}
Permet à l'analyste de modifier le flux de travail
\end{quote}
\end{minipage} \\
\begin{minipage}[t]{\linewidth}\raggedright
\begin{quote}
Acteurs
\end{quote}
\end{minipage} & \begin{minipage}[t]{\linewidth}\raggedright
\begin{quote}
Analyste Niveau 1, Analyste Niveau 2, Analyste Niveau 3
\end{quote}
\end{minipage} \\
\begin{minipage}[t]{\linewidth}\raggedright
\begin{quote}
Pré Condition
\end{quote}
\end{minipage} & \begin{minipage}[t]{\linewidth}\raggedright
\begin{quote}
L'analyste doit être authentifié sur la solution et disposer de
l'autorisation "standard user"
\end{quote}
\end{minipage} \\
\begin{minipage}[t]{\linewidth}\raggedright
\begin{quote}
Post Condition
\end{quote}
\end{minipage} & \begin{minipage}[t]{\linewidth}\raggedright
\begin{quote}
Un flux de travail sera modifié
\end{quote}
\end{minipage} \\
\begin{minipage}[t]{\linewidth}\raggedright
\begin{quote}
Flux principal
\end{quote}
\end{minipage} & \begin{minipage}[t]{\linewidth}\raggedright
\begin{quote}
1. L'analyste accède à la page des flux de travail 2.L'analyste
sélectionne un flux de travail\\
3.L'analyste modifie le flux de travail\\
4.Le système met à jour le flux de travail\\
5.Le cas d'utilisation se termine
\end{quote}\strut
\end{minipage} \\
\begin{minipage}[t]{\linewidth}\raggedright
\begin{quote}
Des exceptions
\end{quote}
\end{minipage} & \begin{minipage}[t]{\linewidth}\raggedright
\begin{quote}
Erreur système lors du processus de modification
\end{quote}
\end{minipage} \\
\bottomrule()
\end{longtable}

\textbf{TABLEAU 3.6 -- Description textuelle du sous-cas d'utilisation
«Modification du flux de travail»}

\begin{quote}
\textbf{Suppression du flux de travail}
\end{quote}

\begin{longtable}[]{@{}
  >{\raggedright\arraybackslash}p{(\columnwidth - 2\tabcolsep) * \real{0.5000}}
  >{\raggedright\arraybackslash}p{(\columnwidth - 2\tabcolsep) * \real{0.5000}}@{}}
\toprule()
\begin{minipage}[b]{\linewidth}\raggedright
\begin{quote}
Cas d'utilisation
\end{quote}
\end{minipage} & \begin{minipage}[b]{\linewidth}\raggedright
\begin{quote}
Suppression du flux de travail
\end{quote}
\end{minipage} \\
\midrule()
\endhead
\begin{minipage}[t]{\linewidth}\raggedright
\begin{quote}
Description
\end{quote}
\end{minipage} & \begin{minipage}[t]{\linewidth}\raggedright
\begin{quote}
Permet à l'analyste de supprimer le flux de travail
\end{quote}
\end{minipage} \\
\begin{minipage}[t]{\linewidth}\raggedright
\begin{quote}
Acteurs
\end{quote}
\end{minipage} & \begin{minipage}[t]{\linewidth}\raggedright
\begin{quote}
Analyste Niveau 1, Analyste Niveau 2, Analyste Niveau 3
\end{quote}
\end{minipage} \\
\begin{minipage}[t]{\linewidth}\raggedright
\begin{quote}
Pré Condition
\end{quote}
\end{minipage} & \begin{minipage}[t]{\linewidth}\raggedright
\begin{quote}
L'analyste doit être authentifié sur la solution et disposer de
l'autorisation "standard user"
\end{quote}
\end{minipage} \\
\begin{minipage}[t]{\linewidth}\raggedright
\begin{quote}
Post Condition
\end{quote}
\end{minipage} & \begin{minipage}[t]{\linewidth}\raggedright
\begin{quote}
Un flux de travail sera supprimé
\end{quote}
\end{minipage} \\
\begin{minipage}[t]{\linewidth}\raggedright
\begin{quote}
Flux principal
\end{quote}
\end{minipage} & \begin{minipage}[t]{\linewidth}\raggedright
\begin{quote}
1. L'analyste accède à la page des flux de travail 2.L'analyste
sélectionne un flux de travail\\
3.l'analyste clique sure supprimer\\
4.Le système supprime le flux de travail\\
5.Le cas d'utilisation se termine
\end{quote}\strut
\end{minipage} \\
\begin{minipage}[t]{\linewidth}\raggedright
\begin{quote}
Des exceptions
\end{quote}
\end{minipage} & \begin{minipage}[t]{\linewidth}\raggedright
\begin{quote}
Erreur système lors du processus de Suppression
\end{quote}
\end{minipage} \\
\bottomrule()
\end{longtable}

\textbf{TABLEAU 3.7 -- Description textuelle du sous-cas d'utilisation
«Suppression du flux de travail»}

\begin{longtable}[]{@{}
  >{\raggedright\arraybackslash}p{(\columnwidth - 2\tabcolsep) * \real{0.5000}}
  >{\raggedright\arraybackslash}p{(\columnwidth - 2\tabcolsep) * \real{0.5000}}@{}}
\toprule()
\begin{minipage}[b]{\linewidth}\raggedright
\begin{quote}
Dhia Abdelli
\end{quote}
\end{minipage} & \begin{minipage}[b]{\linewidth}\raggedright
Page 38
\end{minipage} \\
\midrule()
\endhead
\bottomrule()
\end{longtable}

\begin{quote}
ANALYSE ET CONCEPTION
\end{quote}

\begin{longtable}[]{@{}
  >{\raggedright\arraybackslash}p{(\columnwidth - 2\tabcolsep) * \real{0.5000}}
  >{\raggedright\arraybackslash}p{(\columnwidth - 2\tabcolsep) * \real{0.5000}}@{}}
\toprule()
\begin{minipage}[b]{\linewidth}\raggedright
\begin{quote}
\textbf{3.4.2.3}
\end{quote}
\end{minipage} & \begin{minipage}[b]{\linewidth}\raggedright
\begin{quote}
\textbf{Authentification}
\end{quote}
\end{minipage} \\
\midrule()
\endhead
\bottomrule()
\end{longtable}

\begin{longtable}[]{@{}
  >{\raggedright\arraybackslash}p{(\columnwidth - 2\tabcolsep) * \real{0.5000}}
  >{\raggedright\arraybackslash}p{(\columnwidth - 2\tabcolsep) * \real{0.5000}}@{}}
\toprule()
\begin{minipage}[b]{\linewidth}\raggedright
\begin{quote}
Cas d'utilisation
\end{quote}
\end{minipage} & \begin{minipage}[b]{\linewidth}\raggedright
\begin{quote}
Authentification
\end{quote}
\end{minipage} \\
\midrule()
\endhead
\begin{minipage}[t]{\linewidth}\raggedright
\begin{quote}
Acteurs
\end{quote}
\end{minipage} & \begin{minipage}[t]{\linewidth}\raggedright
\begin{quote}
Administrateur, Analyste Niveau 1, Analyste Niveau 2, Analyste Niveau 3
\end{quote}
\end{minipage} \\
\begin{minipage}[t]{\linewidth}\raggedright
\begin{quote}
Pré Condition
\end{quote}
\end{minipage} & \begin{minipage}[t]{\linewidth}\raggedright
\begin{quote}
Un compte utilisateur existe
\end{quote}
\end{minipage} \\
\begin{minipage}[t]{\linewidth}\raggedright
\begin{quote}
Post Condition
\end{quote}
\end{minipage} & \begin{minipage}[t]{\linewidth}\raggedright
\begin{quote}
Un nouveau workflow sera ajouté
\end{quote}
\end{minipage} \\
\begin{minipage}[t]{\linewidth}\raggedright
\begin{quote}
Flux principal
\end{quote}
\end{minipage} & \begin{minipage}[t]{\linewidth}\raggedright
\begin{quote}
1.L'utilisateur accède à la page de connexion\\
2.l'utilisateur saisit des informations d'identification valides (nom
d'utilisateur et mot de passe)\\
3.Le système vérifie les informations d'identification\\
4.Le système accorde l'accès à l'utilisateur\\
5.Le cas d'utilisation se termine
\end{quote}\strut
\end{minipage} \\
\begin{minipage}[t]{\linewidth}\raggedright
\begin{quote}
Des exceptions
\end{quote}
\end{minipage} & \begin{minipage}[t]{\linewidth}\raggedright
\begin{quote}
-Erreur système lors du processus d'authentification-Le compte
utilisateur est suspendu ou désactivé
\end{quote}
\end{minipage} \\
\bottomrule()
\end{longtable}

\textbf{TABLEAU 3.8 -- Description textuelle du sous-cas d'utilisation
«Authentification»}

\begin{quote}
\textbf{3.4.2.4} \textbf{Gestion des organisations}

\textbf{Ajout d'une Organisation}
\end{quote}

\begin{longtable}[]{@{}
  >{\raggedright\arraybackslash}p{(\columnwidth - 2\tabcolsep) * \real{0.5000}}
  >{\raggedright\arraybackslash}p{(\columnwidth - 2\tabcolsep) * \real{0.5000}}@{}}
\toprule()
\begin{minipage}[b]{\linewidth}\raggedright
\begin{quote}
Cas d'utilisation
\end{quote}
\end{minipage} & \begin{minipage}[b]{\linewidth}\raggedright
\begin{quote}
Ajouter une Organisation
\end{quote}
\end{minipage} \\
\midrule()
\endhead
\begin{minipage}[t]{\linewidth}\raggedright
\begin{quote}
Description
\end{quote}
\end{minipage} & \begin{minipage}[t]{\linewidth}\raggedright
\begin{quote}
Permet à un administrateur d'ajouter une nouvelle organisation à la
solution
\end{quote}
\end{minipage} \\
\begin{minipage}[t]{\linewidth}\raggedright
\begin{quote}
Acteurs
\end{quote}
\end{minipage} & \begin{minipage}[t]{\linewidth}\raggedright
\begin{quote}
Administrateur
\end{quote}
\end{minipage} \\
\begin{minipage}[t]{\linewidth}\raggedright
\begin{quote}
Pré Condition
\end{quote}
\end{minipage} & \begin{minipage}[t]{\linewidth}\raggedright
\begin{quote}
L'administrateur est connecté et a les permissions nécessaires
\end{quote}
\end{minipage} \\
\begin{minipage}[t]{\linewidth}\raggedright
\begin{quote}
Post Condition
\end{quote}
\end{minipage} & \begin{minipage}[t]{\linewidth}\raggedright
\begin{quote}
Une nouvelle organisation est ajoutée avec succès à la solution
\end{quote}
\end{minipage} \\
\begin{minipage}[t]{\linewidth}\raggedright
\begin{quote}
Flux principal
\end{quote}
\end{minipage} & \begin{minipage}[t]{\linewidth}\raggedright
\begin{quote}
1.L'administrateur accède à la page 'Ajouter une Organisation'\\
2.L'administrateur entre les détails de l'organisation (nom, adresse,
coordonnées)\\
3.La solution valide l'information\\
4.La solution crée un nouveau profil d'organisation\\
5.Le cas d'utilisation se termine
\end{quote}\strut
\end{minipage} \\
\begin{minipage}[t]{\linewidth}\raggedright
\begin{quote}
Des exceptions
\end{quote}
\end{minipage} & \begin{minipage}[t]{\linewidth}\raggedright
\begin{quote}
-Erreur système lors du processus de création de
l'organisation-L'administrateur ne possède pas les permissions
nécessaires
\end{quote}
\end{minipage} \\
\bottomrule()
\end{longtable}

\textbf{TABLEAU 3.9 -- Description textuelle du sous-cas d'utilisation
«Ajout d'une Organisation»}

\begin{longtable}[]{@{}
  >{\raggedright\arraybackslash}p{(\columnwidth - 2\tabcolsep) * \real{0.5000}}
  >{\raggedright\arraybackslash}p{(\columnwidth - 2\tabcolsep) * \real{0.5000}}@{}}
\toprule()
\begin{minipage}[b]{\linewidth}\raggedright
\begin{quote}
Dhia Abdelli
\end{quote}
\end{minipage} & \begin{minipage}[b]{\linewidth}\raggedright
Page 39
\end{minipage} \\
\midrule()
\endhead
\bottomrule()
\end{longtable}

\begin{quote}
ANALYSE ET CONCEPTION

\textbf{Suppression d'une Organisation}
\end{quote}

\begin{longtable}[]{@{}
  >{\raggedright\arraybackslash}p{(\columnwidth - 2\tabcolsep) * \real{0.5000}}
  >{\raggedright\arraybackslash}p{(\columnwidth - 2\tabcolsep) * \real{0.5000}}@{}}
\toprule()
\begin{minipage}[b]{\linewidth}\raggedright
\begin{quote}
Cas d'utilisation
\end{quote}
\end{minipage} & \begin{minipage}[b]{\linewidth}\raggedright
\begin{quote}
Suppression d'une Organisation
\end{quote}
\end{minipage} \\
\midrule()
\endhead
\begin{minipage}[t]{\linewidth}\raggedright
\begin{quote}
Description
\end{quote}
\end{minipage} & \begin{minipage}[t]{\linewidth}\raggedright
\begin{quote}
Permet à un administrateur de supprimer une organisation du système
\end{quote}
\end{minipage} \\
\begin{minipage}[t]{\linewidth}\raggedright
\begin{quote}
Acteurs
\end{quote}
\end{minipage} & \begin{minipage}[t]{\linewidth}\raggedright
\begin{quote}
Administrateur
\end{quote}
\end{minipage} \\
\begin{minipage}[t]{\linewidth}\raggedright
\begin{quote}
Pré Condition
\end{quote}
\end{minipage} & \begin{minipage}[t]{\linewidth}\raggedright
\begin{quote}
L'administrateur est connecté et a les permissions nécessaires
\end{quote}
\end{minipage} \\
\begin{minipage}[t]{\linewidth}\raggedright
\begin{quote}
Post Condition
\end{quote}
\end{minipage} & \begin{minipage}[t]{\linewidth}\raggedright
\begin{quote}
'organisation est supprimée avec succès de la solution
\end{quote}
\end{minipage} \\
\begin{minipage}[t]{\linewidth}\raggedright
\begin{quote}
Flux principal
\end{quote}
\end{minipage} & \begin{minipage}[t]{\linewidth}\raggedright
\begin{quote}
1.L'administrateur accède à la page de gestion des Organisations
2L'administrateur sélectionne l'organisation à supprimer\\
3.La solution demande une confirmation pour la suppression
4.L'administrateur confirme la suppression\\
5.La solution supprime l'organisation de la solution\\
6.Le cas d'utilisation se termine
\end{quote}\strut
\end{minipage} \\
\begin{minipage}[t]{\linewidth}\raggedright
\begin{quote}
Des exceptions
\end{quote}
\end{minipage} & \begin{minipage}[t]{\linewidth}\raggedright
\begin{quote}
-Erreur système lors du processus de suppression de
l'organisation-L'administrateur ne possède pas les permissions
nécessaires
\end{quote}
\end{minipage} \\
\bottomrule()
\end{longtable}

\begin{quote}
\textbf{TABLEAU 3.10 -- Description textuelle du sous-cas d'utilisation
«Suppression d'une Organisation»}

\textbf{Modification d'une Organisation}
\end{quote}

\begin{longtable}[]{@{}
  >{\raggedright\arraybackslash}p{(\columnwidth - 2\tabcolsep) * \real{0.5000}}
  >{\raggedright\arraybackslash}p{(\columnwidth - 2\tabcolsep) * \real{0.5000}}@{}}
\toprule()
\begin{minipage}[b]{\linewidth}\raggedright
\begin{quote}
Cas d'utilisation
\end{quote}
\end{minipage} & \begin{minipage}[b]{\linewidth}\raggedright
\begin{quote}
Modifier une Organisation
\end{quote}
\end{minipage} \\
\midrule()
\endhead
\begin{minipage}[t]{\linewidth}\raggedright
\begin{quote}
Description
\end{quote}
\end{minipage} & \begin{minipage}[t]{\linewidth}\raggedright
\begin{quote}
Permet à un administrateur de modifier les détails d'une organisation
existante dans la solution
\end{quote}
\end{minipage} \\
\begin{minipage}[t]{\linewidth}\raggedright
\begin{quote}
Acteurs
\end{quote}
\end{minipage} & \begin{minipage}[t]{\linewidth}\raggedright
\begin{quote}
Administrateur
\end{quote}
\end{minipage} \\
\begin{minipage}[t]{\linewidth}\raggedright
\begin{quote}
Pré Condition
\end{quote}
\end{minipage} & \begin{minipage}[t]{\linewidth}\raggedright
\begin{quote}
L'administrateur est connecté et a les permissions nécessaires
\end{quote}
\end{minipage} \\
\begin{minipage}[t]{\linewidth}\raggedright
\begin{quote}
Post Condition
\end{quote}
\end{minipage} & Les détails de l'organisation sont modifiés avec succès
dans la solution \\
\begin{minipage}[t]{\linewidth}\raggedright
\begin{quote}
Flux principal
\end{quote}
\end{minipage} & \begin{minipage}[t]{\linewidth}\raggedright
\begin{quote}
1.L'administrateur sélectionne l'organisation à modifier 2.Le système
affiche les détails de l'organisation\\
3.L'administrateur clique sur le bouton "Modifier" 4.L'administrateur
modifie les informations nécessaires 5.Le système valide et enregistre
les modifications 6.Le cas d'utilisation se termine
\end{quote}\strut
\end{minipage} \\
\begin{minipage}[t]{\linewidth}\raggedright
\begin{quote}
Des exceptions
\end{quote}
\end{minipage} & \begin{minipage}[t]{\linewidth}\raggedright
\begin{quote}
-Erreur système lors du processus de suppression de
l'organisation-L'administrateur ne possède pas les permissions
nécessaires
\end{quote}
\end{minipage} \\
\bottomrule()
\end{longtable}

\begin{quote}
\textbf{TABLEAU 3.11 -- Description textuelle du sous-cas d'utilisation
«Modification d'une Organisation»}
\end{quote}

\begin{longtable}[]{@{}
  >{\raggedright\arraybackslash}p{(\columnwidth - 2\tabcolsep) * \real{0.5000}}
  >{\raggedright\arraybackslash}p{(\columnwidth - 2\tabcolsep) * \real{0.5000}}@{}}
\toprule()
\begin{minipage}[b]{\linewidth}\raggedright
\begin{quote}
Dhia Abdelli
\end{quote}
\end{minipage} & \begin{minipage}[b]{\linewidth}\raggedright
Page 40
\end{minipage} \\
\midrule()
\endhead
\bottomrule()
\end{longtable}

\begin{quote}
ANALYSE ET CONCEPTION

\textbf{3.5 Architecture du projet}
\end{quote}

\includegraphics[width=7.81111in,height=8.41944in]{vertopal_fa5b2edd966b48bfbc6a830f7abfbc35/media/image21.png}

\begin{longtable}[]{@{}
  >{\raggedright\arraybackslash}p{(\columnwidth - 2\tabcolsep) * \real{0.5000}}
  >{\raggedright\arraybackslash}p{(\columnwidth - 2\tabcolsep) * \real{0.5000}}@{}}
\toprule()
\begin{minipage}[b]{\linewidth}\raggedright
\begin{quote}
Dhia Abdelli
\end{quote}
\end{minipage} & \begin{minipage}[b]{\linewidth}\raggedright
Page 41
\end{minipage} \\
\midrule()
\endhead
\bottomrule()
\end{longtable}

\begin{quote}
ANALYSE ET CONCEPTION

La figure ci-dessus donne un aperçu clair de l'architecture intégrée de
notre projet, démontrant l'intégration et la communication entre toutes
les solutions. Cette approche cohérente garantit des performances
optimisées et des opérations rationalisées.

\textbf{3.6 CONCLUSION}

Dans ce chapitre, nous avons défini les exigences clés de notre
plateforme SOAR, en mettant l'accent sur la satisfaction des besoins
spécifiques de nos utilisateurs. Nous avons également défini les
principaux rôles au sein de la plateforme. De plus, nous avons classé
diverses fonctions dans des tableaux structurés pour plus de clarté.
Enfin, nous avons discuté de l'architecture unique qui prend en charge
notre solution.
\end{quote}

\begin{longtable}[]{@{}
  >{\raggedright\arraybackslash}p{(\columnwidth - 2\tabcolsep) * \real{0.5000}}
  >{\raggedright\arraybackslash}p{(\columnwidth - 2\tabcolsep) * \real{0.5000}}@{}}
\toprule()
\begin{minipage}[b]{\linewidth}\raggedright
\begin{quote}
Dhia Abdelli
\end{quote}
\end{minipage} & \begin{minipage}[b]{\linewidth}\raggedright
Page 42
\end{minipage} \\
\midrule()
\endhead
\bottomrule()
\end{longtable}

Chapitre

\begin{longtable}[]{@{}
  >{\raggedright\arraybackslash}p{(\columnwidth - 2\tabcolsep) * \real{0.5000}}
  >{\raggedright\arraybackslash}p{(\columnwidth - 2\tabcolsep) * \real{0.5000}}@{}}
\toprule()
\begin{minipage}[b]{\linewidth}\raggedright
\begin{longtable}[]{@{}
  >{\raggedright\arraybackslash}p{(\columnwidth - 0\tabcolsep) * \real{1.0000}}@{}}
\toprule()
\begin{minipage}[b]{\linewidth}\raggedright
\textbf{4}
\end{minipage} \\
\midrule()
\endhead
\bottomrule()
\end{longtable}
\end{minipage} & \begin{minipage}[b]{\linewidth}\raggedright
\begin{quote}
\textbf{Réalisation}
\end{quote}
\end{minipage} \\
\midrule()
\endhead
\bottomrule()
\end{longtable}

\begin{quote}
\textbf{Sommaire}
\end{quote}

\begin{longtable}[]{@{}
  >{\raggedright\arraybackslash}p{(\columnwidth - 16\tabcolsep) * \real{0.1111}}
  >{\raggedright\arraybackslash}p{(\columnwidth - 16\tabcolsep) * \real{0.1111}}
  >{\raggedright\arraybackslash}p{(\columnwidth - 16\tabcolsep) * \real{0.1111}}
  >{\raggedright\arraybackslash}p{(\columnwidth - 16\tabcolsep) * \real{0.1111}}
  >{\raggedright\arraybackslash}p{(\columnwidth - 16\tabcolsep) * \real{0.1111}}
  >{\raggedright\arraybackslash}p{(\columnwidth - 16\tabcolsep) * \real{0.1111}}
  >{\raggedright\arraybackslash}p{(\columnwidth - 16\tabcolsep) * \real{0.1111}}
  >{\raggedright\arraybackslash}p{(\columnwidth - 16\tabcolsep) * \real{0.1111}}
  >{\raggedright\arraybackslash}p{(\columnwidth - 16\tabcolsep) * \real{0.1111}}@{}}
\toprule()
\begin{minipage}[b]{\linewidth}\raggedright
\textbf{4.1}
\end{minipage} &
\multicolumn{7}{>{\raggedright\arraybackslash}p{(\columnwidth - 16\tabcolsep) * \real{0.7778} + 12\tabcolsep}}{%
\begin{minipage}[b]{\linewidth}\raggedright
\textbf{Inroduction . . . . . . . . . . . . . . . . . . . . . . . . . .
. . . .}
\end{minipage}} & \begin{minipage}[b]{\linewidth}\raggedright
\begin{quote}
\textbf{44}
\end{quote}
\end{minipage} \\
\midrule()
\endhead
\textbf{4.2} &
\multicolumn{7}{>{\raggedright\arraybackslash}p{(\columnwidth - 16\tabcolsep) * \real{0.7778} + 12\tabcolsep}}{%
\textbf{Environnement de travail . . . . . . . . . . . . . . . . . . . .
. .}} & \begin{minipage}[t]{\linewidth}\raggedright
\begin{quote}
\textbf{44}
\end{quote}
\end{minipage} \\
\multirow{3}{*}{\textbf{4.3}} & 4.2.1 &
\multicolumn{6}{>{\raggedright\arraybackslash}p{(\columnwidth - 16\tabcolsep) * \real{0.6667} + 10\tabcolsep}}{%
Composants logiciels . . . . . . . . . . . . . . . . . . . . . . . .} &
\begin{minipage}[t]{\linewidth}\raggedright
\begin{quote}
44
\end{quote}
\end{minipage} \\
& 4.2.2 &
\multicolumn{6}{>{\raggedright\arraybackslash}p{(\columnwidth - 16\tabcolsep) * \real{0.6667} + 10\tabcolsep}}{%
Composants matériels . . . . . . . . . . . . . . . . . . . . . . .} &
\begin{minipage}[t]{\linewidth}\raggedright
\begin{quote}
44
\end{quote}
\end{minipage} \\
&
\multicolumn{7}{>{\raggedright\arraybackslash}p{(\columnwidth - 16\tabcolsep) * \real{0.7778} + 12\tabcolsep}}{%
\textbf{La solution SOAR . . . . . . . . . . . . . . . . . . . . . . . .
. .}} & \begin{minipage}[t]{\linewidth}\raggedright
\begin{quote}
\textbf{46}
\end{quote}
\end{minipage} \\
\multirow{14}{*}{\textbf{4.4}} & 4.3.1 & Interface des cas &
\multicolumn{5}{>{\raggedright\arraybackslash}p{(\columnwidth - 16\tabcolsep) * \real{0.5556} + 8\tabcolsep}}{%
. . . . . . . . . . . . . . . . . . . . . . . . . .} &
\begin{minipage}[t]{\linewidth}\raggedright
\begin{quote}
46
\end{quote}
\end{minipage} \\
& 4.3.2 &
\multicolumn{6}{>{\raggedright\arraybackslash}p{(\columnwidth - 16\tabcolsep) * \real{0.6667} + 10\tabcolsep}}{%
Interface d'alertes . . . . . . . . . . . . . . . . . . . . . . . . . .}
& \begin{minipage}[t]{\linewidth}\raggedright
\begin{quote}
46
\end{quote}
\end{minipage} \\
& 4.3.3 &
\multicolumn{6}{>{\raggedright\arraybackslash}p{(\columnwidth - 16\tabcolsep) * \real{0.6667} + 10\tabcolsep}}{%
Aperçu de cas . . . . . . . . . . . . . . . . . . . . . . . . . . . .} &
\begin{minipage}[t]{\linewidth}\raggedright
\begin{quote}
47
\end{quote}
\end{minipage} \\
& 4.3.4 &
\multicolumn{4}{>{\raggedright\arraybackslash}p{(\columnwidth - 16\tabcolsep) * \real{0.4444} + 6\tabcolsep}}{%
\begin{minipage}[t]{\linewidth}\raggedright
\begin{quote}
Interface gestion des organisations
\end{quote}
\end{minipage}} &
\multicolumn{2}{>{\raggedright\arraybackslash}p{(\columnwidth - 16\tabcolsep) * \real{0.2222} + 2\tabcolsep}}{%
. . . . . . . . . . . . . . . .} &
\begin{minipage}[t]{\linewidth}\raggedright
\begin{quote}
48
\end{quote}
\end{minipage} \\
& 4.3.5 &
\multicolumn{2}{>{\raggedright\arraybackslash}p{(\columnwidth - 16\tabcolsep) * \real{0.2222} + 2\tabcolsep}}{%
Interface d'organisation} &
\multicolumn{4}{>{\raggedright\arraybackslash}p{(\columnwidth - 16\tabcolsep) * \real{0.4444} + 6\tabcolsep}}{%
\begin{minipage}[t]{\linewidth}\raggedright
\begin{quote}
. . . . . . . . . . . . . . . . . . . . . .
\end{quote}
\end{minipage}} & \begin{minipage}[t]{\linewidth}\raggedright
\begin{quote}
49
\end{quote}
\end{minipage} \\
& 4.3.6 &
\multicolumn{6}{>{\raggedright\arraybackslash}p{(\columnwidth - 16\tabcolsep) * \real{0.6667} + 10\tabcolsep}}{%
Interface de gestion des utilisateurs . . . . . . . . . . . . . . . .} &
\begin{minipage}[t]{\linewidth}\raggedright
\begin{quote}
50
\end{quote}
\end{minipage} \\
& 4.3.7 &
\multicolumn{4}{>{\raggedright\arraybackslash}p{(\columnwidth - 16\tabcolsep) * \real{0.4444} + 6\tabcolsep}}{%
Interface de gestion des modules} &
\multicolumn{2}{>{\raggedright\arraybackslash}p{(\columnwidth - 16\tabcolsep) * \real{0.2222} + 2\tabcolsep}}{%
\begin{minipage}[t]{\linewidth}\raggedright
\begin{quote}
. . . . . . . . . . . . . . . . .
\end{quote}
\end{minipage}} & \begin{minipage}[t]{\linewidth}\raggedright
\begin{quote}
50
\end{quote}
\end{minipage} \\
& 4.3.8 &
\multicolumn{5}{>{\raggedright\arraybackslash}p{(\columnwidth - 16\tabcolsep) * \real{0.5556} + 8\tabcolsep}}{%
Intégration de Vélociraptor sur DFIR IRIS} & . . . . . . . . . . . &
\begin{minipage}[t]{\linewidth}\raggedright
\begin{quote}
51
\end{quote}
\end{minipage} \\
& 4.3.9 &
\multicolumn{2}{>{\raggedright\arraybackslash}p{(\columnwidth - 16\tabcolsep) * \real{0.2222} + 2\tabcolsep}}{%
\begin{minipage}[t]{\linewidth}\raggedright
\begin{quote}
Liste des fichiers analysés
\end{quote}
\end{minipage}} &
\multicolumn{4}{>{\raggedright\arraybackslash}p{(\columnwidth - 16\tabcolsep) * \real{0.4444} + 6\tabcolsep}}{%
. . . . . . . . . . . . . . . . . . . . .} &
\begin{minipage}[t]{\linewidth}\raggedright
\begin{quote}
51
\end{quote}
\end{minipage} \\
&
\multicolumn{3}{>{\raggedright\arraybackslash}p{(\columnwidth - 16\tabcolsep) * \real{0.3333} + 4\tabcolsep}}{%
4.3.10 Aperçu du fichier analysé} &
\multicolumn{4}{>{\raggedright\arraybackslash}p{(\columnwidth - 16\tabcolsep) * \real{0.4444} + 6\tabcolsep}}{%
. . . . . . . . . . . . . . . . . . . . .} &
\begin{minipage}[t]{\linewidth}\raggedright
\begin{quote}
52
\end{quote}
\end{minipage} \\
&
\multicolumn{7}{>{\raggedright\arraybackslash}p{(\columnwidth - 16\tabcolsep) * \real{0.7778} + 12\tabcolsep}}{%
\begin{minipage}[t]{\linewidth}\raggedright
\begin{quote}
4.3.11 Interface du Honeypot . . . . . . . . . . . . . . . . . . . . . .
.
\end{quote}
\end{minipage}} & \begin{minipage}[t]{\linewidth}\raggedright
\begin{quote}
53
\end{quote}
\end{minipage} \\
&
\multicolumn{4}{>{\raggedright\arraybackslash}p{(\columnwidth - 16\tabcolsep) * \real{0.4444} + 6\tabcolsep}}{%
4.3.12 Interface des flux de travail} &
\multicolumn{3}{>{\raggedright\arraybackslash}p{(\columnwidth - 16\tabcolsep) * \real{0.3333} + 4\tabcolsep}}{%
. . . . . . . . . . . . . . . . . . . .} &
\begin{minipage}[t]{\linewidth}\raggedright
\begin{quote}
54
\end{quote}
\end{minipage} \\
&
\multicolumn{5}{>{\raggedright\arraybackslash}p{(\columnwidth - 16\tabcolsep) * \real{0.5556} + 8\tabcolsep}}{%
\begin{minipage}[t]{\linewidth}\raggedright
\begin{quote}
4.3.13 Scénarios de flux de travail réels
\end{quote}
\end{minipage}} &
\multicolumn{2}{>{\raggedright\arraybackslash}p{(\columnwidth - 16\tabcolsep) * \real{0.2222} + 2\tabcolsep}}{%
\begin{minipage}[t]{\linewidth}\raggedright
\begin{quote}
. . . . . . . . . . . . . . . . .
\end{quote}
\end{minipage}} & \begin{minipage}[t]{\linewidth}\raggedright
\begin{quote}
55
\end{quote}
\end{minipage} \\
&
\multicolumn{7}{>{\raggedright\arraybackslash}p{(\columnwidth - 16\tabcolsep) * \real{0.7778} + 12\tabcolsep}}{%
\multirow{2}{*}{\textbf{CONCLUSION . . . . . . . . . . . . . . . . . . .
. . . . . . . . .}}} & \begin{minipage}[t]{\linewidth}\raggedright
\begin{quote}
\textbf{57}
\end{quote}
\end{minipage} \\
Dhia Abdelli & & & & & & & & \begin{minipage}[t]{\linewidth}\raggedright
\begin{quote}
Page 43
\end{quote}
\end{minipage} \\
\bottomrule()
\end{longtable}

\begin{quote}
RÉALISATION

\textbf{4.1 Inroduction}

Dans le présent chapitre, nous présenterons le travail accompli, en
mettant en lumière l'environnement propice à sa conception. De plus,
nous démontrerons les interfaces proposées par notre plateforme SOAR et
les scénarios que nous avons mis en place pour évaluer notre projet avec
précision et fiabilité.

\textbf{4.2 Environnement de travail}

Nous stipulons dans cette partie l'ensemble d'environnement matériel et
logiciel employés pour développer et réaliser notre projet .

\textbf{4.2.1} \textbf{Composants logiciels}

Dans la partie logicielle, nous avons choisi d'utiliser ces logiciels
spécifiés dans le tableau 4.1 afin de mettre en œuvre notre projet .

\textbf{TABLEAU 4.1 -- Environnement Logiciel du Travail}
\end{quote}

\begin{longtable}[]{@{}
  >{\raggedright\arraybackslash}p{(\columnwidth - 2\tabcolsep) * \real{0.5000}}
  >{\raggedright\arraybackslash}p{(\columnwidth - 2\tabcolsep) * \real{0.5000}}@{}}
\toprule()
\begin{minipage}[b]{\linewidth}\raggedright
\begin{quote}
\textbf{Outils}
\end{quote}
\end{minipage} & \begin{minipage}[b]{\linewidth}\raggedright
\begin{quote}
\textbf{Nom du logiciel}
\end{quote}
\end{minipage} \\
\midrule()
\endhead
\begin{minipage}[t]{\linewidth}\raggedright
\begin{quote}
SIEM
\end{quote}
\end{minipage} & \begin{minipage}[t]{\linewidth}\raggedright
\begin{quote}
ELK Stack
\end{quote}
\end{minipage} \\
\begin{minipage}[t]{\linewidth}\raggedright
\begin{quote}
Plateforme Honeypot
\end{quote}
\end{minipage} & \begin{minipage}[t]{\linewidth}\raggedright
\begin{quote}
T-POT
\end{quote}
\end{minipage} \\
Outil de collecte de données T-POT &
\begin{minipage}[t]{\linewidth}\raggedright
\begin{quote}
GreedyBear
\end{quote}
\end{minipage} \\
\begin{minipage}[t]{\linewidth}\raggedright
\begin{quote}
Outil de réponse aux incidents
\end{quote}
\end{minipage} & \begin{minipage}[t]{\linewidth}\raggedright
\begin{quote}
DFIR IRIS
\end{quote}
\end{minipage} \\
\begin{minipage}[t]{\linewidth}\raggedright
\begin{quote}
Outil d'orchestration
\end{quote}
\end{minipage} & \begin{minipage}[t]{\linewidth}\raggedright
\begin{quote}
Shuffle
\end{quote}
\end{minipage} \\
\begin{minipage}[t]{\linewidth}\raggedright
\begin{quote}
Outil numérique d'investigation et de réponse aux incidents
\end{quote}
\end{minipage} & \begin{minipage}[t]{\linewidth}\raggedright
\begin{quote}
Velociraptor
\end{quote}
\end{minipage} \\
\begin{minipage}[t]{\linewidth}\raggedright
\begin{longtable}[]{@{}
  >{\raggedright\arraybackslash}p{(\columnwidth - 6\tabcolsep) * \real{0.2500}}
  >{\raggedright\arraybackslash}p{(\columnwidth - 6\tabcolsep) * \real{0.2500}}
  >{\raggedright\arraybackslash}p{(\columnwidth - 6\tabcolsep) * \real{0.2500}}
  >{\raggedright\arraybackslash}p{(\columnwidth - 6\tabcolsep) * \real{0.2500}}@{}}
\toprule()
\begin{minipage}[b]{\linewidth}\raggedright
Outil
\end{minipage} & \begin{minipage}[b]{\linewidth}\raggedright
d'analyse
\end{minipage} &
\multirow{2}{*}{\begin{minipage}[b]{\linewidth}\raggedright
des
\end{minipage}} &
\multirow{2}{*}{\begin{minipage}[b]{\linewidth}\raggedright
logiciels
\end{minipage}} \\
\multicolumn{2}{@{}>{\raggedright\arraybackslash}p{(\columnwidth - 6\tabcolsep) * \real{0.5000} + 2\tabcolsep}}{%
\begin{minipage}[b]{\linewidth}\raggedright
\begin{quote}
malveillants
\end{quote}
\end{minipage}} \\
\midrule()
\endhead
\bottomrule()
\end{longtable}
\end{minipage} & \begin{minipage}[t]{\linewidth}\raggedright
\begin{quote}
Cuckoo Sanbox
\end{quote}
\end{minipage} \\
\begin{minipage}[t]{\linewidth}\raggedright
\begin{quote}
Outil d'alerte
\end{quote}
\end{minipage} & \begin{minipage}[t]{\linewidth}\raggedright
\begin{quote}
Praeco
\end{quote}
\end{minipage} \\
\begin{minipage}[t]{\linewidth}\raggedright
\begin{quote}
système de virtualisation
\end{quote}
\end{minipage} & \begin{minipage}[t]{\linewidth}\raggedright
\begin{quote}
Hyper-V
\end{quote}
\end{minipage} \\
\begin{minipage}[t]{\linewidth}\raggedright
\begin{quote}
Pare-feu
\end{quote}
\end{minipage} & \begin{minipage}[t]{\linewidth}\raggedright
\begin{quote}
Sophos
\end{quote}
\end{minipage} \\
\begin{minipage}[t]{\linewidth}\raggedright
\begin{quote}
Collecteur de journaux
\end{quote}
\end{minipage} & \begin{minipage}[t]{\linewidth}\raggedright
\begin{quote}
Filebeat
\end{quote}
\end{minipage} \\
\bottomrule()
\end{longtable}

\begin{longtable}[]{@{}
  >{\raggedright\arraybackslash}p{(\columnwidth - 2\tabcolsep) * \real{0.5000}}
  >{\raggedright\arraybackslash}p{(\columnwidth - 2\tabcolsep) * \real{0.5000}}@{}}
\toprule()
\begin{minipage}[b]{\linewidth}\raggedright
\begin{quote}
\textbf{4.2.2}
\end{quote}
\end{minipage} & \begin{minipage}[b]{\linewidth}\raggedright
\begin{quote}
\textbf{Composants matériels}
\end{quote}
\end{minipage} \\
\midrule()
\endhead
\bottomrule()
\end{longtable}

\begin{quote}
Les logiciels spécifiés dans le tableau suivant nécessitent des
équipements afin d'assurer le bon fonctionnement de la solution.Nous
avons utilisé les équipements illustrés dans le tableau suivant
\end{quote}

\begin{longtable}[]{@{}
  >{\raggedright\arraybackslash}p{(\columnwidth - 2\tabcolsep) * \real{0.5000}}
  >{\raggedright\arraybackslash}p{(\columnwidth - 2\tabcolsep) * \real{0.5000}}@{}}
\toprule()
\begin{minipage}[b]{\linewidth}\raggedright
\begin{quote}
Dhia Abdelli
\end{quote}
\end{minipage} & \begin{minipage}[b]{\linewidth}\raggedright
Page 44
\end{minipage} \\
\midrule()
\endhead
\bottomrule()
\end{longtable}

\begin{quote}
RÉALISATION
\end{quote}

\textbf{TABLEAU 4.2 -- Environnement Matériel du Travail}

\begin{longtable}[]{@{}
  >{\raggedright\arraybackslash}p{(\columnwidth - 2\tabcolsep) * \real{0.5000}}
  >{\raggedright\arraybackslash}p{(\columnwidth - 2\tabcolsep) * \real{0.5000}}@{}}
\toprule()
\begin{minipage}[b]{\linewidth}\raggedright
\begin{quote}
\textbf{Nom}
\end{quote}
\end{minipage} & \begin{minipage}[b]{\linewidth}\raggedright
\begin{quote}
Équipement
\end{quote}
\end{minipage} \\
\midrule()
\endhead
\begin{minipage}[t]{\linewidth}\raggedright
\begin{quote}
\textbf{ELK Stack + Praeco}
\end{quote}
\end{minipage} & \begin{minipage}[t]{\linewidth}\raggedright
\begin{quote}
Mémoire RAM : 32 GB\\
Nombre de processeur : 4 CPU\\
Espace disque : 600 GB\\
Système d'exploitation : Ubuntu Server 22.04 LTS Carte Réseaux : 1
\end{quote}\strut
\end{minipage} \\
\begin{minipage}[t]{\linewidth}\raggedright
\begin{quote}
\textbf{T-POT}
\end{quote}
\end{minipage} & \begin{minipage}[t]{\linewidth}\raggedright
\begin{quote}
Mémoire RAM : 16 GB\\
Nombre de processeur : 4 CPU\\
Espace disque : 300 GB\\
Système d'exploitation : Ubuntu Server 22.04 LTS Carte Réseaux : 1
\end{quote}\strut
\end{minipage} \\
\begin{minipage}[t]{\linewidth}\raggedright
\begin{quote}
\textbf{Velociraptor}
\end{quote}
\end{minipage} & \begin{minipage}[t]{\linewidth}\raggedright
\begin{quote}
Mémoire RAM : 4 GB\\
Nombre de processeur : 2 CPU\\
Espace disque : 100 GB\\
Système d'exploitation : Ubuntu Server 22.04 LTS Carte Réseaux : 1
\end{quote}\strut
\end{minipage} \\
\begin{minipage}[t]{\linewidth}\raggedright
\begin{quote}
\textbf{DFIR IRIS}
\end{quote}
\end{minipage} & \begin{minipage}[t]{\linewidth}\raggedright
\begin{quote}
Mémoire RAM : 4 GB\\
Nombre de processeur : 2 CPU\\
Espace disque : 300 GB\\
Système d'exploitation : Ubuntu Server 22.04 LTS Carte Réseaux : 1
\end{quote}\strut
\end{minipage} \\
\begin{minipage}[t]{\linewidth}\raggedright
\begin{quote}
\textbf{Misp}
\end{quote}
\end{minipage} & \begin{minipage}[t]{\linewidth}\raggedright
\begin{quote}
Mémoire RAM : 8 GB\\
Nombre de processeur : 4 CPU\\
Espace disque : 300 GB\\
Système d'exploitation : Ubuntu Server 22.04 LTS Carte Réseaux : 1
\end{quote}\strut
\end{minipage} \\
\begin{minipage}[t]{\linewidth}\raggedright
\begin{quote}
\textbf{Cuckoo Sanbox}
\end{quote}
\end{minipage} & \begin{minipage}[t]{\linewidth}\raggedright
\begin{quote}
Mémoire RAM : 16 GB\\
Nombre de processeur : 4 CPU\\
Espace disque : 350 GB\\
Système d'exploitation : Ubuntu Server 22.04 LTS Carte Réseaux : 1
\end{quote}\strut
\end{minipage} \\
\begin{minipage}[t]{\linewidth}\raggedright
\begin{quote}
\textbf{Greedybear}
\end{quote}
\end{minipage} & \begin{minipage}[t]{\linewidth}\raggedright
\begin{quote}
Mémoire RAM : 4 GB\\
Nombre de processeur : 2 CPU\\
Espace disque : 200 GB\\
Système d'exploitation : Ubuntu Server 22.04 LTS Carte Réseaux : 1
\end{quote}\strut
\end{minipage} \\
\begin{minipage}[t]{\linewidth}\raggedright
\begin{quote}
\textbf{Shufle}
\end{quote}
\end{minipage} & \begin{minipage}[t]{\linewidth}\raggedright
\begin{quote}
Mémoire RAM : 16 GB\\
Nombre de processeur : 8 CPU\\
Espace disque : 300 GB\\
Système d'exploitation : Ubuntu Server 22.04 LTS Carte Réseaux : 1
\end{quote}\strut
\end{minipage} \\
\begin{minipage}[t]{\linewidth}\raggedright
\begin{quote}
\textbf{Filebeat (syslog)}
\end{quote}
\end{minipage} & \begin{minipage}[t]{\linewidth}\raggedright
\begin{quote}
Mémoire RAM : 2 GB\\
Nombre de processeur : 2 CPU\\
Espace disque : 100 GB\\
Système d'exploitation : Ubuntu Server 22.04 LTS Carte Réseaux : 1
\end{quote}\strut
\end{minipage} \\
\bottomrule()
\end{longtable}

\begin{longtable}[]{@{}
  >{\raggedright\arraybackslash}p{(\columnwidth - 2\tabcolsep) * \real{0.5000}}
  >{\raggedright\arraybackslash}p{(\columnwidth - 2\tabcolsep) * \real{0.5000}}@{}}
\toprule()
\begin{minipage}[b]{\linewidth}\raggedright
\begin{quote}
Dhia Abdelli
\end{quote}
\end{minipage} & \begin{minipage}[b]{\linewidth}\raggedright
Page 45
\end{minipage} \\
\midrule()
\endhead
\bottomrule()
\end{longtable}

\begin{quote}
RÉALISATION

\textbf{4.3 La solution SOAR}

Dans cette partie, nous vous montrerons des captures d'écran pour
illustrer les différentes situations et utilisations courantes de notre
solution. Cela vous donnera une idée complète de son efficacité et de sa
polyvalence.

\textbf{4.3.1} \textbf{Interface des cas}

La page illustrée dans la figure ci-dessous présente une liste complète
des cas ouverts qui nécessitent une attention particulière. Les
analystes ont la possibilité d'accéder à une multitude d'informations
sur chaque cas sans avoir besoin de les parcourir individuellement. De
plus, ils peuvent accélérer le processus en utilisant une fonction de
filtrage rapide pour localiser et examiner rapidement des cas
spécifiques.
\end{quote}

\includegraphics[width=6.29861in,height=2.79444in]{vertopal_fa5b2edd966b48bfbc6a830f7abfbc35/media/image22.png}

\textbf{FIGURE 4.1 -- Liste des cas sur DFIR IRIS}

\begin{quote}
\textbf{4.3.2} \textbf{Interface d'alertes}

Cette page fournit un aperçu complet de toutes les alertes au sein de
notre solution. Il donne une image claire de l'état et des détails de
chaque alerte. La mise en page est organisée pour une compréhension
facile. Les analystes peuvent également prendre des mesures telles que
le fusionner avec un cas existant, le supprimer ou le transformer en
cas. Cela rend la gestion des alertes plus rapide et plus efficace.
\end{quote}

\begin{longtable}[]{@{}
  >{\raggedright\arraybackslash}p{(\columnwidth - 2\tabcolsep) * \real{0.5000}}
  >{\raggedright\arraybackslash}p{(\columnwidth - 2\tabcolsep) * \real{0.5000}}@{}}
\toprule()
\begin{minipage}[b]{\linewidth}\raggedright
\begin{quote}
Dhia Abdelli
\end{quote}
\end{minipage} & \begin{minipage}[b]{\linewidth}\raggedright
Page 46
\end{minipage} \\
\midrule()
\endhead
\bottomrule()
\end{longtable}

\begin{quote}
RÉALISATION
\end{quote}

\includegraphics[width=6.29861in,height=3.23611in]{vertopal_fa5b2edd966b48bfbc6a830f7abfbc35/media/image23.png}

\textbf{FIGURE 4.2 -- Liste des alertes sur DFIR IRIS}

\begin{quote}
\textbf{4.3.3} \textbf{Aperçu de cas}\\
Comme illustré dans la Figure 4.3, la page de prévisualisation offre un
aperçu complet du

cas, englobant des informations essentielles telles que :
\end{quote}

\begin{longtable}[]{@{}
  >{\raggedright\arraybackslash}p{(\columnwidth - 2\tabcolsep) * \real{0.5000}}
  >{\raggedright\arraybackslash}p{(\columnwidth - 2\tabcolsep) * \real{0.5000}}@{}}
\toprule()
\multicolumn{2}{@{}>{\raggedright\arraybackslash}p{(\columnwidth - 2\tabcolsep) * \real{1.0000} + 2\tabcolsep}@{}}{%
\begin{minipage}[b]{\linewidth}\raggedright
\begin{quote}
\emph{•} \textbf{Nom du Cas} : Il s'agit de l'identifiant spécifique
attribué au cas, facilitant la référence et l'organisation.

\emph{•} \textbf{Nom de l'organisation} : Il indique le nom de
l'organisation associé au cas, fournissant un contexte précieux pour la
résolution.

\emph{•} \textbf{Date du Cas} : Cela signifie la date à laquelle le cas
a été initié.

\emph{•} \textbf{Description Sommaire du Cas} : Cette section offre une
vue d'ensemble concise mais informative du cas, offrant un aperçu rapide
de sa nature et de son urgence.

\emph{•} \textbf{Alertes Associées} : Elle répertorie les alertes
éventuellement liées au cas, facilitant la compréhension du contexte et
de l'impact potentiel.

\emph{•} \textbf{Statut Actuel} : Cela indique l'état actuel du cas,
qu'il soit ouvert, en cours ou résolu, assurant une visibilité claire
sur son avancement.
\end{quote}
\end{minipage}} \\
\midrule()
\endhead
\begin{minipage}[t]{\linewidth}\raggedright
\begin{quote}
Dhia Abdelli
\end{quote}
\end{minipage} & Page 47 \\
\bottomrule()
\end{longtable}

\begin{quote}
RÉALISATION
\end{quote}

\includegraphics[width=6.29861in,height=2.49028in]{vertopal_fa5b2edd966b48bfbc6a830f7abfbc35/media/image24.png}

\textbf{FIGURE 4.3 -- Aperçu de cas sur DFIR IRIS}

\begin{quote}
Lorsque nous accédons à l'onglet IOC, cette page affiche les données de
l'indicateur de compromission (IOC) liées à toutes les alertes associées
au cas. Cela inclut des informations telles que les adresses IP, les
domaines, les hachages de fichiers et d'autres détails pertinents.
\end{quote}

\includegraphics[width=6.29861in,height=1.2125in]{vertopal_fa5b2edd966b48bfbc6a830f7abfbc35/media/image25.png}

\textbf{FIGURE 4.4 -- Liste des IoCs d'un cas}

\begin{quote}
L'onglet tâches permet aux analystes SOC de superviser et d'exécuter des
tâches désignées, englobant à la fois des affectations manuelles et des
processus automatisés. il garantit un flux opérationnel fluide,
renforçant la transparence et la responsabilisation des équipes.
\end{quote}

\includegraphics[width=6.29861in,height=1.44167in]{vertopal_fa5b2edd966b48bfbc6a830f7abfbc35/media/image26.png}

\textbf{FIGURE 4.5 -- Liste des tâches d'un cas}

\begin{quote}
\textbf{4.3.4} \textbf{Interface gestion des organisations}

La page illustrée dans la figure 4.6 est la page de gestion des
organisations, accessible uniquement par l'administrateur de la
solution. Les organisations sur notre solution sont isolées
\end{quote}

\begin{longtable}[]{@{}
  >{\raggedright\arraybackslash}p{(\columnwidth - 2\tabcolsep) * \real{0.5000}}
  >{\raggedright\arraybackslash}p{(\columnwidth - 2\tabcolsep) * \real{0.5000}}@{}}
\toprule()
\begin{minipage}[b]{\linewidth}\raggedright
\begin{quote}
Dhia Abdelli
\end{quote}
\end{minipage} & \begin{minipage}[b]{\linewidth}\raggedright
Page 48
\end{minipage} \\
\midrule()
\endhead
\bottomrule()
\end{longtable}

\begin{quote}
RÉALISATION
\end{quote}

pour garantir la confidentialité. Chaque client disposera de ses propres
données et les analystes

\begin{quote}
pourront surveiller les incidents de chaque client individuellement.
\end{quote}

\includegraphics[width=6.29861in,height=2.4in]{vertopal_fa5b2edd966b48bfbc6a830f7abfbc35/media/image27.png}

\textbf{FIGURE 4.6 -- Liste des organisations sur DFIR Iris}

\begin{quote}
\textbf{4.3.5} \textbf{Interface d'organisation}

Après avoir sélectionné une organisation spécifique dans la liste, nous
sommes dirigés
\end{quote}

vers la page d'aperçu, où une multitude de statistiques relatives à
l'organisation choisie sont

\begin{quote}
disponibles, telles que :

\emph{•} \textbf{Nombre total de cas} : nombre total de cas traités pour
l'organisation.

\emph{•} \textbf{Cas ouverts actuels} : nombre de cas actuellement non
résolus ou en cours.\emph{•} \textbf{Cas du mois en cours} : nombre de
cas enregistrés au cours du mois en cours.\emph{•} \textbf{Cas du mois
dernier} : nombre de cas enregistrés au cours du mois précédent.

De plus, la page donne accès à des informations de contact de
l'organisation.
\end{quote}

\includegraphics[width=6.29861in,height=1.99167in]{vertopal_fa5b2edd966b48bfbc6a830f7abfbc35/media/image28.png}

\textbf{FIGURE 4.7 -- Interface de l'organisation}

\begin{longtable}[]{@{}
  >{\raggedright\arraybackslash}p{(\columnwidth - 2\tabcolsep) * \real{0.5000}}
  >{\raggedright\arraybackslash}p{(\columnwidth - 2\tabcolsep) * \real{0.5000}}@{}}
\toprule()
\begin{minipage}[b]{\linewidth}\raggedright
\begin{quote}
Dhia Abdelli
\end{quote}
\end{minipage} & \begin{minipage}[b]{\linewidth}\raggedright
Page 49
\end{minipage} \\
\midrule()
\endhead
\bottomrule()
\end{longtable}

\begin{quote}
RÉALISATION
\end{quote}

\begin{longtable}[]{@{}
  >{\raggedright\arraybackslash}p{(\columnwidth - 2\tabcolsep) * \real{0.5000}}
  >{\raggedright\arraybackslash}p{(\columnwidth - 2\tabcolsep) * \real{0.5000}}@{}}
\toprule()
\begin{minipage}[b]{\linewidth}\raggedright
\begin{quote}
\textbf{4.3.6}
\end{quote}
\end{minipage} & \begin{minipage}[b]{\linewidth}\raggedright
\begin{quote}
\textbf{Interface de gestion des utilisateurs}
\end{quote}
\end{minipage} \\
\midrule()
\endhead
\bottomrule()
\end{longtable}

\begin{quote}
La page illustrée dans la figure 4.10 permet à l'administrateur de gérer
efficacement les comptes d'utilisateurs au sein de notre solution. Cela
inclut la possibilité de créer des comptes adaptés à des autorisations
et des rôles spécifiques, mais également de mener des audits approfondis
des activités des utilisateurs. Par exemple, l'administrateur peut créer
des comptes pour les analystes SOC et des comptes désignés pour
l'utilisation du service.

\includegraphics[width=6.29861in,height=2.57639in]{vertopal_fa5b2edd966b48bfbc6a830f7abfbc35/media/image29.png}
\end{quote}

\textbf{FIGURE 4.8 -- Liste des utilisateurs}

\begin{quote}
\textbf{4.3.7} \textbf{Interface de gestion des modules}

Au sein de notre solution, nous avons intégré tous les outils déployés
créant un environnement centralisé où diverses solutions collaborent
ensemble. Cette intégration rationalise considérablement le travail des
analystes SOC, en leur fournissant une interface unifiée. Sur cette
page, nous pouvons observer que les outils comme Misp, Velociprator et
Cortex ont été intégrés avec succès sur DFIR IRIS.

\includegraphics[width=6.29861in,height=1.68194in]{vertopal_fa5b2edd966b48bfbc6a830f7abfbc35/media/image30.png}
\end{quote}

\textbf{FIGURE 4.9 -- Liste des modules}

\begin{longtable}[]{@{}
  >{\raggedright\arraybackslash}p{(\columnwidth - 2\tabcolsep) * \real{0.5000}}
  >{\raggedright\arraybackslash}p{(\columnwidth - 2\tabcolsep) * \real{0.5000}}@{}}
\toprule()
\begin{minipage}[b]{\linewidth}\raggedright
\begin{quote}
Dhia Abdelli
\end{quote}
\end{minipage} & \begin{minipage}[b]{\linewidth}\raggedright
Page 50
\end{minipage} \\
\midrule()
\endhead
\bottomrule()
\end{longtable}

\begin{quote}
RÉALISATION
\end{quote}

\begin{longtable}[]{@{}
  >{\raggedright\arraybackslash}p{(\columnwidth - 2\tabcolsep) * \real{0.5000}}
  >{\raggedright\arraybackslash}p{(\columnwidth - 2\tabcolsep) * \real{0.5000}}@{}}
\toprule()
\begin{minipage}[b]{\linewidth}\raggedright
\begin{quote}
\textbf{4.3.8}
\end{quote}
\end{minipage} & \begin{minipage}[b]{\linewidth}\raggedright
\begin{quote}
\textbf{Intégration de Vélociraptor sur DFIR IRIS}
\end{quote}
\end{minipage} \\
\midrule()
\endhead
\bottomrule()
\end{longtable}

\begin{quote}
L'intégration présentée dans la figure ci-dessous permet aux analystes
SOC d'utiliser l'outil Velociraptor directement à partir de cette page.
Par exemple, ils peuvent rapidement isoler ou inverser le statut d'un
hôte spécifique ou même exécuter des artefacts d'investigation numérique
sans avoir besoin d'accéder au tableau de bord principal de l'outil.

\includegraphics[width=6.29861in,height=2.80694in]{vertopal_fa5b2edd966b48bfbc6a830f7abfbc35/media/image31.png}
\end{quote}

\textbf{FIGURE 4.10 -- Intégration Vélociraptor}

\begin{quote}
\textbf{4.3.9} \textbf{Liste des fichiers analysés}

La figure 4.11 affiche une liste des fichiers analysés dans notre outil
sandbox. Cette fonctionnalitéest importante pour voir la malveillance
d'un fichier sur notre solution. Comme illustré, l'outil fournit une
note pour chaque fichier, guidant nos actions en fonction de son
évaluation. Ce système de notation rationalise la prise de décision
concernant les menaces potentielles.

\includegraphics[width=6.29861in,height=1.91805in]{vertopal_fa5b2edd966b48bfbc6a830f7abfbc35/media/image32.png}

\textbf{FIGURE 4.11 -- Liste des fichiers analysés sur Cuckoo Sanbox}
\end{quote}

\begin{longtable}[]{@{}
  >{\raggedright\arraybackslash}p{(\columnwidth - 2\tabcolsep) * \real{0.5000}}
  >{\raggedright\arraybackslash}p{(\columnwidth - 2\tabcolsep) * \real{0.5000}}@{}}
\toprule()
\begin{minipage}[b]{\linewidth}\raggedright
\begin{quote}
Dhia Abdelli
\end{quote}
\end{minipage} & \begin{minipage}[b]{\linewidth}\raggedright
Page 51
\end{minipage} \\
\midrule()
\endhead
\bottomrule()
\end{longtable}

\begin{quote}
RÉALISATION
\end{quote}

\begin{longtable}[]{@{}
  >{\raggedright\arraybackslash}p{(\columnwidth - 2\tabcolsep) * \real{0.5000}}
  >{\raggedright\arraybackslash}p{(\columnwidth - 2\tabcolsep) * \real{0.5000}}@{}}
\toprule()
\begin{minipage}[b]{\linewidth}\raggedright
\begin{quote}
\textbf{4.3.10}
\end{quote}
\end{minipage} & \begin{minipage}[b]{\linewidth}\raggedright
\begin{quote}
\textbf{Aperçu du fichier analysé}
\end{quote}
\end{minipage} \\
\midrule()
\endhead
\bottomrule()
\end{longtable}

\begin{quote}
Dans cette page, nous avons accès à un ensemble complet d'informations
sur nos fichiers analysés dans Cuckoo Sandbox. Il nous fournit :

\emph{•} Un score indiquant la malveillance potentielle du fichier.

\emph{•} Captures d'écran capturant le comportement du fichier exécuté
dans notre sandbox.

\emph{•} Détails sur le trafic réseau, y compris les adresses IP et les
domaines contactés.\\
\emph{•} Et divers autres types d'options d'analyse pour approfondir le
comportement et les caractéristiques du fichier.

\includegraphics[width=6.29861in,height=3.45139in]{vertopal_fa5b2edd966b48bfbc6a830f7abfbc35/media/image33.png}
\end{quote}

\textbf{FIGURE 4.12 -- Résumé d'un fichier analysé}

\begin{quote}
Les captures d'écran affichées dans cette figure sont capturées dans
notre environnement isolé (Sandbox) pendant le processus d'exécution.
Ils fournissent un enregistrement visuel de chaque étape de l'exécution
du fichier, offrant des informations précieuses sur son comportement et
les menaces potentielles.

\includegraphics[width=6.29861in,height=1.38472in]{vertopal_fa5b2edd966b48bfbc6a830f7abfbc35/media/image34.png}

\textbf{FIGURE 4.13 -- Les Captures d'écran d'un fichier analysé}
\end{quote}

\begin{longtable}[]{@{}
  >{\raggedright\arraybackslash}p{(\columnwidth - 2\tabcolsep) * \real{0.5000}}
  >{\raggedright\arraybackslash}p{(\columnwidth - 2\tabcolsep) * \real{0.5000}}@{}}
\toprule()
\begin{minipage}[b]{\linewidth}\raggedright
\begin{quote}
Dhia Abdelli
\end{quote}
\end{minipage} & \begin{minipage}[b]{\linewidth}\raggedright
Page 52
\end{minipage} \\
\midrule()
\endhead
\bottomrule()
\end{longtable}

\begin{quote}
RÉALISATION
\end{quote}

\begin{longtable}[]{@{}
  >{\raggedright\arraybackslash}p{(\columnwidth - 2\tabcolsep) * \real{0.5000}}
  >{\raggedright\arraybackslash}p{(\columnwidth - 2\tabcolsep) * \real{0.5000}}@{}}
\toprule()
\begin{minipage}[b]{\linewidth}\raggedright
\begin{quote}
\textbf{4.3.11}
\end{quote}
\end{minipage} & \begin{minipage}[b]{\linewidth}\raggedright
\begin{quote}
\textbf{Interface du Honeypot}
\end{quote}
\end{minipage} \\
\midrule()
\endhead
\bottomrule()
\end{longtable}

\begin{quote}
La page affichée ci-dessous donne accès à des données complètes
provenant de notre Honeypot intentionnellement exposé, accessibles sur
Internet. Cette configuration stratégique sert à améliorer notre
plateforme de renseignement sur les menaces en permettant la collecte
d'adresses IPà partir de scanners et les tentatives de pirates
informatiques de pénétrer dans notre réseau.

Ces informations précieuses nous aident à comprendre les types
d'attaques utilisées et nous permettent de rassembler de nouveaux
fichiers potentiels que les pirates pourraient tenter de déployer sur
nos systèmes simulés.

\includegraphics[width=6.29861in,height=2.97778in]{vertopal_fa5b2edd966b48bfbc6a830f7abfbc35/media/image35.png}
\end{quote}

\textbf{FIGURE 4.14 -- Liste des flux de travail}

\begin{quote}
La figure 4.15 affiche une liste des noms d'utilisateur et des mots de
passe utilisés dans les attaques par force brute dirigées contre notre
pot de miel.

\includegraphics[width=6.29861in,height=3.03472in]{vertopal_fa5b2edd966b48bfbc6a830f7abfbc35/media/image36.png}

\textbf{FIGURE 4.15 -- Liste des mots de passe et noms d'utilisateur
collectés}
\end{quote}

\begin{longtable}[]{@{}
  >{\raggedright\arraybackslash}p{(\columnwidth - 2\tabcolsep) * \real{0.5000}}
  >{\raggedright\arraybackslash}p{(\columnwidth - 2\tabcolsep) * \real{0.5000}}@{}}
\toprule()
\begin{minipage}[b]{\linewidth}\raggedright
\begin{quote}
Dhia Abdelli
\end{quote}
\end{minipage} & \begin{minipage}[b]{\linewidth}\raggedright
Page 53
\end{minipage} \\
\midrule()
\endhead
\bottomrule()
\end{longtable}

\begin{quote}
RÉALISATION

Dans cette section, nous collectons différents types d'informations.
Nous suivons les pays qui tentent le plus d'accéder à notre honeypot,
identifions les services les plus ciblés et notons les ports spécifiques
utilisés. Par exemple, nous gardons un œil sur les clients SSH les plus
courants qui tentent d'accéder à notre service SSH.

\includegraphics[width=6.29861in,height=3.03056in]{vertopal_fa5b2edd966b48bfbc6a830f7abfbc35/media/image37.png}

\textbf{FIGURE 4.16 -- Liste des mots de passe et noms d'utilisateur
collectés}

\textbf{4.3.12 Interface des flux de travail}

L'interface illustrée ci-dessous affiche une liste des workflows
disponibles dans notre solution. Les analystes SOC ont la possibilité de
créer de nouveaux flux de travail ou de modifier ceux existants, en les
adaptant à leurs besoins spécifiques. Cette flexibilité garantit que les
analystes peuvent adapter et optimiser les flux de travail selon les
besoins pour relever efficacement les défis de sécurité.

\includegraphics[width=5.11806in,height=2.70278in]{vertopal_fa5b2edd966b48bfbc6a830f7abfbc35/media/image38.png}
\end{quote}

\textbf{FIGURE 4.17 -- Liste des flux de travail}

\begin{longtable}[]{@{}
  >{\raggedright\arraybackslash}p{(\columnwidth - 2\tabcolsep) * \real{0.5000}}
  >{\raggedright\arraybackslash}p{(\columnwidth - 2\tabcolsep) * \real{0.5000}}@{}}
\toprule()
\begin{minipage}[b]{\linewidth}\raggedright
\begin{quote}
Dhia Abdelli
\end{quote}
\end{minipage} & \begin{minipage}[b]{\linewidth}\raggedright
Page 54
\end{minipage} \\
\midrule()
\endhead
\bottomrule()
\end{longtable}

\begin{quote}
RÉALISATION
\end{quote}

\begin{longtable}[]{@{}
  >{\raggedright\arraybackslash}p{(\columnwidth - 2\tabcolsep) * \real{0.5000}}
  >{\raggedright\arraybackslash}p{(\columnwidth - 2\tabcolsep) * \real{0.5000}}@{}}
\toprule()
\begin{minipage}[b]{\linewidth}\raggedright
\begin{quote}
\textbf{4.3.13}
\end{quote}
\end{minipage} & \begin{minipage}[b]{\linewidth}\raggedright
\begin{quote}
\textbf{Scénarios de flux de travail réels}
\end{quote}
\end{minipage} \\
\midrule()
\endhead
\bottomrule()
\end{longtable}

\begin{quote}
Dans cette section, nous présenterons des flux de travail détaillés pour
démontrer comment notre orchestrateur exécute et gère les tâches. De
plus, nous illustrerons comment ces flux de travail jouent un rôle
central dans la réduction de la charge de travail des analystes.

\textbf{4.3.13.1} \textbf{Flux de travail numéro 1}

Le premier flux de travail illustre comment une application tente de se
connecter à une adresse IP en utilisant un port inhabituel. Les étapes
de ce workflow sont les suivantes :

1. La première étape consiste à recevoir une alerte de Wazuh via des
webhooks.

2. Une alerte est générée.

3. Cette alerte est ensuite élevée au statut de cas formel.

4. Les informations tels que les adresses IP, les domaines, les URL, les
hachages de fichiers, et toute autre information pertinente pouvant
indiquer une éventuelle compromission seront envoyés pour analyse à
l'aide de Cortex.

5. En cas d'indication positive de compromission de la part de
Cortex,ces inofmations seront ajoutés à la liste des des indicateurs de
compromission (IoCs) et nous récupérons l'ID de l'équipement de
Velociraptor.

6. L'équipement à l'origine de l'alerte est ajouté à la liste des
actifs.

7. L'hôte est mis en quarantaine.

8. La réponse active Wazuh mettra fin au processus et tentera de
supprimer le fichier. Une fois la réponse active réussie, l'hôte sera
sorti de quarantaine.

9. Enfin, une note est ajoutée pour l'analyste SOC, marquant la clôture
du cas.
\end{quote}

\includegraphics[width=6.29861in,height=3.525in]{vertopal_fa5b2edd966b48bfbc6a830f7abfbc35/media/image39.png}

\textbf{FIGURE 4.18 -- Scénario de connexion à l'aide d'un port
inhabituel}

\begin{quote}
Les figures ci-dessous montrent le cas créé grâce au processus de
Shuffle.
\end{quote}

\begin{longtable}[]{@{}
  >{\raggedright\arraybackslash}p{(\columnwidth - 2\tabcolsep) * \real{0.5000}}
  >{\raggedright\arraybackslash}p{(\columnwidth - 2\tabcolsep) * \real{0.5000}}@{}}
\toprule()
\begin{minipage}[b]{\linewidth}\raggedright
\begin{quote}
Dhia Abdelli
\end{quote}
\end{minipage} & \begin{minipage}[b]{\linewidth}\raggedright
Page 55
\end{minipage} \\
\midrule()
\endhead
\bottomrule()
\end{longtable}

\begin{quote}
RÉALISATION
\end{quote}

\includegraphics[width=6.29861in,height=2.81806in]{vertopal_fa5b2edd966b48bfbc6a830f7abfbc35/media/image40.png}

\textbf{FIGURE 4.19 -- Cas créé sur DFIR IRIS}

\begin{quote}
Les figures 4.21 montrent l'ajout de l'équipement à la liste des actifs
grâce au processus de Shuffle.
\end{quote}

\includegraphics[width=6.29861in,height=0.7875in]{vertopal_fa5b2edd966b48bfbc6a830f7abfbc35/media/image41.png}

\textbf{FIGURE 4.20 -- L'équipement a été ajouté à la liste des actifs}

\begin{quote}
Les figures 4.21 montrent l'ajout de IoCs grâce au processus de Shuffle.
\end{quote}

\includegraphics[width=6.29861in,height=0.75833in]{vertopal_fa5b2edd966b48bfbc6a830f7abfbc35/media/image42.png}

\textbf{FIGURE 4.21 -- Les IoCs a été ajouté à la liste}

\begin{quote}
\textbf{4.3.13.2} \textbf{Flux de travail numéro 2}

Le deuxième flux de travail illustré dans la figure 4.22 représente la
détection d'une tentative d'accès à notre pot de miel local. Cela
signifie un compromis potentiel de notre réseau et aide àidentifier
d'éventuelles activités de reconnaissance.
\end{quote}

\begin{longtable}[]{@{}
  >{\raggedright\arraybackslash}p{(\columnwidth - 2\tabcolsep) * \real{0.5000}}
  >{\raggedright\arraybackslash}p{(\columnwidth - 2\tabcolsep) * \real{0.5000}}@{}}
\toprule()
\begin{minipage}[b]{\linewidth}\raggedright
\begin{quote}
Dhia Abdelli
\end{quote}
\end{minipage} & \begin{minipage}[b]{\linewidth}\raggedright
Page 56
\end{minipage} \\
\midrule()
\endhead
\bottomrule()
\end{longtable}

\begin{quote}
RÉALISATION
\end{quote}

\includegraphics[width=6.29861in,height=3.49167in]{vertopal_fa5b2edd966b48bfbc6a830f7abfbc35/media/image43.png}

\textbf{FIGURE 4.22 -- Scénario de connexion à un honeypot local}

\begin{quote}
\textbf{4.4 CONCLUSION}

Dans ce chapitre, nous avons proposé une visite visuelle de notre
plateforme à travers une série de captures d'écran soigneusement
sélectionnées. Ces images offrent une vue complète de l'interface
utilisateur et des fonctionnalités.
\end{quote}

\begin{longtable}[]{@{}
  >{\raggedright\arraybackslash}p{(\columnwidth - 2\tabcolsep) * \real{0.5000}}
  >{\raggedright\arraybackslash}p{(\columnwidth - 2\tabcolsep) * \real{0.5000}}@{}}
\toprule()
\begin{minipage}[b]{\linewidth}\raggedright
\begin{quote}
Dhia Abdelli
\end{quote}
\end{minipage} & \begin{minipage}[b]{\linewidth}\raggedright
Page 57
\end{minipage} \\
\midrule()
\endhead
\bottomrule()
\end{longtable}

\begin{quote}
\textbf{CONCLUSION GÉNÉRALE}

Ce rapport offre un compte rendu complet de notre stage de projet de fin
d'études, au cours duquel nous avons déployé avec succès la solution
SOAR-as-a-service chez Advancia. Cette réalisation s'inscrit dans notre
effort pour l'obtention du diplôme national d'ingénieur en informatique.

Pour concrétiser notre objectif, nous avons entamé le processus en
étroite collaboration avec notre encadrant professionnel. Cette phase
initiale a été cruciale pour définir avec précision les solutions que
nous allions cibler. En parallèle, nous avons dédié un temps
significatif àdes études théoriques approfondies. Celles-ci ont porté
sur le fonctionnement global d'une plateforme SOAR, ainsi que sur son
rôle vital au sein d'un SOC. Cette phase préliminaire de recherche et
d'analyse nous a fourni une solide base de compréhension, nous
permettant d'appréhender pleinement les enjeux et les exigences liés à
la mise en place d'une solution SOAR performante.

Ce projet a été une aventure éducative majeure dans le domaine de la
sécurité informatique, apportant de nouvelles connaissances et
compétences essentielles pour le début de notre carrière.

Il va bien au-delà d'une simple tâche, offrant une expérience immersive
où chaque étape a consolidé nos bases professionnelles. Outre notre
projet principal, nous avons également eu l'opportunité d'explorer
d'autres facettes de la sécurité réseau, endossant des rôles de
consultants en sécurité. Cette expérience a été une opportunité de
développer des relations professionnelles fructueuses avec notre équipe,
combinant professionnalisme et camaraderie respectueuse. En définitive,
cette période d'enrichissement et de croissance professionnelle restera
une part indélébile de notre parcours.

Après des efforts soutenus et une mise en œuvre minutieuse, notre projet
a pleinement atteint ses objectifs. Le déploiement de la plateforme SOAR
s'est déroulé avec succès et les tests ainsi que les simulations
d'attaques ont validé son efficacité. Nous sommes également ravis de
souligner que notre plateforme repose sur des solutions open source, ce
qui signifie que nous avons la possibilité d'incorporer de nouvelles
fonctionnalités et d'adapter notre système en fonction des besoins qui
se présenteront à l'avenir. Cette flexibilité inhérente à notre
architecture témoigne de notre engagement envers l'innovation continue
et l'amélioration constante de nos capacités de sécurité.

Enfin, nous espérons que notre travail constituera une source
d'inspiration et de progrès pour Advancia, ainsi qu'une satisfaction
pour les responsables et membres du jury. Nous sommes impatients de
partager nos résultats avec tous ceux intéressés par la sécurité
informatique. En somme, nous visons à laisser une empreinte positive et
durable dans ce domaine.
\end{quote}

\begin{longtable}[]{@{}
  >{\raggedright\arraybackslash}p{(\columnwidth - 2\tabcolsep) * \real{0.5000}}
  >{\raggedright\arraybackslash}p{(\columnwidth - 2\tabcolsep) * \real{0.5000}}@{}}
\toprule()
\begin{minipage}[b]{\linewidth}\raggedright
\begin{quote}
Dhia Abdelli
\end{quote}
\end{minipage} & \begin{minipage}[b]{\linewidth}\raggedright
Page 58
\end{minipage} \\
\midrule()
\endhead
\bottomrule()
\end{longtable}

\begin{quote}
\textbf{BIBLIOGRAPHIE}

{[}1{]} Site de la société Advancia IT System. {[}En ligne{]}\\
https ://www.advancia-itsystem.com/

{[}2{]} Méthode Agile et méthode classique. {[}En ligne{]}\\
https
://www.axiocode.com/methode-agile-vs-classique-quelle-methode-utiliser/

{[}3{]} Méthodologie Kanban. {[}En ligne{]} :\\
https
://www.journaldunet.fr/web-tech/guide-de-l-entreprise-digitale/1443832-kanban/

{[}4{]} La triade CIA. {[}En ligne{]} :\\
https ://www.jedha.co/blog/cybersecurite-quest-ce-que-la-triade-cia

{[}5{]} SOAR. {[}En ligne{]} :\\
https ://www.sekoia.io/fr/glossaire/qu-est-ce-qu-un-soar/

{[}6{]} SOAR Market. {[}En ligne{]} :\\
https
://www.marketsandmarkets.com/Market-Reports/security-orchestration-automation-response-market
\end{quote}

\begin{longtable}[]{@{}
  >{\raggedright\arraybackslash}p{(\columnwidth - 2\tabcolsep) * \real{0.5000}}
  >{\raggedright\arraybackslash}p{(\columnwidth - 2\tabcolsep) * \real{0.5000}}@{}}
\toprule()
\begin{minipage}[b]{\linewidth}\raggedright
\begin{quote}
Dhia Abdelli
\end{quote}
\end{minipage} & \begin{minipage}[b]{\linewidth}\raggedright
Page 59
\end{minipage} \\
\midrule()
\endhead
\bottomrule()
\end{longtable}

\begin{quote}
\textbf{ANNEXES}

\textbf{A.0.1 Misp}

\textbf{A.0.1.1} \textbf{Installation}

Pour débuter, nous avons d'abord téléchargé le script d'installation,
tel qu'indiqué dans la figure ci-dessous. Ensuite, lors du lancement du
script, il installera tous les packages et
\end{quote}

\includegraphics[width=6.29861in,height=1.41389in]{vertopal_fa5b2edd966b48bfbc6a830f7abfbc35/media/image44.png}

\textbf{FIGURE A.1 -- Téléchargement du script d'installation}

\begin{quote}
dépendances nécessaires et créera la base de données MISP où tous les
feeds seront stockés, comme le montre la figure A.2,
\end{quote}

\includegraphics[width=6.29861in,height=2.71805in]{vertopal_fa5b2edd966b48bfbc6a830f7abfbc35/media/image45.png}

\textbf{FIGURE A.2 -- Installation MISP}

\begin{longtable}[]{@{}
  >{\raggedright\arraybackslash}p{(\columnwidth - 2\tabcolsep) * \real{0.5000}}
  >{\raggedright\arraybackslash}p{(\columnwidth - 2\tabcolsep) * \real{0.5000}}@{}}
\toprule()
\begin{minipage}[b]{\linewidth}\raggedright
\begin{quote}
Dhia Abdelli
\end{quote}
\end{minipage} & \begin{minipage}[b]{\linewidth}\raggedright
Page 60
\end{minipage} \\
\midrule()
\endhead
\bottomrule()
\end{longtable}

\begin{quote}
ANNEXES

Une fois l'installation terminée et MISP en cours d'exécution, nous
pouvons accéder àl'interface web en utilisant l'URL suivante : https
://@IP
\end{quote}

\includegraphics[width=6.29861in,height=2.46667in]{vertopal_fa5b2edd966b48bfbc6a830f7abfbc35/media/image46.png}

\textbf{FIGURE A.3 -- Page d'Authentification de MISP}

\begin{quote}
\textbf{A.0.1.2} \textbf{Importer des flux de données}

Pour Importer les flux, nous cliquons sur "Sync Actions \textgreater{}
List Feeds \textgreater{} Import Feeds from JSON". Ensuite, nous avons
copié les métadonnées des flux par défaut à partir du dépôt GitHub
(https ://github.com/MISP/MISP) et les avons collées dans le champ de
saisie.Enfin, nous cliquons sur le bouton "Add".
\end{quote}

\includegraphics[width=5.66944in,height=2.56805in]{vertopal_fa5b2edd966b48bfbc6a830f7abfbc35/media/image47.png}

\textbf{FIGURE A.4 -- MISP flux de données}

\begin{quote}
\textbf{A.0.1.3} \textbf{Activation des flux}

Pour activer ces flux, nous cliquons sur \textbf{«Fetch and store all
feed data»} dans la page de la "List Feeds".
\end{quote}

\begin{longtable}[]{@{}
  >{\raggedright\arraybackslash}p{(\columnwidth - 2\tabcolsep) * \real{0.5000}}
  >{\raggedright\arraybackslash}p{(\columnwidth - 2\tabcolsep) * \real{0.5000}}@{}}
\toprule()
\begin{minipage}[b]{\linewidth}\raggedright
\begin{quote}
Dhia Abdelli
\end{quote}
\end{minipage} & \begin{minipage}[b]{\linewidth}\raggedright
Page 61
\end{minipage} \\
\midrule()
\endhead
\bottomrule()
\end{longtable}

\begin{quote}
ANNEXES

\includegraphics[width=5.66944in,height=3.025in]{vertopal_fa5b2edd966b48bfbc6a830f7abfbc35/media/image48.png}
\end{quote}

\textbf{FIGURE A.5 -- MISP Activation des flux}

\begin{quote}
\textbf{A.0.2 Shuffle}

\textbf{A.0.2.1} \textbf{Installation}

Pour commencer, nous commençons par cloner le répertoire d'installation
de Shuffle à partir de GitHub. (voir A.6)
\end{quote}

\includegraphics[width=6.92917in,height=1.45833in]{vertopal_fa5b2edd966b48bfbc6a830f7abfbc35/media/image49.png}

\begin{quote}
\textbf{FIGURE A.6 -- Clonage du Projet Shuffle depuis GitHub}

Une fois que le clonage est terminé, nous allons maintenant créer le
dossier de base de données et l'assigner à l'utilisateur et au groupe
avec l'ID 1000.(voir A.9)
\end{quote}

\includegraphics[width=6.92917in,height=1.10278in]{vertopal_fa5b2edd966b48bfbc6a830f7abfbc35/media/image50.png}

\begin{quote}
\textbf{FIGURE A.7 -- Prérequis pour la Base de Données OpenSearch}
\end{quote}

\begin{longtable}[]{@{}
  >{\raggedright\arraybackslash}p{(\columnwidth - 2\tabcolsep) * \real{0.5000}}
  >{\raggedright\arraybackslash}p{(\columnwidth - 2\tabcolsep) * \real{0.5000}}@{}}
\toprule()
\begin{minipage}[b]{\linewidth}\raggedright
\begin{quote}
Dhia Abdelli
\end{quote}
\end{minipage} & \begin{minipage}[b]{\linewidth}\raggedright
Page 62
\end{minipage} \\
\midrule()
\endhead
\bottomrule()
\end{longtable}

\begin{quote}
ANNEXES

Ensuite, nous exécuterons la commande \textbf{« docker-compose up -d »},
comme indiqué dans la figure ci-dessous. Cela automatisera le
déploiement de "Shuffle" au sein de conteneurs isolés. Une fois terminé,
nous pourrons accéder à l'interface web à l'adresse https ://@IP :3443
Enfin,

\includegraphics[width=5.04028in,height=1.99722in]{vertopal_fa5b2edd966b48bfbc6a830f7abfbc35/media/image51.png}
\end{quote}

\textbf{FIGURE A.8 -- Déploiement de Shuffle}

dès que nous accéderons à l'interface web, 'Shuffle' nous invitera à
créer notre compte administrateur.

\begin{quote}
\includegraphics[width=3.77917in,height=2.45833in]{vertopal_fa5b2edd966b48bfbc6a830f7abfbc35/media/image52.png}
\end{quote}

\textbf{FIGURE A.9 -- Shuffle création de compte}

\begin{quote}
\textbf{A.0.3 DFIR Iris}

\textbf{A.0.3.1} \textbf{Installation}

Pour commencer, nous commençons par cloner le répertoire d'installation
de DFIR IRIS (voir figure A.12.)
\end{quote}

\begin{longtable}[]{@{}
  >{\raggedright\arraybackslash}p{(\columnwidth - 2\tabcolsep) * \real{0.5000}}
  >{\raggedright\arraybackslash}p{(\columnwidth - 2\tabcolsep) * \real{0.5000}}@{}}
\toprule()
\begin{minipage}[b]{\linewidth}\raggedright
\begin{quote}
Dhia Abdelli
\end{quote}
\end{minipage} & \begin{minipage}[b]{\linewidth}\raggedright
Page 63
\end{minipage} \\
\midrule()
\endhead
\bottomrule()
\end{longtable}

\begin{quote}
ANNEXES

\includegraphics[width=6.29861in,height=1.67917in]{vertopal_fa5b2edd966b48bfbc6a830f7abfbc35/media/image53.png}

\textbf{FIGURE A.10 -- Clonage du Projet DFIR IRIS depuis GitHub}

Comme illustré dans la figure ci-dessous, nous allons exécuter la
commande \textbf{« docker-compose up -d »}, qui automatisera le
déploiement de "DFIR IRIS" au sein de conteneurs isolés.

\includegraphics[width=6.29861in,height=2.73889in]{vertopal_fa5b2edd966b48bfbc6a830f7abfbc35/media/image54.png}
\end{quote}

\textbf{FIGURE A.11 -- Déploiement de DFIR IRIS}

\begin{quote}
Une fois terminé, nous pourrons accéder à l'interface web à l'adresse
https ://@IP en utilisant les identifiants générés.

\includegraphics[width=3.14861in,height=1.3625in]{vertopal_fa5b2edd966b48bfbc6a830f7abfbc35/media/image55.png}
\end{quote}

\textbf{FIGURE A.12 -- Déploiement de DFIR IRIS}

\begin{longtable}[]{@{}
  >{\raggedright\arraybackslash}p{(\columnwidth - 2\tabcolsep) * \real{0.5000}}
  >{\raggedright\arraybackslash}p{(\columnwidth - 2\tabcolsep) * \real{0.5000}}@{}}
\toprule()
\begin{minipage}[b]{\linewidth}\raggedright
\begin{quote}
Dhia Abdelli
\end{quote}
\end{minipage} & \begin{minipage}[b]{\linewidth}\raggedright
Page 64
\end{minipage} \\
\midrule()
\endhead
\bottomrule()
\end{longtable}

\begin{quote}
ANNEXES
\end{quote}

\begin{longtable}[]{@{}
  >{\raggedright\arraybackslash}p{(\columnwidth - 2\tabcolsep) * \real{0.5000}}
  >{\raggedright\arraybackslash}p{(\columnwidth - 2\tabcolsep) * \real{0.5000}}@{}}
\toprule()
\begin{minipage}[b]{\linewidth}\raggedright
\textbf{A.0.4}\\
\textbf{A.0.4.1}\strut
\end{minipage} & \begin{minipage}[b]{\linewidth}\raggedright
\begin{quote}
\textbf{ELK Stack}\\
\textbf{Installation ElasicSearch}
\end{quote}\strut
\end{minipage} \\
\midrule()
\endhead
\bottomrule()
\end{longtable}

\begin{quote}
\textbf{T}out d'abord, nous procéderons au téléchargement et à
l'installation de la clé de signature publique d'Elastic

\includegraphics[width=6.29861in,height=0.45in]{vertopal_fa5b2edd966b48bfbc6a830f7abfbc35/media/image56.png}

\textbf{FIGURE A.13 -- l'installation de la clé de publique d'Elastic}

Nous allons maintenant procéder au téléchargement du package
elasticSearch à partir du site officiel en utilisant la commande
\textbf{« Wget »}.

\includegraphics[width=6.29861in,height=1.66528in]{vertopal_fa5b2edd966b48bfbc6a830f7abfbc35/media/image57.png}

\textbf{FIGURE A.14 -- Téléchargement du package elasticSearch}

Nous débutons l'installation d'Elasticsearch en utilisant la commande
\textbf{« dpkg »}.

\includegraphics[width=6.29861in,height=1.02083in]{vertopal_fa5b2edd966b48bfbc6a830f7abfbc35/media/image58.png}

\textbf{FIGURE A.15 -- Installation du package elasticSearch}

Une fois l'installation terminée, nous devrons apporter quelques
ajustements dans le fichier "/etc/elasticsearch/elasticsearch.yml" (voir
figure A.16)

\includegraphics[width=6.29861in,height=0.92361in]{vertopal_fa5b2edd966b48bfbc6a830f7abfbc35/media/image59.png}

\textbf{FIGURE A.16 -- Fichier configuration elasticSearch}

Une fois que tout est configuré, nous pouvons essayer d'accéder à
Elasticsearch sur https ://@IP :9200
\end{quote}

\begin{longtable}[]{@{}
  >{\raggedright\arraybackslash}p{(\columnwidth - 2\tabcolsep) * \real{0.5000}}
  >{\raggedright\arraybackslash}p{(\columnwidth - 2\tabcolsep) * \real{0.5000}}@{}}
\toprule()
\begin{minipage}[b]{\linewidth}\raggedright
\begin{quote}
Dhia Abdelli
\end{quote}
\end{minipage} & \begin{minipage}[b]{\linewidth}\raggedright
Page 65
\end{minipage} \\
\midrule()
\endhead
\bottomrule()
\end{longtable}

\begin{quote}
ANNEXES
\end{quote}

\includegraphics[width=4.40972in,height=3.36667in]{vertopal_fa5b2edd966b48bfbc6a830f7abfbc35/media/image60.png}

\textbf{FIGURE A.17 -- Interface Web elasticSearch}

\begin{quote}
\textbf{A.0.4.2} \textbf{Installation Kibana}

Pour commencer, nous allons récupérer le package Kibana à l'aide de la
commande \textbf{«Wget»}.
\end{quote}

\includegraphics[width=6.29861in,height=1.35139in]{vertopal_fa5b2edd966b48bfbc6a830f7abfbc35/media/image61.png}

\textbf{FIGURE A.18 -- Téléchargement du package Kibana}

\begin{quote}
Dans la figure A.19, nous procéderons à l'installation de Kibana
\end{quote}

\includegraphics[width=6.29861in,height=1.19028in]{vertopal_fa5b2edd966b48bfbc6a830f7abfbc35/media/image62.png}

\textbf{FIGURE A.19 -- Installation du package Kibana}

\begin{quote}
Pour sécuriser Kibana, nous allons ajouter notre certificat auto-signé
dans le fichier de configuration et configurer Kibana pour écouter sur
toutes les interfaces réseau et non seulement sur localhost. (voir
figure A.22)
\end{quote}

\begin{longtable}[]{@{}
  >{\raggedright\arraybackslash}p{(\columnwidth - 2\tabcolsep) * \real{0.5000}}
  >{\raggedright\arraybackslash}p{(\columnwidth - 2\tabcolsep) * \real{0.5000}}@{}}
\toprule()
\begin{minipage}[b]{\linewidth}\raggedright
\begin{quote}
Dhia Abdelli
\end{quote}
\end{minipage} & \begin{minipage}[b]{\linewidth}\raggedright
Page 66
\end{minipage} \\
\midrule()
\endhead
\bottomrule()
\end{longtable}

\begin{quote}
ANNEXES
\end{quote}

\includegraphics[width=6.29861in,height=1.27361in]{vertopal_fa5b2edd966b48bfbc6a830f7abfbc35/media/image63.png}

\textbf{FIGURE A.20 -- Installation du package Kibana}

\begin{quote}
Maintenant, nous pouvons accéder à Kibana depuis notre navigateur en
utilisant l'adresse http ://@IP :5601.
\end{quote}

\includegraphics[width=4.40972in,height=2.11805in]{vertopal_fa5b2edd966b48bfbc6a830f7abfbc35/media/image64.png}

\textbf{FIGURE A.21 -- Interface Kibana}

\begin{quote}
\textbf{A.0.4.3} \textbf{Configuration Logstash}

Comme le montre la figure ci-dessous, une fois Logstash installé, il est
nécessaire de modifier le fichier de configuration
(/etc/logstash/conf.d/logstash.conf) pour envoyer tous les journaux vers
Elasticsearch. Pour ce faire, nous devons ajouter notre serveur dans les
hôtes et intégrer les identifiants du serveur.
\end{quote}

\includegraphics[width=4.40972in,height=1.97778in]{vertopal_fa5b2edd966b48bfbc6a830f7abfbc35/media/image65.png}

\textbf{FIGURE A.22 -- Fichier de configuration logstash}

\begin{longtable}[]{@{}
  >{\raggedright\arraybackslash}p{(\columnwidth - 2\tabcolsep) * \real{0.5000}}
  >{\raggedright\arraybackslash}p{(\columnwidth - 2\tabcolsep) * \real{0.5000}}@{}}
\toprule()
\begin{minipage}[b]{\linewidth}\raggedright
\begin{quote}
Dhia Abdelli
\end{quote}
\end{minipage} & \begin{minipage}[b]{\linewidth}\raggedright
Page 67
\end{minipage} \\
\midrule()
\endhead
\bottomrule()
\end{longtable}

\includegraphics[width=8.28333in,height=11.67361in]{vertopal_fa5b2edd966b48bfbc6a830f7abfbc35/media/image66.png}

\begin{quote}
ANNEXES
\end{quote}

\begin{longtable}[]{@{}
  >{\raggedright\arraybackslash}p{(\columnwidth - 2\tabcolsep) * \real{0.5000}}
  >{\raggedright\arraybackslash}p{(\columnwidth - 2\tabcolsep) * \real{0.5000}}@{}}
\toprule()
\begin{minipage}[b]{\linewidth}\raggedright
\begin{quote}
Dhia Abdelli
\end{quote}
\end{minipage} & \begin{minipage}[b]{\linewidth}\raggedright
Page 69
\end{minipage} \\
\midrule()
\endhead
\bottomrule()
\end{longtable}

\end{document}
