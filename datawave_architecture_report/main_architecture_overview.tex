\documentclass[12pt,a4paper]{article}
\usepackage[utf8]{inputenc}
\usepackage[english]{babel}
\usepackage{geometry}
\usepackage{graphicx}
\usepackage{amsmath}
\usepackage{amsfonts}
\usepackage{amssymb}
\usepackage{hyperref}
\usepackage{listings}
\usepackage{xcolor}
\usepackage{fancyhdr}
\usepackage{tocloft}
\usepackage{titlesec}
\usepackage{minted}
\usepackage{float}

% Mermaid diagram support
\usepackage{tikz}
\usetikzlibrary{shapes.geometric, arrows}

% Page setup
\geometry{margin=1in}
\pagestyle{fancy}
\fancyhf{}
\rhead{DataWave Architecture Report}
\lhead{Enterprise Data Governance System}
\cfoot{\thepage}

% Title formatting
\titleformat{\section}{\Large\bfseries}{\thesection}{1em}{}
\titleformat{\subsection}{\large\bfseries}{\thesubsection}{1em}{}

\title{\textbf{DataWave Enterprise Data Governance System\\
Advanced Software Architecture Report}}

\author{
\textbf{Architecture Team}\\
Advanced Data Governance Engineering\\
Production-Ready Enterprise System
}

\date{\today}

\begin{document}

\maketitle

\newpage

\tableofcontents

\newpage

\section{Executive Summary}

This comprehensive architecture report presents the complete software architecture design for the DataWave Enterprise Data Governance System, a production-ready, large-scale data management platform designed to handle enterprise-level data governance, compliance, security, and intelligent data discovery across multiple data sources and cloud environments.

\subsection{System Overview}

DataWave represents a sophisticated, microservices-based data governance platform architected to provide:

\begin{itemize}
    \item \textbf{Comprehensive Data Catalog Management}: Advanced metadata discovery, classification, and lineage tracking
    \item \textbf{Intelligent Scanning \& Discovery}: AI-powered data source scanning with advanced pattern recognition
    \item \textbf{Enterprise Security \& Compliance}: Role-based access control (RBAC), compliance rule management, and audit trails
    \item \textbf{Advanced Analytics \& ML Integration}: Machine learning-driven insights, recommendations, and predictive analytics
    \item \textbf{Multi-Cloud Integration}: Support for Azure Purview, AWS, Google Cloud, and on-premises data sources
    \item \textbf{Real-time Collaboration}: Advanced workflow management, notifications, and team collaboration features
\end{itemize}

\subsection{Architectural Philosophy}

The DataWave system adopts a \textbf{Domain-Driven Design (DDD)} approach combined with \textbf{Clean Architecture} principles, ensuring:

\begin{enumerate}
    \item \textbf{Separation of Concerns}: Clear boundaries between business logic, data access, and presentation layers
    \item \textbf{Scalability}: Horizontal scaling capabilities through microservices architecture
    \item \textbf{Maintainability}: Modular design with well-defined interfaces and dependencies
    \item \textbf{Extensibility}: Plugin-based architecture for custom integrations and extensions
    \item \textbf{Security-First}: Zero-trust security model with comprehensive access controls
    \item \textbf{Performance}: Optimized for high-throughput data processing and real-time operations
\end{enumerate}

\section{Core Architecture Decision Rationale}

\subsection{Why This Architecture?}

The DataWave architecture was specifically designed to address the complex challenges of enterprise data governance:

\subsubsection{Microservices Architecture Selection}

We chose a microservices architecture over monolithic design for several critical reasons:

\begin{itemize}
    \item \textbf{Scalability Requirements}: Different components (scanning, catalog, compliance) have varying load patterns
    \item \textbf{Technology Diversity}: Different services can use optimal technologies (Python for ML, Node.js for real-time features)
    \item \textbf{Team Autonomy}: Independent development and deployment cycles for different functional areas
    \item \textbf{Fault Isolation}: Service failures don't cascade across the entire system
    \item \textbf{Compliance Boundaries}: Separate services for different compliance domains (PCI, HIPAA, GDPR)
\end{itemize}

\subsubsection{Event-Driven Architecture Integration}

The system incorporates event-driven patterns to handle:

\begin{itemize}
    \item \textbf{Real-time Data Discovery}: Immediate processing of new data sources
    \item \textbf{Compliance Monitoring}: Instant alerts for policy violations
    \item \textbf{Workflow Orchestration}: Complex multi-step data governance processes
    \item \textbf{Audit Trail Generation}: Comprehensive logging of all system activities
\end{itemize}

\subsection{Core Management Strategy}

The architecture implements a sophisticated management strategy based on:

\subsubsection{Centralized Configuration Management}
\begin{itemize}
    \item Configuration as Code approach
    \item Environment-specific configurations
    \item Dynamic configuration updates without service restarts
\end{itemize}

\subsubsection{Advanced Monitoring \& Observability}
\begin{itemize}
    \item Distributed tracing across all services
    \item Comprehensive metrics collection
    \item Intelligent alerting and anomaly detection
    \item Performance optimization through data-driven insights
\end{itemize}

\subsubsection{Security-First Design}
\begin{itemize}
    \item Zero-trust security model
    \item Multi-layer authentication and authorization
    \item End-to-end encryption for data in transit and at rest
    \item Comprehensive audit logging
\end{itemize}

\section{System Architecture Sections Overview}

This report is organized into the following comprehensive sections, each providing detailed architectural analysis with advanced diagrams and implementation specifications:

\subsection{Section 1: Data Model Architecture}
\textbf{File}: \texttt{01\_data\_model\_architecture.tex}
\begin{itemize}
    \item Complete entity relationship modeling
    \item Advanced database schema design
    \item Data classification and lineage models
    \item Performance optimization strategies
\end{itemize}

\subsection{Section 2: Service Layer Architecture}
\textbf{File}: \texttt{02\_service\_layer\_architecture.tex}
\begin{itemize}
    \item Microservices design patterns
    \item Business logic orchestration
    \item Inter-service communication protocols
    \item Service discovery and load balancing
\end{itemize}

\subsection{Section 3: API Gateway \& Route Architecture}
\textbf{File}: \texttt{03\_api\_gateway\_architecture.tex}
\begin{itemize}
    \item RESTful API design principles
    \item GraphQL integration strategies
    \item Rate limiting and throttling
    \item API versioning and documentation
\end{itemize}

\subsection{Section 4: Security \& RBAC Architecture}
\textbf{File}: \texttt{04\_security\_rbac\_architecture.tex}
\begin{itemize}
    \item Role-based access control implementation
    \item Authentication and authorization flows
    \item Security policy enforcement
    \item Compliance and audit mechanisms
\end{itemize}

\subsection{Section 5: Data Discovery \& Scanning Architecture}
\textbf{File}: \texttt{05\_data\_discovery\_architecture.tex}
\begin{itemize}
    \item Intelligent scanning algorithms
    \item Multi-source data connectors
    \item Pattern recognition and classification
    \item Real-time discovery pipelines
\end{itemize}

\subsection{Section 6: Catalog \& Metadata Management Architecture}
\textbf{File}: \texttt{06\_catalog\_metadata\_architecture.tex}
\begin{itemize}
    \item Metadata storage and indexing
    \item Search and discovery interfaces
    \item Data lineage tracking
    \item Catalog collaboration features
\end{itemize}

\subsection{Section 7: Compliance \& Governance Architecture}
\textbf{File}: \texttt{07\_compliance\_governance\_architecture.tex}
\begin{itemize}
    \item Compliance rule engine design
    \item Policy management frameworks
    \item Automated compliance checking
    \item Governance workflow orchestration
\end{itemize}

\subsection{Section 8: AI/ML Integration Architecture}
\textbf{File}: \texttt{08\_ai\_ml\_integration\_architecture.tex}
\begin{itemize}
    \item Machine learning pipeline design
    \item AI-powered data classification
    \item Predictive analytics integration
    \item Model training and deployment
\end{itemize}

\subsection{Section 9: Frontend \& User Experience Architecture}
\textbf{File}: \texttt{09\_frontend\_ux\_architecture.tex}
\begin{itemize}
    \item React-based component architecture
    \item State management strategies
    \item Real-time UI updates
    \item Responsive design principles
\end{itemize}

\subsection{Section 10: Integration \& Cloud Architecture}
\textbf{File}: \texttt{10\_integration\_cloud\_architecture.tex}
\begin{itemize}
    \item Multi-cloud integration strategies
    \item Azure Purview connectivity
    \item Data pipeline orchestration
    \item Hybrid cloud deployment models
\end{itemize}

\subsection{Section 11: Performance \& Scalability Architecture}
\textbf{File}: \texttt{11\_performance\_scalability\_architecture.tex}
\begin{itemize}
    \item System performance optimization
    \item Horizontal scaling strategies
    \item Caching and data optimization
    \item Load balancing and distribution
\end{itemize}

\subsection{Section 12: Deployment \& DevOps Architecture}
\textbf{File}: \texttt{12\_deployment\_devops\_architecture.tex}
\begin{itemize}
    \item Containerization strategies
    \item CI/CD pipeline design
    \item Infrastructure as Code
    \item Monitoring and logging systems
\end{itemize}

\section{Architecture Quality Attributes}

\subsection{Scalability}
The system is designed to handle:
\begin{itemize}
    \item 10,000+ concurrent users
    \item Petabyte-scale data processing
    \item Real-time processing of millions of events per day
    \item Horizontal scaling across multiple data centers
\end{itemize}

\subsection{Performance}
Key performance targets:
\begin{itemize}
    \item API response times < 200ms for 95\% of requests
    \item Data discovery completion within minutes for large datasets
    \item Real-time compliance checking with < 1 second latency
    \item High-throughput data ingestion (10GB/s+)
\end{itemize}

\subsection{Security}
Comprehensive security measures:
\begin{itemize}
    \item End-to-end encryption for all data
    \item Multi-factor authentication
    \item Fine-grained access controls
    \item Comprehensive audit logging
    \item Regular security assessments and penetration testing
\end{itemize}

\subsection{Reliability}
System reliability features:
\begin{itemize}
    \item 99.9\% uptime SLA
    \item Automatic failover and recovery
    \item Data backup and disaster recovery
    \item Circuit breaker patterns for fault tolerance
\end{itemize}

\section{Technology Stack Overview}

\subsection{Backend Technologies}
\begin{itemize}
    \item \textbf{Programming Language}: Python 3.11+
    \item \textbf{Web Framework}: FastAPI for high-performance APIs
    \item \textbf{Database}: PostgreSQL with advanced indexing
    \item \textbf{Message Queue}: Redis for caching and pub/sub
    \item \textbf{Search Engine}: Elasticsearch for full-text search
    \item \textbf{ML Framework}: TensorFlow/PyTorch for AI capabilities
\end{itemize}

\subsection{Frontend Technologies}
\begin{itemize}
    \item \textbf{Framework}: React 18 with TypeScript
    \item \textbf{State Management}: Redux Toolkit
    \item \textbf{UI Components}: Custom design system with Tailwind CSS
    \item \textbf{Build Tool}: Vite for fast development
    \item \textbf{Testing}: Jest and React Testing Library
\end{itemize}

\subsection{Infrastructure \& DevOps}
\begin{itemize}
    \item \textbf{Containerization}: Docker and Kubernetes
    \item \textbf{Cloud Platforms}: Azure, AWS, Google Cloud
    \item \textbf{CI/CD}: GitLab CI/CD or GitHub Actions
    \item \textbf{Monitoring}: Prometheus, Grafana, ELK Stack
    \item \textbf{Infrastructure as Code}: Terraform
\end{itemize}

\section{Next Steps}

To fully understand the DataWave architecture, proceed through each section in order. Each section provides:

\begin{enumerate}
    \item Detailed component analysis
    \item Advanced Mermaid diagrams
    \item Implementation specifications
    \item Best practices and patterns
    \item Performance considerations
    \item Security implications
\end{enumerate}

This modular approach ensures comprehensive coverage of all architectural aspects while maintaining clarity and depth in each domain.

\section{Conclusion}

The DataWave Enterprise Data Governance System represents a state-of-the-art approach to modern data governance challenges. Its architecture combines proven patterns with innovative solutions to deliver a scalable, secure, and maintainable platform capable of meeting the most demanding enterprise requirements.

The following sections provide the detailed architectural analysis necessary for successful implementation and maintenance of this advanced system.

\end{document}